% !TeX root=../main.tex

\section{Conclusions}
%\texttt{\color{red}[TODO]}
\label{sec:conclusion}

We have presented an approach to probabilistic trace alignment via $k$NN.
The approach balances between the likelihood of the aligned trace and the cost of the alignment by providing the top-k 
alignments instead of a single alignment as output. Experiments show that the approximated top-k ranking provides a good 
trade-off between accuracy and efficiency especially when the reference model generates several traces.
Future works will investigate the probabilistic alignment over fuzzy-labeled nodes and declarative process models, thus 
generalizing the proposed approach to noisy walks in plan recognition for non-rational agents \cite{RamirezG10}. We will also improve 
the efficiency and accuracy of our approach by intervening on the embedding and the algorithmic strategies, such as extending 
the Viterbi Algorithm  to perform $k$ candidates instead of the one maximizing probability and similarity value. Last, we will also 
investigate the possibility of representing POMDPs as TGs when the reward function is completely determined by the distance 
between traces' actions.

%\section*{Acknowledgements}
%This research has been partially supported by the project IDEE (FESR1133) funded by the Eur.\ Reg.\ Development Fund (ERDF) 
%Investment for Growth and Jobs Programme 2014-2020. 