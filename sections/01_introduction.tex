\section{Introduction}\label{introduction}


In the existing literature on conformance checking, one of the most common approach is based on trace alignments. However, the proposed approaches based on trace alignments use crisp process models as reference models. On the other hand, recently, probabilistic conformance checking approaches are gaining momentum, but the existing approaches are used to check the degree of conformance of and event log with respect to a stochastic process model instead of finding trace alignments.
In this paper, for the first time, we provide a conformance checking approach based on trace alignments using stochastic reference models. Conceptually, this requires to handle the two possibly contrasting forces of the cost of the alignment on the one hand and the likelihood of the model trace with respect to which the alignment is computed.
Balancing between the likelihood of the model trace with respect to which we are computing the alignment and the cost of the alignment (if the cost of the alignment is too high even if the model trace is very likely applying too many changes in the original trace is in turn not very likely).




%%%%% Proposed part as the last part of the introduction: 
\texttt{\color{red}[TODO]}
Then, we're going to perform experiments for empirically evaluating some of the properties of the proposed embedding strategy (\S\ref{sec:exp}):
\begin{enumerate}
	\item We're going to assess which is the degree of the trace alignment approximation induced by the PPN kernel $k_{\mathcal{P}_\phi}$ (Lemma \ref{lem:approxRank}) if compared to the exact $k$-probabilistic traces alignment (\S\ref{subsec:exbkptap}). We propose a novel scoring function for comparing the two rankings alternatives induced by the alignment functions (\S\ref{sec:exp}). \texttt{\color{red}[TODO]}
	\item We're going to assess the proposed kernel's appropriateness (\S\ref{subsec:katk}) by taking all the traces $\tau$ generated by a Probabilistic Petri Net $P$, add some noise to $\tau$ thus causing noised traces $\tilde{\tau}$, and evaluating the distance between the trace ranking of $\tau$ and $\tilde{\tau}$ over the traces of $P$. \texttt{\color{red}[TODO]}
	\item Last, we're going to compare the time required to compute the exact trace alignment approach (\S\ref{subsec:exbkptap}) against the approximated approach (\S\ref{subsec:akptap}). \texttt{\color{red}[TODO]}
\end{enumerate}
\texttt{\color{red}[TODO]}