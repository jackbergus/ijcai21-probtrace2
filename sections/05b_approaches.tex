\section{Use Case Scenarios}
\subsection{Conformance Checking}
A business process describes (possibly concurrent) sequences of actions $s$ carried out by at least one agent for pursuing a final goal. The sequentialization of each transaction can be described as a \textit{trace} $\sigma$, where different \textit{labels} $\lambda(s)$ are used to distinguish distinct actions. A \textit{log} $\mathcal{L}$ is a set of traces, that can be compactly described using two distinct types of models $\mathcal{D}$: while \textit{procedural models} \cite{DBLP:journals/tosem/PolyvyanyySWCM20} %DBLP:conf/bpm/LeemansSA19,DBLP:conf/icpm/PolyvyanyyK19,
 enlist all the desired traces by unravelling,  \textit{declarative models} \cite{BurattinMAS12} offer the logical constraints that any desired trace should satisfy.  Last, multi-perspective approaches considers property-value associations within the traces' data by extending such models with additional logical predicates \cite{GiacomoMGMM14} or ``guards'' \cite{MannhardtLRA16}. Given that any declarative model $\mathcal{D}$ can be described by a set of log-traces $\mathcal{P}_\mathcal{D}\subseteq\mathcal{L}$ satisfying the constraints in $\mathcal{D}$ \cite{LeoniMA12}, and that $\mathcal{P}_\mathcal{D}$ might generate a procedural model, any declarative model can be transformed to a procedural model. Therefore, our disquisition will only consider procedural models: without loss of generality, procedural models are described by Stochastic Petri Nets \cite{MarsanCB84}: each unravelled trace is associated to a probability score estimating its occurrence frequency within the original log. In fact, Stochstic Petri Nets can be generated from BPMN models \cite{RaedtsPUWGS07} via a distribution estimator \cite{spdwe}. 

Within the context of business processes, \textit{conformance checking} aims at assessing the degree of deviance of a log $\mathcal{L}$ towards the mined model $\mathcal{D}$ and vice-versa. Such deviance is often reduced to a  trace alignment problem \cite{DBLP:conf/edoc/AdriansyahDA11}, where each log-trace $\sigma'\in\mathcal{L}$ is compared to a trace $\sigma$ compatible with the model. Alignments empower explainability, as they describe the repairs required to make $\sigma'$ conformant to $\sigma$, and provides a numerical description of  their similarity.  Despite probabilistic conformance checking approaches are gaining importance \cite{DBLP:conf/bpm/LeemansSA19,DBLP:conf/icpm/PolyvyanyyK19,DBLP:journals/tosem/PolyvyanyySWCM20}, these do not exploit trace alignments techniques, thus failing at explainabilty. Given that it is always possible to sequentialize Stochastic Petri Nets as Transition Graphs \cite{MarsanCB84}, we can describe each trace from the procedural model via $\overline{G}$ projection over traces, thus validating the usage of the proposed computation pipeline as a whole: an alignment of a trace with a procedural model is now the model trace maximizing the combined provision of minimum trace alignment cost and maximum model trace probability.



\subsection{Biological Sequence Analysis}
Similarly to conformance checking, biological sequence analysis assumes that sequence homology can be reduced to aligning the sequences to be compared after representing those as strings \cite{durbin1998biological}. When sequence similarity is weak, it is hard to identify the most significant alignment to prefer. In such situations, the sequence similarity is often combined with the probability for better ranking the preferred alignments \cite{durbin1998biological}. This is a shared point with our proposed approach. 

Biological sequences of interest, such as CpG islands \cite{kxq005} or full DNA sequences \cite{BISHOP1986159}, can be summarized into a probabilistic model, either via node-labelled Discrete-Time Markov Chains \cite{RyabkoU08} or Hidden Markov Models \cite{Helske2018}. Given that it is always possible to represent the latter as the former \cite{DUPONT20051349} and given that the former can be easily represented as Transition Graphs, we will only consider DTMC. The probability of reaching a state $s_t$ with a given label $\lambda(s)$ after visiting $t$ other states only depends on the previously visited state $s_{t+1}$, thus making such models memoryless. Therefore, every biological sequence's probability is the joint probability of the transition probabilities required to visit the multigraph associated to the DTMC for generating the sequence. The probability associated with each biological sequence reflects the likelihood of having the final trace within the set of traces of interest. 

Such techniques often exploit well-known algorithms for returning the probabilistic model's best alignment towards the sequence of interest. E.g., the Viterbi Algorithm \cite{Viterbi67} returns the probabilistic model's path optimizing both trace probability and sequence similarity.  Nevertheless, the user could prefer a sequence with more significant similarity albeit underrepresented within the probabilistic model, while, in other cases, the user could favor a model trace with a higher probability at the expense of a lower homology. Thus, such an algorithm becomes unideal. In such situations, we problem still reduces to first enumerating all the more probable model traces for then skimming out the less likely to be candidate alignments for the given trace. This strategy is indeed the one pursued by our proposed approach. 


\subsection{Plan Recognition}