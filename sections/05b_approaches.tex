\section{Use Case Scenarios}
\subsection{Conformance Checking}
A business process describes (possibly concurrent) sequences of actions carried out by at least one agent for pursuing a final goal. Every sequence can be described as a \textit{trace} $\sigma\in\tasks^*$, where different \textit{labels} $\const{a}\in\tasks$ are used to distinguish actions.
%Within the context of business processes,
\textit{Conformance checking} assesses the degree of deviance of a log $\mathcal{L}$ towards a process model $\mathcal{D}$ and vice-versa. Such deviance is often reduced to a trace alignment problem \cite{DBLP:conf/edoc/AdriansyahDA11}, where each log-trace $\sigma'\in\mathcal{L}$ is compared to a model trace $\sigma$. Alignments empower explainability, as they describe the repairs required to make $\sigma'$ conformant to $\sigma$, and provides a numerical description of their similarity.

%A finite \textit{log} $\mathcal{L}$ can be compactly represented as a model $\mathcal{D}$ via process  mining. 
Each model $\mathcal{D}$ describes a language $L[\mathcal{D}]$. We can fit such models into two main categories: while \textit{procedural models}, like Peri nets, enlist all the desired traces by unfolding, \textit{declarative models} offer the logical constraints that any desired trace should satisfy. Declarative models can be described via temporal logic rules, which can be totally described by node-labeled graphs \cite{GiacomoMM14}. Although the induced language is a countably infinite set of traces, it is always possible to prune it down to a finite set  by enumerating the candidate traces via an A* algorithm \cite{LeoniMA12}, and then representing the remaining traces directly as Transition Graphs in Fig. \ref{fig:taustar}. Henceforth, string alignments will not consider transition probabilities. Similar considerations can be also carried out for procedural models, as Petri nets can be described by node-labeled graphs. %described by Petri Nets.

Multi-perspective approaches additionally consider property-value associations within the traces by extending process models with additional conditions, namely guards \cite{MannhardtLRA16}. After enumerating all the possible model guards $\mathbf{P}$, multi-perspective conformance checking can be always reduced to the non multi-perspective solutions by re-labeling each $\const{a}$-labeled trace state satisfying $P_i\in\mathbf{P}$ as $\const{a}{P_i}$ and applying a similar relabeling in the graph representation associated to the model.


Last, probabilistic conformance checking is gaining importance \cite{DBLP:conf/bpm/LeemansSA19,DBLP:conf/icpm/PolyvyanyyK19,DBLP:journals/tosem/PolyvyanyySWCM20}: such solutions are only assessing the degree of conformance of a log with respect to a stochastic Petri net without providing explainable results. Still, we can always formulate such problem into a probabilistic string alignment problem via our computation pipeline. Indeed, stochastic Petri nets jointly with firing weight estimators \cite{spdwe} can be sequentialized as probabilistic TGs via reachability graphs (Fig. \ref{fig:lmc}), thus enabling alignments via $\overline{G}$ projection over traces. An alignment of a trace with a stochastic Petri net is now the model trace maximizing the combined provision of minimum trace alignment cost and maximum model trace probability.






%Despite probabilistic conformance checking approaches are gaining importance \cite{DBLP:conf/bpm/LeemansSA19,DBLP:conf/icpm/PolyvyanyyK19,DBLP:journals/tosem/PolyvyanyySWCM20}, these do not exploit trace alignments techniques, thus failing at explainabilty. Given that it is always possible to sequentialize Stochastic Petri Nets as Transition Graphs \cite{MarsanCB84}, we can describe each trace from the procedural model via $\overline{G}$ projection over traces, thus validating the usage of the proposed computation pipeline as a whole: an alignment of a trace with a procedural model is now the model trace maximizing the combined provision of minimum trace alignment cost and maximum model trace probability.


