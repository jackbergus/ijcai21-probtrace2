\section{Use Case Scenarios}
We review different scenarios that fit our approach.

\smallskip
\noindent
\textbf{Conformance Checking in BPM.}
Within the BPM field, a process describes how a set of activities has to be executed in coordination to achieve a business goal. Process executions lead to traces consisting of sequences of activities that, in principle, should agree with the constraints imposed by the process.
\textit{Conformance checking} assesses whether this is indeed the case \cite{CDSW18}. Specifically, it aims at computing the degree of deviation between a log $\mathcal{L}$ and a process $\mathcal{D}$. This is often reduced to trace alignment \cite{DBLP:conf/edoc/AdriansyahDA11}, where each log-trace $\sigma'\in\mathcal{L}$ is related to the ``closest' model trace $\sigma$. Alignments \emph{explain} which changes must be applied on $\sigma'$ to transform it into $\sigma$.%., and provides a numerical description of their similarity.
%A finite \textit{log} $\mathcal{L}$ can be compactly represented as a model $\mathcal{D}$ via process  mining. 
Each process $\mathcal{D}$ induces a language $L[\mathcal{D}]$.
%We can fit such models into two main categories: while \textit{procedural models}, like Petri nets, yield all the model traces by unfolding, \textit{declarative models} implicitly describe model traces as all those traces that satisfy a set of temporal constraints. Declarative models can be described via temporal logic rules, which can be fully described by node-labeled graphs \cite{GiacomoMM14}. 
Even though $L[\mathcal{D}]$ may contain infinitely many traces, it is possible to prune the set down to a finite set by enumerating the candidate traces via an A* algorithm \cite{LeoniMA12}, and then representing the remaining traces directly as TGs of the form shown in Fig. \ref{fig:taustar}. 
%Henceforth, string alignments will not consider transition probabilities. %Similar considerations can be also carried out for procedural models, as (bounded) Petri nets can be described by node-labeled graphs. %described by Petri Nets.

%Multi-perspective approaches additionally consider property-value associations within the traces by extending process models with additional conditions, namely guards \cite{MannhardtLRA16}. After enumerating all the possible model guards $\mathbf{P}$, multi-perspective conformance checking can be always reduced to the non multi-perspective solutions by re-labeling each $\const{a}$-labeled trace state satisfying $P_i\in\mathbf{P}$ as $\const{a}{P_i}$ and applying a similar relabeling in the graph representation associated to the model.

Adding probabilities into the picture leads to 
\emph{probabilistic conformance checking}\cite{DBLP:conf/icpm/PolyvyanyyK19}. Solutions in this spectrum typically depart from alignments and compare the entire log with a probabilistic process, usually represented as a (bounded) stochastic Petri net, without providing explainable results. We can improve on this by  comparing single traces to the probabilistic language generated by a stochastic Petri net, in the form of a probabilistic string alignment problem via our computation pipeline. Indeed, stochastic Petri nets  can be sequentialized as probabilistic TGs via reachability graphs (Fig. \ref{fig:lmc}), thus enabling alignments via $\overline{G}$ projection over traces. An alignment of a trace with a stochastic Petri net is now the model trace maximizing the combined provision of minimum trace alignment cost and maximum model trace probability.






%Despite probabilistic conformance checking approaches are gaining importance \cite{DBLP:conf/bpm/LeemansSA19,DBLP:conf/icpm/PolyvyanyyK19,DBLP:journals/tosem/PolyvyanyySWCM20}, these do not exploit trace alignments techniques, thus failing at explainabilty. Given that it is always possible to sequentialize Stochastic Petri Nets as Transition Graphs \cite{MarsanCB84}, we can describe each trace from the procedural model via $\overline{G}$ projection over traces, thus validating the usage of the proposed computation pipeline as a whole: an alignment of a trace with a procedural model is now the model trace maximizing the combined provision of minimum trace alignment cost and maximum model trace probability.


