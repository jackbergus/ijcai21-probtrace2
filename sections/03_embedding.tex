\begin{figure}[!t]
	\hspace*{-4cm}\includegraphics[width=1.7\textwidth]{images/pipeline}
	\caption{Proposed pipeline to assess the Probabilistic Trace Alignment}\label{fig:pipe}
\end{figure}


\section{Probabilistic Trace Alignment Pipeline}
This paper proposes the pipeline provided in Figure \ref{fig:pipe} by connecting several formalisations given in current literature via intermediate processing steps. The input of this pipeline is a query trace $\tau^*$ to be aligned, an USWN, and a minimum probability threshold $p_\theta$. The output of this pipeline is a set of model traces satisfying $p_\theta$ with an alignment ranking.

The pipeline is composed by the following phases: after representing the USWN as a graph of all the sequentially scheduled transitions (\S\ref{sec:seqZ}), we shift the labels from the edges towards the nodes while preserving the set of probabilistic traces (\S\ref{sec:LSift}) and minimize the graph representation by removing the $\varepsilon$-labelled nodes while preserving the traces' probability (\S\ref{sec:clos}). We extract the set of all the traces having $p_\theta$ as a minimum probability threshold (\S\ref{sec:unrav}) over which we  apply two different alignment strategies, the exact one (\S\ref{subsec:eta}) and the approximated one. 


Last, we are going to discuss how we can rank such traces in both the exact and the approximated scenario by reducing the alignment process to a k-nearest neighbour problem. While the exact trace alignment scenario requires to perform the alignment process each time a novel trace $\tau^*$ is introduced (\S\ref{subsec:exbkptap}), the approximated alignment allows to split the alignment into a preliminar loading phase and a query phase; in the former, each stochastic trace from the USWN is represented as a vector (\S\ref{subsec:ate}), and in the latter the to-be-aligned trace $\tau^*$ is first represented as a vector and then compared to all the other vectorial representations.  

\subsection{Sequentialization}\label{sec:seqZ}
In the sequentialization step, we transform a USWN with an initial marking $M$ into a Reachability Graph $(\mathcal{M},\mathcal{E})$ that is generated by the sequentialization process, where the potentially concurrent firing transitions are represented via a sequential scheduling. 

\begin{definition}[Reachability Graph]
	Given an initial marking $M$ for a USWN $\mathcal{U}$,  the \textit{Reachability Graph} for $\mathcal{U}$ is a graph $(\mathcal{M},\mathcal{E})$ where the nodes  $\mathcal{M}$ are composed of of all the reachable markings from $M$, $M$ included, and all the edges $\mathcal{E}$ are induced by the aforementioned relation $M\overset{t}{\to}M'$ among the reachability graph's nodes. To each of such edges $M\overset{t}{\to}M'$, we associate a transition probability $\mathbb{P}\left(M\overset{t}{\to}M'\right)=\frac{W(t)}{\sum_{t'\in E(M)}W(t')}$ \cite{spdwe}. 
\end{definition}

\begin{example}
Given a generic USWN as in Figure \ref{fig:spn}, the Sequentialization process generates a Reachability Graph depicted in Figure \ref{fig:rg}: each node represents a marking $M$ as a vector, while the edges are labelled with the firing transitions. As we can see, the edges associated to this graph describes potentially concurrent firing transitions sequentially. While visiting the graph from $M$, the chaining of the edges' labels generates a trace generated from the Untimed Workflow Net, and the multiplication of the edges' weight provides the probability associated to the trace.
\end{example}



\subsection{Label Shifter}\label{sec:LSift}
Reachability Graphs generated via Sequentialization cannot be directly embedded using the embedding strategy proposed in current literature (\S\ref{ssec:ge}):  Reachability Graphs associated to Stochastic Workflow Nets are edge labelled, while Transition Graphs are Node Labelled. In order to represent the former as the latter, we need to shift the labels from the edges towards the nodes  while guaranteeing that such transformation preserves the set of the traces, as well as their associated probability. We provide such desired transformation in the following definition:

\begin{definition}[Label Shifter]\label{def:transf}
Given a Reachability Graph $(\mathcal{M},\mathcal{E})$ generated from an initial marking $M$, we can transform it into a Transition Graph $(s,t,L,R,1)$, where:
\begin{itemize}
	\item If there exist one single edge $M_1\overset{t}{\to}M_2\in\mathcal{E}$ where $M_1=M$, then $s=M\overset{t}{\to}M_2$; otherwise, we define a new node $\textbf{i}$ and we set it as the initial node for TG: $s=\textbf{i}$.
	\item If there exist one single edge $M_1\overset{t}{\to}M_2\in\mathcal{E}$ having no outgoing edges in the Reachability Graph, then $t=M_1\overset{t}{\to}M_2$; otherwise, we define a new node $\textbf{f}$ and we set it as the accepting node for TG:  $t=\textbf{f}$.
	\item $[L]_{\lambda(t),\;M\overset{t}{\to} M'}=1$ for each $M\overset{t}{\to} M'\in\mathcal{E}$; if $\textbf{i}$ is defined then $[L]_{\varepsilon\textbf{i}}=1$; if $\textbf{f}$ is defined, then $[L]_{\varepsilon\textbf{f}}=1$; $[L]_{ij}=0$ otherwise.
	\item $[R]_{M\overset{t}{\to} M',\;M'\overset{t'}{\to} M''}=\frac{W(t')}{\sum_{\textbf{t}\in E(M')}W(\textbf{t})}$ for each $M\overset{t}{\to} M',M'\overset{t'}{\to} M''\in\mathcal{E}$; if $\textbf{i}$ is defined, then $[R]_{\textbf{i},\;M\overset{t}{\to}M'}=\frac{W(t)}{\sum_{\textbf{t}\in E(M)}W(\textbf{t})}$; if $\textbf{f}$ is defined, then $[R]_{M\overset{t}{\to}M',\;\textbf{i}}=1$ for each $M'$ having no outgoing edges in the Reachability Graph; $[R]_{ij}=0$ otherwise.
\end{itemize}
\end{definition}

We can also show that the Transition Graph obtained in Definition \ref{def:transf} preserves the same set of probabilistic traces associated by the Reachability Graph, but the proof of such claim  is omitted due to the lack of space.

\begin{example}
Figure \ref{fig:lmc} provides the output of such transformation if Figure \ref{fig:rg} is used as an input. All the nodes are labelled using the firing transitions' labels (in green), while the edges preserve the probabilistic information from the Reachability Graph (in red). Intuitively, when a new initial node \textit{\textbf{i}} is inserted, we preserve all the initial probabilistic choices that a transition is fired from an initial marking $M$, while all the intermediate edges inherit the probabilisitc choice of the firing transition from the subsequent choices. When a new final node \textit{\textbf{f}} is added, such edges always have probability $1$, and therefore we do not interfere with the initial traces' probability.
\end{example}

\subsection{$\varepsilon$-closure}\label{sec:clos}
The $\varepsilon$-closure process has two main purposes: first, it reduces the size of the Transition Graph generated in the previous step by removing all the $\varepsilon$-labelled nodes \texttt{\color{blue}w} and preserving the connection between  the nodes \texttt{\color{blue}u} from its ingoing edges   $\texttt{\color{blue}u}\xrightarrow{\color{red}p_i}\texttt{\color{blue}w}$ with the nodes \texttt{\color{blue}v} from its ingoing edges   $\texttt{\color{blue}w}\xrightarrow{\color{red}p_j}\texttt{\color{blue}v}$ by establishing new edges $\texttt{\color{blue}u}\xrightarrow{\color{red}p_ip_j}\texttt{\color{blue}v}$. $\varepsilon$-labelled initial (or accepting) nodes are removed if and only if they have only one outgoing (ingoing) edge with probability $1$.

\begin{example}
	The $\varepsilon$-closure remotes the non-initial and non-accepting nodes within such automaton, while preserving the probabilistic trace equivalence of the two automata: node \texttt{\color{blue}10} is then removed alongside its associated edges, and new edges $\texttt{\color{blue}3}\xrightarrow{\color{red}p_4}\texttt{\color{blue}4}$ and $\texttt{\color{blue}3}\xrightarrow{\color{red}p_5}\texttt{\color{blue}5}$ are introduced. The resulting TG $P$ is represented with the same graphical depiction Figure \ref{fig:closed}.
\end{example}

Consequently, it is always possible to minimize a TG  (e.g., Figure \ref{fig:orig}) via $\varepsilon$-closure, so that the only nodes that are labelled as $\varepsilon$ are the source and the target nodes (Figure \ref{fig:closed}) and the set of weighted traces is preserved. From now on, we always that all the TGs are minimized via $\varepsilon$-closure. 

\subsection{Unraveller}\label{sec:unrav}
Being that both the graph isomorphism problem is NP-Complete and the TGs are fully characterized by the set of the probabilistic traces that they generate,  we can say that two TGs are (probabilistic-trace) equivalent if and only if they share the same set of weighted traces. In particular, we denote as $\mathcal{W}_p^n(P)$ the set of all the weighted traces in $P$ having at least probability $p$ and maximum length $n$. Under these assumptions, the probabilistic trace equivalence is deterministic.

\begin{example}
	The TG in Figure \ref{fig:orig} has the following set $\mathcal{W}_0^{\aleph_0}(P^*)$ of weighted traces:
$$\set{\braket{\underbrace{\color{green}a\dots a}_{n},{\color{red}p_1p_3^np_6}}|n\in \mathbb{N}_{>0}}\cup \set{\braket{{\color{green}c}\underbrace{\color{green}a\dots a}_{n},{\color{red}p_2p_4p_3^np_6}}|n\in \mathbb{N}_{>0}}\cup\{\braket{{\color{green}cb},{\color{red}p_2p_5}}\}$$
After the $\varepsilon$-closure process, $\mathcal{W}_0^{\aleph_0}(P^*)=\mathcal{W}_0^{\aleph_0}(P)$, so the two TGs are (probabilistic-trace) equivalent.
\end{example}
