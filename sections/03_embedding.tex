\section{Probabilistic Petri Net Embedding}
Graph kernels allow mapping graph data structures to feature spaces (usually an Eulcidean space in $\mathbb{R}^n$ for $n\in \mathbb{N}_{>0}$) \cite{Samatova} so to express graph similarity functions that can then be adopted for both classification \cite{TsudaS10} and clustering \cite{Raedt} algorithms. One of the first approaches used in literature involved the definition of topological description vectors \cite{Sidere} for each graph in a graph database, for then defining the graph similarity function as an inner product of their associated vectors. One inconvenience of such a technique is that it is required to perform \yellownote{Not sure if it is better to move this part into a Related Work section, but I think we still need to discuss the structure of the paper.} NP-complete subgraph isomorphisms among a collection of graphs. It has been already proved that the definition of a graph kernel function fully recognizing the structure the graph always boils down solving such  NP-Complete problem \cite{GartnerFW03}. 


Consequently, most recent literature focused on extracting relevant features of such graphs, that are then used to define a graph similarity function. The most common approach adopted in kernel to extract such features is called \textit{propositionalisation}: we might extract all the possible features (e.g., subsequences), and then define a kernel function based on the occurrence and similarity of these features \cite{Gartner03}. 

Given that we previously observed that a PPNs $P$ can be fully characterised (read, similarity) by their associated set of traces $\mathcal{W}_p^n(P)$ (\S\ref{subsec:ppn}), and that the trace embedding can be described as an embedding over a PPN (\S\ref{subsec:katk}), we can characterize a PPN embedding as a transition matrix embedding. In addition to that, when two Petri Nets share similar node labellings but no ${\color{green}\alpha}\rightsquigarrow{\color{green}\beta}$ paths for any ${\color{green}\alpha}$ and ${\color{green}\beta}$, we should combine the former embedding with an embedding characterizing the frequence on how the nodes' labels appear in the generated traces. We can now provide the following definition:

\begin{definition}[PPN Embedding]\label{def:ppne}
Given a finite set of non-empty labels $\Sigma_\varepsilon =\Sigma\backslash\{\varepsilon\}$, $\Sigma_\varepsilon^2$ denotes all the possible pair of labels associated to paths ${\color{green}\alpha}\rightsquigarrow{\color{green}\beta}$ and $\Sigma_\varepsilon$ denotes the set of all the possible non-$\varepsilon$ node labels. Therefore, it is always possible to enumerate $\Sigma_\varepsilon^2\cup\Sigma_\varepsilon$ via an indexing function $\iota\colon \Sigma_\varepsilon^2\cup\Sigma_\varepsilon\to  \mathbb{N}$.
	
Given a PPN $P=(s,t,L,R,w)$ resulting from a $\varepsilon$-closure and a tuning parameter $\tau\in[0,1]\subseteq\mathbb{R}$, the associated embedding is defined as follows:
$$\phi_{\mathcal{P}}(P)_i=\begin{cases}
	\frac{\epsilon_{\color{green}\alpha\beta}}{\|\epsilon\|_2}w\tau^{|R>0|} & i={\color{green}\alpha\beta}\\
	\frac{\nu_{\color{green}\alpha}}{\|\nu\|_2}\tau^{|R>0|} & i={\color{green}\alpha}\\
\end{cases}$$
where $\epsilon$ and $\nu$ respectively represent the non-negatively defined embedding associated to the Petri Net's transition matrix and nodes. 
\end{definition}

Albeit the definition of $\epsilon$ and $\nu$ might vary, we choose a specific definition for them. 
Walking on the footsteps of \S\ref{subsec:katk}, we can define the transition matrix embedding as  $\epsilon(R)_{\color{green}\alpha\beta}=\sum_{i=1}^n\lambda^i[\Lambda^i]_{\color{green}\alpha\beta}$ and $\nu(R,L)_{\color{green}\alpha}=\frac{1}{N}\sum_{\braket{\tau,w}\in\mathcal{W}_p^n(P)}\frac{|\Set{\tau_i\in\tau|\tau_i\neq\varepsilon\wedge \tau_i={\color{green}\alpha}}|}{|\tau|}$, where $N$ is a normalization factor such that $\sum_{{\color{green}\alpha}\in\Sigma_\varepsilon}\nu(R,L)_{\color{green}\alpha}=1$. When the transition matrix is ergodic \cite{StocasticCC},  the transition matrix embedding converges to $\epsilon(R)_{\color{green}\alpha\beta}=(\mathbf{I}-\lambda[\Lambda])^{-1}_{\color{green}\alpha\beta}$ \cite{GartnerFW03} for $n\to+\infty$.

\begin{example}
Given the PPN in Figure \ref{fig:closed}, we assign the following probability values to the edges: $p_1=p_2=0.5$, $p_3=0.9$, $p_6=1$, $p_4=0.3$, and $p_5=0.7$. Given $\Sigma_\varepsilon=\Set{a,b,c}$ and $V=\set{1,2,3,5,6}$, we have that all the embedding space should be at least of size $6$, as $\{a,b,c,aa,ca,cb\}\subseteq \Sigma_\varepsilon^2\cup\Sigma_\varepsilon$ is the whole set of features describing both the transition matrix and the nodes. By choosing $\tau=0.0001$, $\lambda=0.7$, $n=4$, and $p=0$, we generate the following set:
$$\begin{aligned}
\mathcal{W}_0^4(P)=\{&\braket{cb,0.35},\braket{a,0.05},\braket{aa,0.045},\braket{aaa,0.0405},\\
&\braket{aaaa,0.03645},\braket{ca,0.015},\braket{caa,0.0135},\braket{caaa,0.01215}\}\\
\end{aligned}$$
The embedding associated to $P$ is the following:
\begin{center}
		\begin{tabular}{l|l|l|l|l|l|l|}
		\toprule
		& a    & b                                                   & c    & aa   & ca   & cb   \\
		\midrule
		$\phi_{\mathcal{P}}(P)$ & $9.94\cdot10^{-25}$ & $1.18\cdot 10^{-25}$ & $1.04\cdot10^{-25}$ & $6.87\cdot 10^{-25}$ & $2.29\cdot10^{-25}$ & $1.40\cdot10^{-25}$\\
		\bottomrule
	\end{tabular}
\end{center}
This table shows that doing\yellownote{Not sure whether this example is sufficiently clear or something simpler/more clear can be done. Then, I will add further examples.} ${\color{green}a}\rightsquigarrow{\color{green}a}$ is more probable than doing  ${\color{green}c}\rightsquigarrow{\color{green}a}$. Also, even though the probability of performing  ${\color{green}c}\overset{1}{\rightsquigarrow}{\color{green}b}$ is relatively high,  the trace $\color{green}cb$ is relatively unfrequent within the set of traces for $P$, and therefore it is less probable than any other subtrace. 

If we now consider the single nodes, $\color{green}c$ shares a subset of traces with $\color{green}a$ where $\color{green}a$ is more frequent than $\color{green}c$, and therefore the score of the former is higher than the one of the latter. Also, the score associated to the single node $\color{green}b$ is higher than the one for the single node $\color{green}c$ because $\color{green}b$ is overall more frequent than $\color{green}c$:  $\color{green}b$ appears once in just one trace of length $2$, while $\color{green}c$ appears once per trace of lengths from 2 to 4 so, overall, $\color{green}b$ is considered more frequent than $\color{green}c$.
\end{example}

After providing this definition, we can show that this embedding shows some desired features that are independent from the definition of $\epsilon$ and $\nu$, and that $\epsilon$ and $\nu$ only depend on the characterization of both the labelling $L$ and the transition matrix $R$. We provide a rewriting proposition that is going to be used in the incoming subsection to allow proving the aforementioned characterizing properties.

\begin{proposition}\label{lem:rewritinglemma}
Given two PPNs $P=(s,t,L,R,w)$ and $P'=(s',t',L',R',w')$, the PPN Kernel is defined as follows:
$$\begin{aligned}
k_{\phi_{\mathcal{P}}}(P,P')=&ww'\tau^{|R>0|+|R'>0|}\left(1-\frac{\norm{\hat{\epsilon}-\hat{\epsilon}'}{2}^2}{2}\right)+\\
	&+\tau^{|R>0|+|R'>0|}\left(1-\frac{\norm{\hat{\nu}-\hat{\nu}'}{2}^2}{2}\right)\\
\end{aligned}$$
\end{proposition}
\begin{proof}
$$\begin{aligned}
k_{\phi_{\mathcal{P}}}(P,P')&=\Braket{\phi_{\mathcal{P}}(P),\phi_{\mathcal{P}}(P')}\\
	&=\sum_{\alpha\beta\in \Sigma_\varepsilon^2}\frac{\epsilon_{\color{green}\alpha\beta}}{\|\epsilon\|_2}\frac{{\epsilon'}_{\color{green}\alpha\beta}}{\|\epsilon'\|_2}ww'\tau^{|R>0|+|R'>0|}\quad+\quad \sum_{\alpha\in \Sigma_\varepsilon}\frac{\nu_{\color{green}\alpha}}{\|\nu\|_2}\frac{{\nu'}_{\color{green}\alpha}}{\|\nu'\|_2}\tau^{|R>0|+|R'>0|}\\
	&=ww'\tau^{|Rb>0|+|R'>0|}\sum_{\alpha\beta\in \Sigma_\varepsilon^2}\frac{\epsilon_{\color{green}\alpha\beta}}{\|\epsilon\|_2}\frac{{\epsilon'}_{\color{green}\alpha\beta}}{\|\epsilon'\|_2}\quad+\quad \tau^{|R>0|+|R'>0|}\sum_{\alpha\in \Sigma_\varepsilon}\frac{\nu_{\color{green}\alpha}}{\|\nu\|_2}\frac{{\nu'}_{\color{green}\alpha}}{\|\nu'\|_2}\\
	&=ww'\tau^{|R>0|+|R'>0|}\Braket{\hat{\epsilon}, \hat{\epsilon}'}+ \tau^{|R>0|+|R'>0|}\Braket{\hat{\nu}, \hat{\nu}'}\\
	&=ww'\tau^{|R>0|+|R'>0|}\left(1-\frac{\norm{\hat{\epsilon}- \hat{\epsilon}'}{2}^2}{2}\right)+ \tau^{|R>0|+|R'>0|}\left(1-\frac{\norm{\hat{\nu}- \hat{\nu}'}{2}^2}{2}\right)\\
\end{aligned}$$
\end{proof}

\subsection{Properties}
As a first property, we want to show that when the two PPNs are (trace) equivalent, then there exists an embedding configuration for which the kernel computation reduces to the two PPNs' weight product, i.e. $ww'$. The kernel will then represent the probability that both Petri Nets are valid contemporaneously and, when both weights are $1$, the kernel returns $1$. We will call this condition as ``weak equality'' because we cannot possibly prove that $k_{\phi_{\mathcal{P}}}(P,P')=ww'\Leftrightarrow \mathcal{W}_p^n(P)=\mathcal{W}_p^n(P')$, as there could be similar embeddings coming from Petri Nets sharing a different weighted traces set ($\mathcal{W}_p^n(P)\neq\mathcal{W}_p^n(P')$).

\begin{lemma}[Weak Equality]
Given two PPNs $P=(s,t,L,R,w)$ and $P'=(s',t',L',R',w')$ providing the same set of weighted traces, then $k_{\phi_{\mathcal{P}}}(P,P')=ww'$ for $\tau=1$.
\end{lemma}
\begin{proof}
Given Proposition \ref{lem:rewritinglemma} and the positive definition of $\epsilon$ and $\nu$,  we have that $\norm{\hat{\epsilon}-\hat{\epsilon}'}{2}\to 0$ as well as $\norm{\hat{\nu}-\hat{\nu}'}{2}\to 0$, for which we can immediately close the goal.
\end{proof}

As per previous observations, we know that two PPNs should have the maximum dissimilarity when all the non $\varepsilon$-nodes have different labels, thus making it impossible to find an alignment, thus implying that they share a completely dissimilar set of weighted traces.

\begin{lemma}[Strong Dissimilarity]
Given two PPNs $P=(s,t,L,R,w)$ and $P'=(s',t',L',R',w')$, $k_{\phi_{\mathcal{P}}}(P,P')=0$ iff. $P$ and $P'$ have a different set of labels with $\tau,w,w'>0$.
\end{lemma}
\begin{proof}
If we exclude the trivial conditions $\tau=0$, $w=0$ or $w'=0$, the only condition when the kernel returns zero is when  $\Braket{\hat{\epsilon},\hat{\epsilon}'}=0$ and $\Braket{\hat{\nu},\hat{\nu}'}=0$. This implies that, when a component of $\epsilon$ (or $\nu$) is non-zero, the same component of $\epsilon'$ (or $\nu'$) is zero and viceversa. This directly requires that there is a different set of labels associated to the nodes. 
\end{proof}

As a corollary of the two lemmas, we have that the proposed embedding performs weakly-ideally, as the equality condition holds only on a relaxed form.

Last, under the assumption that a PPN is approximately characterized by $\epsilon$ and $\nu$, we might expect that the PPN similarity is characterized by the sum of the distance of both embeddings. Therefore, we show that an increase of both distance embeddings approximately corresponds to a decrease in the kernel output and viceversa.

\begin{lemma}\label{lem:approxRank}
Given two PPNs $P=(s,t,L,R,w)$ and $P'=(s',t',L',R',w')$ having respectively the embeddings $(\epsilon,\nu)$ and $(\epsilon',\nu')$, we have that the kernel $k_{\phi_{\mathcal{T}}}$ induces an inverse ranking of $\norm{\epsilon-\epsilon'}{2}+\norm{\nu-\nu'}{2}$:
$$k_{\phi_{\mathcal{T}}}(P,P')\appropto 2-(\norm{\epsilon-\epsilon'}{2}+\norm{\nu-\nu'}{2})$$
\end{lemma}
\begin{proof}
Let us use $T=\tau^{|R>0|+|R'>0|}$, $\omega=ww'$, $V=\norm{\hat{\nu}-\hat{\nu}'}{2}$, and $E=\norm{\hat{\epsilon}-\hat{\epsilon}'}{2}$ as shorthands. The goal can be rewritten as $k_{\phi_{\mathcal{T}}}(P,P')\appropto 2-(E+V)$. Given that the embeddings $(\epsilon,\nu)$ and $(\epsilon',\nu')$ are normalized kernel function $k_{\phi_{\mathcal{P}}}$ and that they are always positive definite, then we have that $0\leq E +V\leq 2$, so $0\leq 2-(E+V)\leq 2$. Using Proposition \ref{lem:rewritinglemma}, we can write $k_{\phi_{\mathcal{P}}}(P,P')$ as follows:
$$\left(1-\frac{E}{2}\right)\omega T+\left(1-\frac{V}{2}\right)T=T\left(\omega+1-\frac{E\omega+V}{2}\right)$$
Given that the embeddings $(\epsilon,\nu)$ and $(\epsilon',\nu')$ are normalized in $k$ and that they are always positive definite,  we also have that $0\leq E\omega +V\leq 2$ where $0\leq \omega\leq 1$. We can also write  $0\leq \omega+1-\frac{E\omega+V}{2}\leq \frac{2}{T}$. For $\omega,T=1$, we have that $k_{\phi_{\mathcal{P}}}(P,P')=2-\frac{(E+V)}{2}$. Thus, $0<\omega,T<1$ approximates the expected ranking. 
\end{proof}

Given that we can now follow Definition \ref{def:ppne} for representing a trace $\tau$ as a proper embedding after transforming it as a PPN $T$ (\S\ref{subsec:katk}), we can find the PPN $P$ providing the best approximate match with  a trace $\tau$ as follows:
\[\underset{{P}}{\max\arg}\;k_{\phi_{\mathcal{P}}}(P,T)\]
Still, this PPN matching strategy does not allow to find the trace maximizing such score. In order do assess such problem, the next section is going to determine both an exact (\S\ref{subsec:exbkptap}) and an approximated strategy (\S\ref{subsec:akptap}) for probabilistically matching one single trace from the PPN.
