\smallskip
\noindent
\textbf{Biological Sequence Analysis.}
Biological sequence analysis performs sequence alignments for sequence classification and recognition tasks. When sequence similarity is weak, it is hard to identify the most significant alignment to prefer: sequence similarity is then combined with probability for better ranking the preferred alignments \cite{durbin1998biological}. This is a shared point with our proposed approach. 
%
Biological sequences of interest, such as CpG islands or full DNA sequences, can be summarized into a probabilistic model, either via node-labelled Discrete-Time Markov Chains (PNFA) \cite{RyabkoU08} or Hidden Markov Models \cite{Helske2018}. As it is always possible to represent DTMC as HMM \cite{DUPONT20051349} and given that HMM can be represented as TGs, we only consider node-labelled DTMC. %The probability of reaching a state $s_t$ with a given label $\lambda(s)$ after visiting $t$ other states only depends on the previously visited state $s_{t+1}$, thus making such models memoryless. Therefore, e
Every biological sequence's probability is the joint probability of the transition probabilities required to visit the multigraph associated to the DTMC for generating the sequence, thus % The probability associated with each biological sequence reflects 
reflecting the likelihood of having the final trace within the set of traces of interest. 
%
Well-known algorithms such as Viterbi are used for sequence analysis, which might also return top-$k$ alignments \cite{577040} approximating probability and alignment similarity.
%Such techniques often exploit well-known algorithms for returning the probabilistic model's best alignment towards the sequence of interest. E.g., the Viterbi Algorithm  returns 
%the probabilistic model's path optimizing both trace probability and sequence similarity.  Nevertheless, the user could prefer a sequence with more significant similarity albeit underrepresented within the probabilistic model, while, in other cases, the user could favor a model trace with a higher probability at the expense of a lower homology. Thus, such an algorithm becomes unideal. In such situations, we problem still reduces to first enumerating all the more probable model traces for then skimming out the less likely to be candidate alignments for the given trace. 
This strategy is shared with our proposed approach. 
