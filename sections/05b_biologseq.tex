

\subsection{Biological Sequence Analysis}
Similarly to conformance checking, biological sequence analysis assumes that sequence homology can be reduced to sequence alignments where sequences are represented as strings \cite{durbin1998biological}. When sequence similarity is weak, it is hard to identify the most significant alignment to prefer. In such situations, the sequence similarity is often combined with the probability for better ranking the preferred alignments \cite{durbin1998biological}. This is a shared point with our proposed approach. 
%
Biological sequences of interest, such as CpG islands or full DNA sequences, can be summarized into a probabilistic model, either via node-labelled Discrete-Time Markov Chains (PNFA) \cite{RyabkoU08} or Hidden Markov Models \cite{Helske2018}. Given that it is always possible to represent the latter as the former \cite{DUPONT20051349} and given that the former can be easily represented as TGs, we will only consider node-labelled DTMC. %The probability of reaching a state $s_t$ with a given label $\lambda(s)$ after visiting $t$ other states only depends on the previously visited state $s_{t+1}$, thus making such models memoryless. Therefore, e
Every biological sequence's probability is the joint probability of the transition probabilities required to visit the multigraph associated to the DTMC for generating the sequence, thus % The probability associated with each biological sequence reflects 
reflecting the likelihood of having the final trace within the set of traces of interest. 
%
Well-known algorithms such as Viterbi are used for sequence analysis: Viterbi returns
%Such techniques often exploit well-known algorithms for returning the probabilistic model's best alignment towards the sequence of interest. E.g., the Viterbi Algorithm  returns 
the probabilistic model's path optimizing both trace probability and sequence similarity.  Nevertheless, the user could prefer a sequence with more significant similarity albeit underrepresented within the probabilistic model, while, in other cases, the user could favor a model trace with a higher probability at the expense of a lower homology. Thus, such an algorithm becomes unideal. In such situations, we problem still reduces to first enumerating all the more probable model traces for then skimming out the less likely to be candidate alignments for the given trace. This strategy is indeed the one pursued by our proposed approach. 
