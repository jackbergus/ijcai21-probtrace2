% IJCAI-21 Author's response

% Template file with author's reponse

\documentclass{article}
\pdfpagewidth=8.5in
\pdfpageheight=11in
\usepackage{ijcai21-authors-response}


\usepackage{times}
\usepackage{soul}
\usepackage{url}
\usepackage[hidelinks]{hyperref}
\usepackage[utf8]{inputenc}
\usepackage[small]{caption}
\usepackage{graphicx}
\usepackage{amsmath}
\usepackage{amsthm}
\usepackage{booktabs}
\usepackage{algorithm}
\usepackage{algorithmic}
\usepackage{lipsum}
\urlstyle{same}

\newtheorem{example}{Example}
\newtheorem{theorem}{Theorem}


\begin{document}

We thank the reviewers for their insightful comments, as well as for the continued recognition of the clarity and writing quality of our manuscript, and for the graciousness of noting this despite disagreement on some points.



We appreciate the reviewers' concern that the manuscript does not clearly state on how we improve the state of the art on \textit{probabilistic conformance checking}. To the best of our knowledge, this paper proposes for the first time an alignment strategy for conformance checking. In fact, recently developed probabilistic conformance checking approaches
provide a numerical quantification of the degree of conformance of an event log
with a stochastic process model by either assessing the distribution discrepancies
[\ref{a}], or by exploiting entropy-based measures [\ref{b},\ref{c}]. As these strategies are not
 based on trace alignments, these cannot be directly used to repair a given trace
with one of the traces generated by a stochastic process model.
 This approach is not
comparable with the existing literature as
its output is not numeric but consists of a ranked list of alignments.

Still, our optimal-ranking strategy exploits  off-the-shelf crisp trace aligners: they provide the minimum repair distance for making a trace conformant to the TG graph. Such embedding strategy requires to align a log trace with a TG trace, and combines such value with the TG trace's probability. Albeit we can exploit crisp trace aligners, such scenario would require us to re-compute the alignments when the query trace of reference changes. In order to overcome to such limitations providing an additional computational complexity overhead, we also propose a approximate-ranking strategy.


With respect to the latter embedding strategy, we are interested in neither embedding one single node of the graph, nor in providing a vector representation of the whole graph. While these are common strategies for graph embeddings, we are interested in providing a vector representation over traces, which might be composed by different valid sequences (paths). Furthermore, we are interesting in the intersection of embedding strategies for whole node-labelled graphs with string embedding traces, as we are interested to compare vectors from traces either generated from a TG or from a string. To the best of our knowledge, the only class of graph and trace embedding allowing such analogy are the ones by Gärtner et al., which we also referenced in our paper. Such techniques for graph embeddings are tractable, thus not requiring NP-Complete techniques for representing graphs as vectors; inevitably, the vector representation will lose accuracy. Still, we extended an existing approach from literature and adapted it in our use case scenario by projecting the graph over the unfolded traces of interest, and showed that our proposed approach is weakly-ideal: such is an interesting property from the literature of reference, showing that our embedding guarantees weak equality and strong dissimilarity. 


We appreciate the reviewers' concern on having just one dataset of reference. However, we must observe that the same dataset mined with different configurations (different models, such as BPMN with different firing weight estimators and Probabilistic Petri Nets) generate very different probabilistic decision graphs, both in the topology structure and in the edges' probability distribution. We pursued this approach to provide experiments on different settings, and we compared the ranking induced by both Optimal and Approximated ranking techniques using the Spearman Correlation Index. From our preliminary experiments, we showed how using KD-Trees over approximate ranking outperforms in efficiency the Optimal ranking, while providing a good Spearman Correlation with the Optimal Ranking strategy. However, we plan to extend this preliminary feasability study with more in-depth experiments showing the validity of our contribution.

We also appreciate the reviewers' concern that the manuscript does not include a significant empirical component. All things being equal, we agree that scholarly studies are strengthened by the inclusion of supporting empirical evidence. However, we part with the reviewers on the belief that our study is incomplete without such an empirical element. Still, the focus of this paper was more on the formal characterization of our proposed approach: due to the lack of space, we could not include such experiments that were coming from our previous work. That said, we look forward to the kind of empirical experiments would corroborate the theoretical properties that we studied in this present work.

\textbf{References}
\begin{enumerate}
	\item\label{a} Leemans, S.J.J., et al.: \textit{Earth movers’ stochastic
	conformance checking}. In: BPM. vol. 360, pp. 127–143. Springer (2019)
\item\label{b} Polyvyanyy, A., Kalenkova, A.A.: \textit{Monotone conformance checking for partially
matching designed and observed processes}. In: ICPM. pp. 81–88 (2019)
\item\label{c} Polyvyanyy, A. et al.: Monotone precision and recall measures for comparing executions and specifications of dynamic
systems. ACM Trans. Softw. Eng. Methodol.  (2020)
\end{enumerate}

\end{document}

