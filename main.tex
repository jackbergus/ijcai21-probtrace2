\documentclass[runningheads]{llncs}

\setlength{\paperheight}{232.8mm}
\setlength{\paperwidth}{151.5mm}
\setlength\voffset     {-23mm}
\setlength\hoffset     {-34mm}

%
\usepackage{hyperref}
\usepackage{multicol}
\usepackage[final]{changes}
\usepackage{url}
\usepackage{amsmath}
\usepackage{bm}
\usepackage{multirow}
\usepackage{tcolorbox}
\usepackage{rotating}
\usepackage{latexsym,amssymb,amsmath}
\usepackage{makecell}
\usepackage{xspace}
\usepackage{paralist}
\usepackage{wrapfig}
\usepackage{adjustbox}
\usepackage{cite}
%\usepackage{times}
%%%%%%%%%%%%%%%%%%%%%%%%%%%%%%%%%%%%%%%%%%%%%%%%%%%%%%%%%%%%%%%%%%%%%%%%%%%%%%
%%% Time-stamp: "2018-09-07 18:35:03 calvanese"
%%%%%%%%%%%%%%%%%%%%%%%%%%%%%%%%%%%%%%%%%%%%%%%%%%%%%%%%%%%%%%%%%%%%%%%%%%%%%%

%%%%%%%%%%%%%%%%%%%%%%%%%% General Math

\newcommand{\A}{\ensuremath{\mathcal{A}}}
\newcommand{\B}{\ensuremath{\mathcal{B}}}
%\newcommand{\C}{\ensuremath{\mathcal{C}}}
\newcommand{\D}{\ensuremath{\mathcal{D}}}
\newcommand{\E}{\ensuremath{\mathcal{E}}}
\newcommand{\F}{\ensuremath{\mathcal{F}}}
%\newcommand{\G}{\ensuremath{\mathcal{G}}}
\renewcommand{\H}{\ensuremath{\mathcal{H}}}
\newcommand{\I}{\ensuremath{\mathcal{I}}}
\newcommand{\J}{\ensuremath{\mathcal{J}}}
\newcommand{\K}{\ensuremath{\mathcal{K}}}
\renewcommand{\L}{\ensuremath{\mathcal{L}}}
\newcommand{\M}{\ensuremath{\mathcal{M}}}
\newcommand{\N}{\ensuremath{\mathcal{N}}}
\renewcommand{\O}{\ensuremath{\mathcal{O}}}
\renewcommand{\P}{\ensuremath{\mathcal{P}}}
\newcommand{\Q}{\ensuremath{\mathcal{Q}}}
\newcommand{\R}{\ensuremath{\mathcal{R}}}
%\renewcommand{\S}{\ensuremath{\mathcal{S}}}
\newcommand{\T}{\ensuremath{\mathcal{T}}}
%\newcommand{\U}{\ensuremath{\mathcal{U}}}
\newcommand{\V}{\ensuremath{\mathcal{V}}}
\newcommand{\W}{\ensuremath{\mathcal{W}}}
\newcommand{\X}{\ensuremath{\mathcal{X}}}
\newcommand{\Y}{\ensuremath{\mathcal{Y}}}
\newcommand{\Z}{\ensuremath{\mathcal{Z}}}

%%%%%%%%%%%%%%%%%%%%%%%%%% Abbreviations

%%\newcommand{\eset}{\emptyset}
%%\newcommand{\col}{\colon}
\newcommand{\ol}[1]{\overline{#1}}                % overline
%\newcommand{\ul}[1]{\underline{#1}}               % underline
%%\newcommand{\uls}[1]{\underline{\raisebox{0pt}[0pt][0.45ex]{}#1}}
%% ul with space between text and line

\newcommand{\ra}{\rightarrow}
\newcommand{\Ra}{\Rightarrow}
\newcommand{\la}{\leftarrow}
\newcommand{\La}{\Leftarrow}
%\newcommand{\lra}{\leftrightarrow}
\newcommand{\Lra}{\Leftrightarrow}
\newcommand{\lora}{\longrightarrow}
\newcommand{\Lora}{\Longrightarrow}
\newcommand{\lola}{\longleftarrow}
\newcommand{\Lola}{\Longleftarrow}
\newcommand{\lolra}{\longleftrightarrow}
\newcommand{\Lolra}{\Longleftrightarrow}
%\newcommand{\ua}{\uparrow}
\newcommand{\Ua}{\Uparrow}
\newcommand{\da}{\downarrow}
\newcommand{\Da}{\Downarrow}
\newcommand{\uda}{\updownarrow}
\newcommand{\Uda}{\Updownarrow}

%%%%%%%%%%%%%%%%%%%%%%%%%% Relations

%%\newcommand{\incl}{\subseteq}
%%\newcommand{\imp}{\rightarrow}
\newcommand{\per}{\mbox{\bf .}}                  % period

%%%%%%%%%%%%%%%%%%%%%%%%%% Delimiters

%%\newcommand{\quotes}[1]{{\lq\lq #1\rq\rq}}
%\newcommand{\set}[1]{\{#1\}}                      % set
%\newcommand{\Set}[1]{\left\{#1\right\}}
\newcommand{\bigset}[1]{\Bigl\{#1\Bigr\}}
\newcommand{\bigmid}{\Big|}
\newcommand{\size}[1]{|{#1}|}                     % cardinality of a set
%%\newcommand{\Card}[1]{\left| #1\right|}
\newcommand{\card}[1]{\sharp #1}
\newcommand{\tup}[1]{\langle #1\rangle}            % tuple
\newcommand{\Tup}[1]{\Braket{#1}}
\newcommand{\norm}[2]{\|#1\|_{#2}}
\newcommand{\setone}[2][1]{\set{#1\cld #2}}

%%%%%%%%%%%%%%%%%%%%%%%%%% STYLING AND SPACING

%\newcommand{\inlinetitle}[1]{\smallskip\noindent\textbf{#1.}\xspace}





\newcolumntype{C}{>{\centering\arraybackslash}X}

%\makeatletter
%\g@addto@macro\normalsize{%
%\setlength{\abovecaptionskip}{-2pt}
%\setlength{\belowcaptionskip}{12pt}
%\setlength\abovedisplayskip{3pt}
%\setlength\belowdisplayskip{3pt}
%\setlength\abovedisplayshortskip{3pt}
%\setlength\belowdisplayshortskip{3pt}
%}
%\makeatother

\newcounter{dummy} 
\newcounter{dummy1} 
\newcounter{dummy2}
\newcounter{dummy3} 
\newcounter{dummy4}
\newcounter{dummy5} 
\newcounter{dummy6}
\newcounter{dummy7}
%\numberwithin{dummy}{section}

\usepackage[thmmarks,amsmath]{ntheorem}
%\theorempreskip{1pt}
%\theorempostskip{1pt}

%\theoremstyle{plain}
%\theorembodyfont{\normalfont}
%\theoremseparator{.}
%\let\definition\relax
%\theoremsymbol{\ensuremath{\square}}
%\newtheorem{definition}{Definition}


\let\proposition\relax
\let\theorem\relax
\let\lemma\relax
\let\definition\relax
\theoremseparator{.}
\theorembodyfont{\itshape}
\theoremsymbol{$\triangleleft$}
\newtheorem{theorem}[dummy]{Theorem}
\newtheorem{lemma}[dummy1]{Lemma}
\newtheorem{definition}[dummy2]{Definition}
\newtheorem{proposition}[dummy3]{Proposition}

\let\remark\relax
\let\example\relax
%\let\example*\relax
\theorembodyfont{\normalfont}
\newtheorem{example}[dummy4]{Example}
\newtheorem{remark}[dummy5]{Remark}
%\newtheorem{example*}[dummy4]{Example}


\theoremstyle{nonumberplain}
\theoremheaderfont{\itshape}
\theorembodyfont{\normalfont}
\let\proof\relax
\theoremseparator{.}
\theoremsymbol{\ensuremath{\dashv}}
\newtheorem{proof}[dummy6]{Proof}


\qedsymbol{\ensuremath{\dashv}}


%%% Local Variables:
%%% mode: latex
%%% TeX-master: "main"
%%% save-place: t
%%% End:

\usepackage{todonotes}

\usepackage{graphicx}
\usepackage{xcolor,color}
\usepackage{subfig}
\usepackage{tikz}
\usepackage{calc}

\usepackage{tabularx}
\usepackage{booktabs}

\usepackage{ulem}

\usepackage{kbordermatrix}
\usepackage{amsmath,amsfonts}
\usepackage{braket}
\usepackage{xfrac}

\usepackage{pifont}
\usepackage{amssymb}
\usepackage{paralist}
\usepackage[inline]{enumitem}

\newcommand{\pmin}{\rho}

\newcommand{\const}[1]{\mathsf{#1}}

\newcommand{\alphabet}{\Sigma}
\newcommand{\tasks}{\mathcal{A}}
\newcommand{\hidden}{\tau}

\newcommand{\uswn}{SWN\xspace}
\newcommand{\net}{\ensuremath{N}}
\newcommand{\tg}{\ensuremath{G}}
\newcommand{\closed}[1]{\overline{#1}}
\newcommand{\marking}{m}
\newcommand{\enaset}[2]{E_{#2}(#1)}
\newcommand{\fire}[4]{#1\xrightarrow{#2}_{#4}#3}
\newcommand{\probt}[3]{\mathbb{P}_{#2,#3}(#1)}
\newcommand{\prob}[2]{\mathbb{P}_{#2}(#1)}
\newcommand{\rg}[1]{RG(#1)}
\newcommand{\ind}[1]{\textnormal{\texttt{#1}}}
\newcommand{\seq}{\eta}
\newcommand{\run}{\xi}
\newcommand{\trace}{\sigma}
\newcommand{\traces}[1]{\mathit{traces}(#1)}
\newcommand{\ptraces}[2]{\mathit{ptraces}_{#2}(#1)}

\newcommand{\nreach}[3][]{#2 \overset{#1}{\rightsquigarrow} #3}
\newcommand{\runs}[2]{runs_{#2}(#1)}
\newcommand{\seqs}[2]{seqs_{#2}(#1)}
\newcommand{\transp}[1]{#1^\top}
\newcommand{\embed}{\phi}
\newcommand{\trembed}{{\embed^{\text{tr}}}}
\newcommand{\gorgembed}{{\embed^{g}}}


\newcommand{\pa}{\rho_{23}}
\newcommand{\pb}{\rho_{24}}
\newcommand{\pc}{\rho_{55}}
\newcommand{\pd}{\rho_{65}}
\newcommand{\pe}{\rho_{67}}
\newcommand{\pf}{\rho_{57}}

\newcommand{\logtrace}{\trace}
\newcommand{\nonlogtrace}{{\trace'}}


\newcommand{\approptoinn}[2]{\mathrel{\vcenter{
			\offinterlineskip\halign{\hfil$##$\cr
				#1\propto\cr\noalign{\kern2pt}#1\sim\cr\noalign{\kern-2pt}}}}}
\newcommand{\appropto}{\mathpalette\approptoinn\relax}


\newcommand{\unravelled}{unfolded}
\newcommand{\unravelling}{unfolding}
\newcommand{\unravel}{unfold}
\newcommand{\Ind}[1]{\ind{i}_{#1}}


%
%\def\WWITHN{def}
%\ifdefined\WWITHN
%\newcommand{\WCal}[2]{{\mathcal{W}_{#1}^{#2}}}
%\newcommand{\TBf}[2]{{\mathbf{T}_{#1}^{#2}}}
%\else
\newcommand{\TBf}[2]{{\mathbf{G}_{#1}}}
%\fi
\newcommand{\expN}{\closed{\tg_{\rg{\net}}}}

\def\EqualityHolds{itholds}
\ifdefined\EqualityHolds
\newcommand{\probarg}{\tg}
\newcommand{\WCal}[2]{\ptraces{\tg}{#1}}
\else
\newcommand{\probarg}{\expN}
\newcommand{\WCal}[2]{\ptraces{\closed{\tg_{\rg{\net}}}}{#1}}
\fi
\newcommand{\probskip}[1]{\prob{#1}{\probarg}}
\newcommand{\goldenrank}{\mathcal{R}} 



\makeatletter
\g@addto@macro\normalsize{%
\setlength{\abovecaptionskip}{3pt}
\setlength{\belowcaptionskip}{-10pt}
\setlength\abovedisplayskip{3pt}
\setlength\belowdisplayskip{3pt}
\setlength\abovedisplayshortskip{3pt}
\setlength\belowdisplayshortskip{3pt}
}
\makeatother

%\theorempreskip{2pt}
%\theorempostskip{2pt}


\begin{document}

%
\title{Probabilistic Trace Alignment}
%\subtitle{\emph{Vision Paper}}
%
%\titlerunning{Abbreviated paper title}
% If the paper title is too long for the running head, you can set
% an abbreviated paper title here
%
\author{
	Giacomo Bergami\inst{1} \and
	Fabrizio Maria Maggi\inst{1} \and
	Marco Montali\inst{1} \and \\
	Rafael Pe\~naloza\inst{2}}

\authorrunning{G.~Bergami, F.M.~Maggi, M.~Montali and R.~Pe\~naloza}
% First names are abbreviated in the running head.
% If there are more than two authors, 'et al.' is used.
%
\institute{
	Free University of Bozen-Bolzano, Italy \\\email{gibergami@unibz.it,\{maggi,montali\}@inf.unibz.it}
	\and
	University of Milano-Bicocca \\\email{rafael.penaloza@unimib.it}
}

%\vspace{1.5cm}
%
%
\maketitle              % typeset the header of the contribution
\linespread{0.95}
%\vspace{-0.5cm}

\begin{abstract}
%Conformance checking techniques verify whether the observed behavior recorded in an event log matches a modeled behavior. This type of analysis is crucial, because often real process executions deviate from the reference process models. Currently, one of the most common approaches to conformance checking is based on trace alignments. 
Alignments provide sophisticated diagnostics that pinpoint deviations in a trace with respect to a process model and their severity. However, approaches based on trace alignments use crisp process models as reference and recent probabilistic conformance checking approaches check the degree of conformance of an event log with respect to a stochastic process model instead of finding trace alignments. In this paper, for the first time, we provide a conformance checking approach based on trace alignments using stochastic Workflow nets. Conceptually, this requires to handle the two possibly contrasting forces of the cost of the alignment on the one hand and the likelihood of the model trace with respect to which the alignment is computed on the other.
%If the cost of the alignment is too high, even if the considered model trace is very likely, applying too many changes in the original trace is not always the best option. The proposed approach has been evaluated using a real-life event log pertaining to the treatment of the sepsis disease in a hospital.
\end{abstract}

\keywords{Stochastic Workflow nets, Conformance Checking, Alignments}

%

\section{Introduction}\label{introduction}


In the existing literature on conformance checking, one of the most common approach is based on trace alignments. However, the proposed approaches based on trace alignments use crisp process models as reference models. On the other hand, recently, probabilistic conformance checking approaches are gaining momentum, but the existing approaches are used to check the degree of conformance of and event log with respect to a stochastic process model instead of finding trace alignments.
In this paper, for the first time, we provide a conformance checking approach based on trace alignments using stochastic reference models. Conceptually, this requires to handle the two possibly contrasting forces of the cost of the alignment on the one hand and the likelihood of the model trace with respect to which the alignment is computed.
Balancing between the likelihood of the model trace with respect to which we are computing the alignment and the cost of the alignment (if the cost of the alignment is too high even if the model trace is very likely applying too many changes in the original trace is in turn not very likely).




%%%%% Proposed part as the last part of the introduction: 
\texttt{\color{red}[TODO]}
We perform experiments to empirically evaluate some properties of the proposed embedding strategy:
\begin{enumerate}
	\item {We're going to informally assess the degree of the trace alignment approximation induced by the vector kernel $k_{\phi_\mathcal{P}}$  if compared to the exact probabilistic trace alignment while introducing such embedding (\S\ref{subsec:eta}).}
	\item We're going to assess the proposed kernel's appropriateness (\S\ref{subsec:katk}) by taking all the traces $\tau$ generated by a USWN $\mathcal{U}$, add some noise to $\tau$ thus causing noised traces $\tilde{\tau}$, and evaluating the distance between the trace ranking of $\tau$ and $\tilde{\tau}$ over the traces of $P$ (\S\ref{subsec:apprp}). 
	\item Last, we're going to compare the time required to compute the exact trace alignment approach against the embedding-based approach (\S\ref{subsec:efficio}).
\end{enumerate}
\texttt{\color{red}[TODO]}

Implementation\footnote{\url{https://github.com/jackbergus/approxProbTraceAlign}}

\section{Modeling Probabilistic Dynamic Systems}
\label{sec:models}
In this section, we introduce the different models and techniques that will constitute the basis for representing and computing probabilistic trace alignments.

\subsection{Stochastic Workflow Nets}\label{subsec:spn}
As customary in probabilistic conformance checking \cite{DBLP:conf/bpm/LeemansSA19,DBLP:conf/icpm/PolyvyanyyK19,DBLP:journals/tosem/PolyvyanyySWCM20}, we adopt stochastic Petri nets \cite{MarsanCB84,RoggeSoltiAW13} as the underlying formal basis to represent processes. More specifically, we consider an interesting class of stochastic Petri nets with only immediate transitions (i.e., no timed ones), namely untimed Stochastic Workflow Nets (\uswn for short).
We assume to have a set $\alphabet = \tasks \cup \set{\tau}$ of labels, where labels in $\tasks$ indicate process tasks, whereas $\tau$ indicates an invisible execution step ($\tau$-transition). A \emph{trace} is a finite sequence of labels from $\tasks$.

\begin{definition} An \emph{untimed Stochastic Workflow Net (\uswn)}
is a tuple $\net = (P,T,F,\ell,W)$ where:
\begin{compactitem}
\item $(P,T,F)$ is a standard \emph{Workflow net} with places $P$, transitions $T$, and flow relation $F$ such that there is exactly one \emph{input place} with no incoming arc, and exactly one \emph{output place} with no outgoing arcs;
\item $\ell: T \rightarrow \alphabet$ is a \emph{labeling function} mapping each transition $t \in T$ into a label $\ell(t) \in \alphabet$ - this either indicates the task executed upon firing $t$, or the fact that $t$ is an invisible transition (in the latter case, $\ell(t) = \tau$);
\item $W\colon T\to \mathbb{R}^+$ is a \emph{weight function} assigning a positive firing weight to each transition of the net.
\end{compactitem}
\end{definition}
Given an \uswn $\net$, we use dot notation to extract its constitutive components (e.g., $\net.P$ denotes its places). \emph{The same dot notation will be used for the other structures introduced in the paper}. We also use $\net.in$ and $\net.out$ to respectively denote the input and output place of $\net$.

As usual, the current state of execution is captured using a marking of the net, that is, a multiset over places $P$ indicating how many tokens populate each place.
%As pointed out above, \emph{we always assume, as customary in BPM, that the input \uswn is \underline{bounded}}, that is, in every state the number of tokens associated to each place cannot exceed a maximum, fixed threshold.
The notions of transition enablement and firing are also the standard ones, since they do not depend on the weight function, which provides the basis for capturing the stochastic behavior of the net. We use the following notation: given a marking $\marking$ over \uswn $\net$, we denote by $\enaset{\marking}{\net}$ the set of enabled transitions in $\marking$; given transition $t \in \enaset{\marking}{\net}$, we write $\fire{\marking}{t}{\marking'}{\net}$ to capture the fact that, within $\net$, firing $t$ in $\marking$ results in the new marking $\marking'$. A \emph{firing sequence of $\net$ starting from marking $\marking_0$} is a sequence $t_1\cdots t_n$ of transitions from $\net.T$ so that, for every $i \in \set{1,\ldots,n}$, we have that $\fire{\marking_{i-1}}{t_i}{\marking_{i}}{\net}$. We say that the firing sequence results in $\marking_{n}$.

As customary in Workflow nets, we consider two special markings: the \emph{input} (resp.~\emph{output}) marking $m_{in}^\net$ (resp.~$m_{out}^\net$) that assigns a single token to the input (resp.~output) place $\net.in$ (resp.~$\net.out$) of $\net$, and no token elsewhere. A \emph{valid sequence} $\seq = t_1\cdots t_n$ of $\net$ is a firing sequence of $\net$ starting from $m_{in}^\net$ and resulting in $m_{out}^\net$. A sequence of labels $\run = \alpha_1 \cdots \alpha_n$ from $\alphabet$ is a \emph{run} of $\net$ if there exists a valid underlying sequence $\seq = t_1\cdots t_n$ of $\net$  such that, for every $i \in \set{1,\ldots,n}$, we have $\net.\ell(t_i) = \alpha_i$. Run $\run$ may have different underlying valid sequences in $\net$, which we collectively refer to as $\seqs{\run}{\net}$. A trace $\trace$ is a \emph{model trace} of $\net$ (or $\net$-trace for short) if there exists an underlying run $\run$ of $\net$ that corresponds to $\trace$ once all occurrences of $\tau$ are removed. There may be multiple runs underlying an $\net$-trace $\trace$, and we collectively refer to them as $\runs{\trace}{\net}$. Finally, we denote the (possibly infinite) set of $\net$-traces as $\traces{\net}$.
\begin{example} %\small
\label{ex:net}
\figurename~\ref{fig:spn} shows an example of an \uswn with input place $p_1$ and output place $p_7$. One run of the net is $\const{\tau c \tau a a \tau}$, which corresponds to trace $\const{caa}$. Overall, the net supports infinitely many finite traces of the form (represented using regular expressions):
\begin{inparaenum}[\it (i)]
\item $\const{aa^*}$,
\item $\const{cb}$,
\item $\const{caa^*}$.
\end{inparaenum}
\end{example}

When executing an \uswn, the crucial addition to the standard execution semantics of Workflow nets is that, being the net stochastic, in each marking the set of enabled transitions gets associated to a discrete probability distribution. This is defined as follows: given a marking $\marking$ of $N$ and an enabled transition $t \in \enaset{\marking}{\net}$, the \emph{firing probability} of $t$ in $\marking$ is $\probt{t}{\marking}{\net} = \frac{\net.W(t)}{\sum_{t'\in \enaset{\marking}{\net}}\net.W(t')}$. As required, the probabilities associated to all enabled transitions in a marking always add up to 1.

Using firing probabilities as a basic building block, we define the probability $\prob{\seq}{\net}$ of a valid sequence $\seq = t_1\cdots t_n$ of $\net$ as the product of the probabilities associated to each transition: $\prob{\seq}{\net} = \prod_{i \in \set{1,\ldots,n}}\prob{t_i}{\marking_{i-1}}{\net}$. %For a run $\run$ of $\net$, its probability $\prob{\seq}{\net}$ is then obtained by summing up the probabilities of all valid sequences corresponding to $\run$: $\prob{\run}{\net} = \sum_{\seq \in \seqs{\run}{\net}} \prob{\seq}{\net}$. Likewise, for a trace $\trace$ of $\net$, its probability is obtained by summing up
For a trace $\trace$ of $\net$, its probability $\prob{\trace}{\net}$ is then obtained by collecting all its underlying runs, in turn collecting all their underlying valid sequences, and summing up their respective probabilities: $\prob{\trace}{\net} = \sum_{\run \in \runs{\trace}{\net}} \sum_{\seq \in \seqs{\run}{\net}} \prob{\seq}{\net}$. This corresponds to the intuition that, to observe $\trace$, one can equivalently pick any of its underlying valid sequences. Notably, if a trace is not an $\net$-trace (i.e., it does not conform with $\net$), then its probability is 0. For convenience, when needed, we represent an $\net$-trace as a pair $\tup{\trace,\prob{\trace}{\net}}$, where the probability assigned to $\trace$ by $\net$ is retained.


By interpreting concurrency by interleaving, we can represent all transition firings of an \uswn, together with their probabilities, in a reachability graph.

\begin{definition}
The \emph{Reachability Graph} $\rg{\net}$ of \uswn \net is a triple $(M,E,P)$ where:
\begin{compactitem}[$\bullet$]
\item $M$ is the set of all reachable markings from $\marking_0^\net$ (including $\marking_0^\net$ itself).
\item $E \subseteq M \times \alphabet \times M$ is a $\alphabet$-\emph{labeled transition relation} induced by $\net$, that is, for $\marking,\marking' \in M$, we have edge $(\marking,a,\marking') \in E$ if and only if there exists transition $t$ in $\net$ with label $\ell(t) = a$ and such that $\fire{\marking}{t}{\marking'}{\net}$.
\item $P:E \rightarrow [0,1]$ is the \emph{transition probability} function assigning to each transition $(\marking,a,\marking') \in E$ its corresponding probability, obtained from the firing probability of the \uswn transition(s) that lead from $\marking$ to $\marking'$ and are labeled by $a$: $P(\marking,a,\marking') = \sum_{t_i \in \enaset{\marking}{\net} \text{ s.t.~} \net.\ell(t) = a \text{ and } \fire{\marking}{t}{\marking'}{\net}} \prob{t}{\marking}{\net}$.
\end{compactitem}
\end{definition}
Notice that, in the definition, we have to account for the possible case where, in a given state, distinct net transitions with the same label produce the same consequent state. In this case, they are indistinguishable when observing the execution traces of the net, and in fact they collapse into a single edge of the reachability graph. This is why, in this case, we accumulate all their firing probabilities into a single value.


In the remainder of the paper, given an \uswn $\net$, we always assume that it satisfies two structural assumptions that are natural in the BPM setting:
\begin{compactitem}
\item $\net$ is \emph{bounded}, that is, every marking in $\rg{\net}$ assigns at most a pre-defined number of tokens to each place;
\item $\rg{\net}$ does not contain loops where all edges are labeled with $\tau$.
\end{compactitem}
The first assumption indicates that a case of the process does not generate unboundedly many parallel threads, and guarantees in turn that the reachability graph contains finitely many states. The second assumption naturally corresponds to how $\tau$-transitions are used when modeling business processes, where they are essential in representing gateways (such as exclusive and parallel splits/joins), cascaded gateways without tasks in between, and skippable tasks.  In all these cases, multiple $\tau$-transitions may be used, but never creating completely invisible loops. Under this assumption, $\net$ enjoys a very interesting property: given a trace $\trace$, there are only boundedly many valid sequences that can produce it. Hence, the probability of $\trace$ can be computed by:
\begin{inparaenum}[\it (i)]
\item exhaustively enumerating all its valid sequences;
\item calculating the probability of each such sequence;
\item summing up all the so-obtained probabilities.
\end{inparaenum}
\figurename~\ref{fig:rg} shows an example of a reachability graph.
\begin{example} %\small
  \label{ex:trace}
Consider the \uswn \net of Example~\ref{ex:net}. Considering trace $\const{caa}$, it is easy to see that it has only one underlying run, namely $\const{\tau c \tau a a \tau}$, in turn produced by a single underlying valid sequence, and that
%The firing probability of picking the first $\tau$-transition starting from the input marking is $1$, as there are no alternatives. In the new marking, where only one token is assigned to $p_2$, the firing probability of choosing the $\tau$-transition above is $\rho_{23} = \frac{v_{\tau_2}}{v_{\tau_2}+v_c}$, whereas that of choosing the $c$-transition below is $\rho_{24} = \frac{v_c}{v_{\tau_2}+v_c}$. Upon choosing the transition below, the new marking assigns only to $p_4$ one token, leaving just one choice to continue by moving that token to $p_6$. In that marking, the probability of choosing the $a$-transition above is $\rho_{65} = \frac{v_{a_3}}{v_{a_3}+v_b}$, resulting in the token being moved to $p_5$. In this new marking, the probability of iterating over the $a$-transition above is $\rho_{55} = \frac{v_{\tau_3}}{v_{\tau_3}+v_{a_2}}$, while that of completing in the output marking via the enabled $\tau$-transition is $\rho_{57} = \frac{v_{\tau_3}}{v_{\tau_3}+v_{a_2}}$. Hence, all in all
$\prob{\const{caa}}{\net} = 1 \cdot \rho_{24} \cdot 1 \cdot \rho_{65} \cdot \rho_{55} \cdot \rho_{57}$.
\end{example}



%Technically:
%\begin{compactitem}
%\item $P$ is a finite set of \textit{places}.
%\item $T$ is a finite set of \textit{transitions}, each of which is associate to a label. Each label either denotes a task executed upon transition firing, or indicates an invisible transition; in the latter case, we employ the special label $\varepsilon$.\footnotesize{This corresponds to the standard notion of $\tau$-transitions in Petri nets, but we use $\varepsilon$ since in the remainder of the paper $\tau$ is used to refer to an execution trace.}
%%to which we associate a label $\lambda(t)\in\Sigma$, where $\Sigma$ also includes the empty string\footnote{Given that we are going to denote the traces as $\tau$ and $t$ as the Petri net Transitions, we choose to denote the empty string as such instead of $\tau$ as in current literature from Petri nets.} $\varepsilon$.
%\item $F\subseteq (P\times T)\cup (T\times P)$ is the flow relation, representing arcs linking places to transitions and transitions to places.
%%to which we associate a \textit{firing cost} $\omega\colon F\to\mathbb{N}$.
%\item The initial place $i\in P$ has no ingoing edges ($\not\exists t\in T. (t,i)\in F$).
%\item The final place $f\in P$ has no outgoing edges ($\not\exists t\in T. (f,t)\in F$).
%\item $W\colon T\to \mathbb{R}^+_{>0}$ defines a \textit{firing weight} associated to each transition.
%\end{compactitem}

%A \textit{marking} is an assignment of a given amount of indistinguishable tokens to places described by a vector $M\colon P\to \mathbb{N}$. We say that a given transition $t$ is \textit{enabled} if $M(p)\geq 1$ for each ingoing $p$ to $t$ ($(p,t)\in F$). If such transition is enabled, then it can \textit{fire} a token. The \textit{enabling transitions} $E(M)$ for a given marking $M$ are all the $t$ reachable from $p$ ($(p,t)\in F$) with $M(p)\neq 0$ where $t$ is enabled. When $t$ can fire a token for a marking $M$, we can generate a novel marking $M'$ from $M$ by moving the tokens from the ingoing places towards the outgoing places as follows:
%\[\forall p\in P.\; M'(p)=M(p)-\mathbf{1}_{(p,t)\in F}+\mathbf{1}_{(t,p)\in F}\]
%We denote the transition from marking $M$ to marking $M'$ via an enabling $t$ as a relation $M\overset{t}{\to}M'$. We say that an \uswn with initial marking $M$ is $k$-\textit{bounded} if each of the markings $M'$ reachable from $M$, $M$ included, have $\forall p\in P.\; M(p)\leq k$\\

\begin{figure*}[!t]
	\begin{minipage}{.49\textwidth}
		\includegraphics[width=.9\textwidth]{images/petri.pdf}
		\caption{A sample \uswn. Labels are shown in green, $\tau$ transitions in grey, weights in magenta.}\label{fig:spn}
	\end{minipage}\hfill \begin{minipage}{.49\textwidth}
		\includegraphics[width=.9\textwidth]{images/rg.pdf}
		\caption{Reachability graph $RG(N)$ of the \uswn $N$. Probabilities are shown in violet.}\label{fig:rg}
	\end{minipage}
\end{figure*} \begin{figure*}[!t]
	\begin{minipage}{.49\textwidth} \includegraphics[width=.9\textwidth]{images/running_example.pdf}
	\caption{Transition graph $G_{RG(N)}$ encoding the reachability graph $RG(N)$.}\label{fig:lmc}\label{fig:orig}
\end{minipage}\hfill \begin{minipage}{.49\textwidth} \includegraphics[width=.9\textwidth]{images/closed_example.pdf}
	\caption{Transition graph $\closed{G_{RG(N)}}$ resulting from the transition graph in $G_{RG(N)}$ after $\tau$-closure.}\label{fig:closed}
\end{minipage}
\end{figure*}


%Figures \ref{fig:spn} and \ref{fig:rg} respectively show a sample \uswn and its corresponding reachability graph. This net will be our running example throughout the paper.


%\begin{example}
%Figure \ref{fig:spn} provides a sample \uswn defined as such, and \ref{fig:rg} provides its associated Reachability Graph. This representation can be beneficial when such \uswns are inferred and extracted from log files \cite{PPNFromLog} for extracting the set of the probabilistic traces associated to the \uswn.
%\end{example}
%
%
%We use \uswns for modelling business processes: in fact, it can be shown \cite{RaedtsPUWGS07} that it is always possible to convert BPMNs to \uswns. Last, we also assume that a transition is enabled when all of its input places contain at least one token and that, when a transition fires, we remove one token from each of its input places and depose tokens for each of its output places.



\subsection{Transition Graphs}\label{subsec:ppn}

The graph and trace embedding techniques that we will use as the basis for computing probabilistic alignments cannot be directly defined over reachability graphs. In fact, these techniques rely on graphs where edges are only labeled by probabilities, whereas labels are attached to nodes. In addition, towards readily enabling efficient algorithmic techniques, such graphs are compactly defined using transition matrixes. We therefore take inspiration from \cite{GartnerFW03} and introduce the so-called \emph{probabilistic transition graphs}, which we will later use to encode \uswn{s} via their reachability graphs.

%For a matrix $Q$ with row set $A$ and column set $B$, notation $[Q]_{ab}$ for $a \in A$ and $b \in B$ denotes the corresponding element in the matrix. In addition $\transp{Q}$ denotes the transposed matrix where rows and columns are inverted. We employ the usual sum and product operations over matrixes and arrays, and denote, for a square matrix $Q$, the repeated multiplication of $Q$ with itself $n$ times by $Q^n$.\todo{Rimuovere questo paragrafo se serve spazio,}\todo{NOn si capisce il significato di $\omega$}
%In our technical treatment, we continue to assume the existence of a set $\alphabet$ of labels (including the special label $\tau$).
\begin{definition} A \emph{(Probabilistic) Transition Graph} is a tuple $(V,s,t,L,R)$ where:
  \begin{inparaenum}[\itshape (i)]
    \item $V \subset \mathbb{N}$ is a set of \emph{nodes};
    \item $s\in V$ is the \emph{initial node};
    \item $e\in V$ is the \emph{accepting node};
    \item $L: \alphabet \times V \rightarrow \{0,1\}$ is a \emph{label matrix} associating each node in $V$ to a single label in $\alphabet$, where for label $\alpha \in \alphabet$ and node $\ind{i} \in V$, $[L]_{\alpha\ind{i}}$ gives $1$ if $\ind{i}$ is labeled by $\alpha$, $0$ otherwise;
    \item $R: V \times V \rightarrow [0,1]$ is a \emph{(probabilistic) transition matrix} indicating, for each pair of nodes, what is the probability that executing a transition from the first node leads to the second node.
   % \item $\omega \in [0,1]$ is a \emph{graph weight} indicating an overall value associated to the entire graph.
  \end{inparaenum}
$L$ and $R$ satisfy the following well-formedness conditions:
\begin{inparaenum}[\itshape (i)]
\item for every $i \in V$ there is one and only one label $\alpha \in \alphabet$ so   that $[L]_{\alpha\ind{i}}=1$;
\item  for  every $\ind{i} \in V$, we have that $\sum_{\ind{j}\in V}[R]_{\ind{ij}}=1$.
\end{inparaenum}

\ADD{A Weigthed (Probabilistic) Transition Graph is a pair $(G,\omega)$ where $G$ is a (Probabilistic) Transition Graph and $\omega\in[0,1]$ is a weight associated to such graph.}
\end{definition}
The condition for $L$ indicates that each node is mapped by $L$ to a single label, while the same label may be used for multiple nodes. The condition for $R$ ensures that the values contained therein can be interpreted as a probability distribution when choosing which next node to pick upon executing a transition. Matrices $L$ and $R$ can be exploited to determine the probability of reaching a node labeled by $\beta\in\Sigma$ from any node labeled $\alpha\in\Sigma$ in $n$ steps with $[LR^n\transp{L}]_{\alpha\beta}/[L\transp{L}]_{\alpha\alpha}$, that we can shorthand as $[G.\Lambda^n]_{\alpha\beta}$ (see \cite{GartnerFW03} and Example \ref{ex:wheredotiszero}).

A transition graph $\tg$ can be visualized as shown in \figurename~\ref{fig:lmc} (and in \figurename~\ref{fig:closed} after $\tau$-closure). There, the various elements have the obvious interpretation, with the only important consideration that an edge from node $\ind{i}$ to node $\ind{j}$ is only shown if the transition probability $[\tg.R]_{\ind{i}\ind{j}}$ is positive.
%There, each node $\ind{i} \in \tg.V$ is  represented as a circle with its identifying number. The initial node is decorated by a small incoming edge, while the final node is double circled. The label of the node is shown close to the circle, in agreement with $\tg.L$. Finally, an edge from $\ind{i} \in \tg.V$ to $\ind{j} \in \tg.V$ is shown if the transition probability $[\tg.R]_{\ind{i}\ind{j}}$ is positive. Each edge is decorated with the positive probability assigned by $\tg.R$.

%\begin{definition}[Path, trace]
%A \emph{path} in a transition graph $\tg$ is a finite sequence of nodes $\ind{i}_1 \cdots \ind{i}_n$ (with $n > 1$) such that, for every $j \in \set{1,\ldots,n-1}$, we have that $[\tg.R]_{\ind{i}_j\ind{i}_{j+1}} > 0$. Such a path is \emph{valid} if it starts from the initial node and ends in the accepting node of $\tg$, that is, $\ind{i}_1 = \tg.s$ and $\ind{i}_n = \tg.e$.
%
%A \emph{trace} is a finite sequence of nodes that can be turned into a valid sequence by introducing in the sequence an arbitrary number of $\tau$ labels (so as to account for hidden transitions in the graph).
%\end{definition}
%From the definition, it is clear that every valid path can be straightforwardly converted into a corresponding trace by removing all $\tau$ labels from the sequence.

%$\npath{\ind{i}}{\ind{j}}$

By mirroring to definitions of \uswn{s} taking into account that now labels are on nodes, a \emph{valid sequence} of $\tg$ is a sequence $\ind{i}_0\ldots\ind{i}_n$ of nodes in $\tg.V$ that leads from the initial to the accepting node by only traversing transitions with nonzero probability:
\begin{inparaenum}[\it (i)]
\item $\ind{i}_0 = \tg.s$;
\item $\ind{i}_n = \tg.e$;
\item if the sequence contains at least two nodes, each two consecutive nodes are connected by a positive transition probability, i.e., for every $j \in \set{1,\ldots,n}$ we have $[R]_{\ind{i}_{j-1}\ind{i}_{j}} > 0$.
\end{inparaenum}
Runs and model traces of transition graphs are then defined as in \uswn{s}, and we employ the same notation to indicate the runs underlying a model trace, and the valid sequences underlying a run. The computation of probabilities for runs and traces is hence defined equivalently.

%We close this section by introducing how some  matrix operations defined in the literature \cite{GartnerFW03} are applied to matrixes $L$ and $R$ of $\tg$, towards tackling interesting probability computations. These will be instrumental later on in the paper. Given two nodes $\ind{i},\ind{j} \in \tg.V$, $[R^n]_{\ind{i}\ind{j}}$ returns the probability of having a path in $\tg$ that connects $\ind{i}$ to $\ind{j}$ and has length $n$. Given two labels $\alpha,\beta \in \alphabet$, with $[LR^n\transp{L}]_{\alpha\beta}/[L\transp{L}]_{\alpha\alpha}$, we obtain the probability that, starting in any node labeled by $\alpha$, we reach a node labeled by $\beta$ through  $n$ consecutive steps in $\tg$. As a shortcut notation, we call the result $[\tg.\Lambda^n]_{\alpha\beta}$. Since there may be different nodes labeled by $\alpha$, we need to normalize the resulting probabilities. This is obtained with the division by $L\transp{L}$, which does so by assuming a uniform distribution when picking from which specific $\alpha$-labeled node one wants to start. Notice that these calculations need to be refined so as to consider proper runs and  traces. This will be done in Section~\ref{subsec:as}. %\todo{Rimandare alla sezione giusta}



%
%$\texttt{\color{blue}i}\overset{n}{\rightsquigarrow}\texttt{\color{blue}j}$ of length $n$: therefore, $[\Lambda^n]_{\color{green}\alpha\beta}:=[LR^nL^t]_{\color{green}\alpha\beta}/[LL^t]_{\color{green}\alpha\alpha}$ denotes the probability that, having started at any node labeled $\color{green}\alpha$ and taking $n$ steps, we arrive at any node labeled $\color{green}\beta$ (${\color{green}\alpha}\overset{n}{\rightsquigarrow}{\color{green}\beta}$). We denote as ${\color{green}\alpha}{\rightsquigarrow}{\color{green}\beta}$ an aforementioned path of arbitrary length.
%We can also associate a weight $\omega\in[0,1]\subseteq\mathbb{R}$ to a TG, so to express the probability associated with the TG itself as valid.
%




%\begin{example}
%We can graphically represent such TG as in \cite{Myers1989}.
%Figure \ref{fig:orig} is a  TG $P^*=(\mathtt{\color{blue}1},\mathtt{\color{blue}8},L,R,1)$ where $\omega=1$, where the matrices $L$ and $R$ can be both defined as follows:
%$$L:=\kbordermatrix{
%             & \texttt{\color{blue}1}&\texttt{\color{blue}2}&\texttt{\color{blue}3}&\texttt{\color{blue}4}&\texttt{\color{blue}5}&\texttt{\color{blue}6}&\texttt{\color{blue}7}&\texttt{\color{blue}8}&\texttt{\color{blue}9}&\texttt{\color{blue}10}\\
%\color{green}\varepsilon  & \textbf{1}&0&0&0&0&0&\textbf{1}&\textbf{1}&\textbf{1}&\textbf{1}\\
%\color{green}a            & 0&\textbf{1}&0&\textbf{1}&0&\textbf{1}&0&0&0&0\\
%\color{green}b            & 0&0&0&0&\textbf{1}&0&0&0&0&0\\
%\color{green}c            & 0&0&\textbf{1}&0&0&0&0&0&0&0\\
%}\qquad R:=\kbordermatrix{
%& \texttt{\color{blue}1}&\texttt{\color{blue}2}&\texttt{\color{blue}3}&\texttt{\color{blue}4}&\texttt{\color{blue}5}&\texttt{\color{blue}6}&\texttt{\color{blue}7}&\texttt{\color{blue}8}&\texttt{\color{blue}9}&\texttt{\color{blue}10}\\
%\texttt{\color{blue}1}  & 0&0&{\color{red}p_2}&0&0&0&0&0&{\color{red}p_1}&0\\
%\texttt{\color{blue}2}  & 0&0&0&0&0&{\color{red}p_3}&{\color{red}p_6}&0&0&0\\
%\texttt{\color{blue}3}  & 0&0&0&0&0&0&0&0&0&{\color{red}1}\\
%\texttt{\color{blue}4}  & 0&0&0&0&0&{\color{red}p_3}&{\color{red}p_6}&0&0&0\\
%\texttt{\color{blue}5}  & 0&0&0&0&0&0&0&{\color{red}1}&0&0\\
%\texttt{\color{blue}6}  & 0&0&0&0&0&{\color{red}p_3}&{\color{red}p_6}&0&0&0\\
%\texttt{\color{blue}5}  & 0&0&0&0&0&0&0&{\color{red}1}&0&0\\
%\texttt{\color{blue}8}  & 0&0&0&0&0&0&0&0&0&0\\
%\texttt{\color{blue}9}  & 0&{\color{red}1}&0&0&0&0&0&0&0&0\\
%\texttt{\color{blue}10}  & 0&0&0&{\color{red}p_4}&{\color{red}p_5}&0&0&0&0&0\\
%}$$
%\end{example}

% Given a TG $P=(s,t,L,R,\omega)$, a trace $\tau$ is a tuple in $(\Sigma\backslash\{\varepsilon\})^*$ denoting a path always originating from $s$ and terminating in $t$.


\subsection{Kernels and Trace Kernels}\label{subsec:katk}
As a foundational basis to compute trace alignments, we adapt similarity measures from the database literature.  Given a set of data examples $\mathcal{X}$, (e.g., strings or traces, transition graphs) a (positive definite) \emph{kernel} function $k\colon \mathcal{X}\times \mathcal{X}\to \mathbb{R}$ denotes the similarity of elements in $\mathcal{X}$. If $\mathcal{X}$ is the $d$-dimensional Euclidean Space $\mathbb{R}^d$, the simplest kernel function is the inner product $\Braket{\mathbf{x},\mathbf{x}'}=\sum_{1\leq i\leq d}\mathbf{x}_i\mathbf{x}'_i$.
A kernel is said to \emph{perform ideally} \cite{Gartner03} when $k(x,x')=1$ whenever $x$ and $x'$ are the same object (\textit{strong equality}) and $k(x,x')=0$ whenever $x$ and $x'$ are distinct objects (\textit{strong dissimilarity}). A kernel is also said to be \emph{appropriate} when similar elements $x,x'\in\mathcal{X}$ are also close in the feature space. Notice that appropriateness can be only assessed  empirically \cite{Gartner03}.
A positive definite kernel induces a distance metric as:
\begin{equation}\label{eq:dofk}
d_k(\mathbf{x},\mathbf{x}'):=\sqrt{k(\mathbf{x},\mathbf{x})-2k(\mathbf{x},\mathbf{x}')+k(\mathbf{x}',\mathbf{x}')}
\end{equation}
When the kernel of choice is the inner product, the resulting distance is the Euclidean distance $\norm{\mathbf{x}-\mathbf{x}'}{2}$. A normalized vector $\hat{\mathbf{x}}$ is defined as $\mathbf{x}/\norm{\mathbf{x}}{2}$. For a normalized vector we can easily prove that: $\norm{\hat{\mathbf{x}}-\hat{\mathbf{x}}'}{2}^2=2(1-\Braket{\hat{\mathbf{x}},\hat{\mathbf{x}}'})$.

When $\mathcal{X}$ does not represent directly a $d$-dimensional Euclidean space, we can use an \emph{embedding} $\embed\colon\mathcal{X}\to \mathbb{R}^d$ to define a kernel $k_\embed\colon \mathcal{X}\times \mathcal{X}\to\mathbb{R}$ as $k_\embed(x,x'):=\Braket{\embed(x),\embed(x')}$. As a result, $k_\embed(x,x')=k_\embed(x',x)$ for each $x,x'\in\mathcal{X}$.

The literature also provides a kernel representation for strings \cite{LodhiSSCW02,GartnerFW03}, which we can directly employ for our traces. We now provide an intuition describing the desired features of this representation \cite{LodhiSSCW02}. If we associate each dimension in $\mathbb{R}^d$ to a different subtrace $\alpha\beta$ of size $2$ (i.e., $2$-grams\footnote{\label{fn:caveat}For our experiments, we choose to consider only $2$-grams, but any $p$-grams of arbitrary length $p\geq 2$ might be adopted \cite{Gartner03}. An increased size of $p$ improves precision but also incurs in a worse computational complexity, as it requires to consider all the arbitrary subtraces of length $p$ whose constitutive elements occur at any distance from each other within the trace.}), the associated coordinate should represent how frequently and ``compactly'' this subtrace is embedded in the trace $\trace$ of interest. Therefore, we introduce a \emph{decay factor} $\lambda\in[0,1]\subseteq\mathbb{R}$ that, for all $m$ subtraces where $\alpha$ and $\beta$ appear in $\trace$ at the same relative distance $L < |\trace|$, weights the resulting embedding as $\lambda^Lm$.

We need to lift this approach so as to consider all occurrences of subtraces $\alpha\beta$ at every distance between $1$ and $|\trace|-1$. To do so, we proceed in two steps. First, we encode $\trace$ into a ``linear'' transition graph $\tg_\trace$ (\figurename~\ref{fig:taustar}) in the obvious way. %\todo{Tagliare dopo i due punti se necessario.} each node in $G_\sigma.V$  corresponds to an element of the trace labeled correspondingly, and the nodes representing two consecutive elements in the trace are connected with a transition probability of 1 (whereas in all the other cases, the probability is 0).
As a second step, we rely on the matrix operations to calculate a simplified version of the embedding defined in \cite{LodhiSSCW02} as $\trembed_{\alpha\beta}(\trace)=\sum_{1\leq i\leq |\trace|}\lambda^i[(\tg_{\trace}.\Lambda)^i]_{\alpha\beta}$. %\todo{No spazio per spiegare cosa succede...}
%This value can be seen as a reward.
The kernel between two traces corresponds to the sum of the products of such values calculated 2-gram by 2-gram for the two traces.
%, namely it is equal to the \emph{kernel convolution}. %\todo{L'ho provato a scrivere intuitivamente, ma non e' chiaro da dove arrivi questo modo di calcolarlo... deriva dalle formule sopra ma la digressione in mezzo e' lunga. Come possiamo fare per chiarire? L'esempio spiega bene tutto!}
This trace kernel returns strong dissimilarity when the two traces have no shared 2-grams at any arbitrary occurring length, but does not enjoy strong equality (as the similarity of a trace with itself is at least $\lambda^2$ - returned when the trace is a 2-gram).

%
%we can represent it as a TG \cite{Myers1989} $(1,{|\tau|},L_\tau,R_\tau,1)$ having $[L_\tau]_{{\color{green}\alpha}\texttt{\color{blue}i}}=1\Leftrightarrow \tau_{\texttt{\color{blue}i}}={\color{green}\alpha}$ and $[L_\tau]_{{\color{green}\alpha}\texttt{\color{blue}i}}=0$ otherwise, and $\forall i<|\tau|.\; [R_\tau]_{\texttt{\color{blue}i(i+1)}}=1 $ and $[R_\tau]_{\texttt{\color{blue}ij}}=0$ otherwise.
%Exploiting this encoding, we can adopt a simplified version of the embedding defined in \cite{LodhiSSCW02,Raedt} as $\embed_{\mathcal{T}}(\tau)_{{\color{green}\alpha\beta}}=\sum_{1\leq i\leq |\tau|}\lambda^i[(\Lambda_\tau)^i]_{\color{green}\alpha\beta}$.
%Please note that this definition is similar to a transition matrix embedding proposed in \cite{GartnerFW03} via geometric series, that is $\sum_i\lambda^i[R^i]_{\color{green}\alpha\beta}$.

\begin{figure}[!t]
	\centering
	\includegraphics[width=.4\textwidth]{images/taustar.pdf}
	\caption{Graphical representation of the transition graph encoding trace $\const{caba}$.}\label{fig:taustar}
\end{figure}
%
%\begin{example}\label{ex:tracembed}
%	{Let us suppose that we want to align a trace $\tau^*$ to one of the traces from a transition graph: in order to carry out an approximate alignment, we need to transform it to a transition graph first.} A trace $\tau^*=\textup{caba}$ can be graphically represented in Figure \ref{fig:taustar}. The associated TG $T=(\mathtt{\color{blue}1},\mathtt{\color{blue}4},L,R,1)$ has matrices $L$ and $R$  defined as follows:
%	$$L:=\kbordermatrix{
%		& \texttt{\color{blue}1}&\texttt{\color{blue}2}&\texttt{\color{blue}3}&\texttt{\color{blue}4}\\
%		\color{green}a            & 0&\textbf{1}&0&\textbf{1}\\
%		\color{green}b            & 0&0&\textbf{1}&0\\
%		\color{green}c            & \textbf{1}&0&0&0\\
%	}\qquad R:=\kbordermatrix{
%		& \texttt{\color{blue}1}&\texttt{\color{blue}2}&\texttt{\color{blue}3}&\texttt{\color{blue}4}\\
%		\texttt{\color{blue}1}  & 0&\color{red}1&0&0\\
%		\texttt{\color{blue}2}  & 0&0&\color{red}1&0\\
%		\texttt{\color{blue}3}  & 0&0&0&\color{red}1\\
%		\texttt{\color{blue}4}  & 0& 0& 0& 0\\
%	}$$
%We can similarly represent all the traces from the USPN.
%\end{example}

%\begin{example}
%The subtrace \textit{\textbf{\uline{hi}}} is represented in \textit{\textbf{\uline{hi}}deous},   \textit{\uline{\textbf{h}}e\uline{{i}}d\textbf{i}}, and \textit{\uline{{\textbf{h}i}}nd\textbf{i}}, but with different frequencies and subtrace distances. We have $\embed_{\mathcal{T}}(\textit{hideous})_{{\color{green}hi}}=\lambda$,  $\embed_{\mathcal{T}}(\textit{heidi})_{{\color{green}hi}}=\lambda^2+\lambda^4$, and $\embed_{\mathcal{T}}(\textit{hindi})_{{\color{green}hi}}=\lambda+\lambda^4$.
%\end{example}



\begin{table}[t!]
\vspace{+0.5cm}
\caption{Embedding of traces $\const{caba}$, $\const{caa}$ and $\const{cb}$.}\label{tb:embedding}
\vspace{-0.4cm}
\begin{center}
\scalebox{0.45}{
	\begin{tabularx}{\textwidth}{
>{\hsize=.1\hsize}X
>{\hsize=.2\hsize}X
>{\hsize=.1\hsize}X
>{\hsize=.1\hsize}X
>{\hsize=.1\hsize}X
>{\hsize=.1\hsize}X
>{\hsize=.1\hsize}X
>{\hsize=.25\hsize}X
>{\hsize=.2\hsize}X
>{\hsize=.1\hsize}X
}
		\toprule
		& $\const{aa}$    & $\const{ab}$   & $\const{ac}$    & $\const{ba}$   & $\const{bb}$   & $\const{bc}$ & $\const{ca}$ & $\const{cb}$ & $\const{cc}$   \\
		\midrule
		$\const{caba}$ & $\lambda^2$ & $\lambda$ & $0$ & $\lambda$  & $0$  & $0$ & $\lambda+\lambda^3$ & $\lambda^2$ & $0$\\
		%$\const{caaa}$ & $2\lambda+\lambda^2$& $0$ & $0$ & $0$ & $0$ & $0$ & $\lambda+\lambda^2+\lambda^3$ & $0$ & $0$ \\
		$\const{caa}$  & $\lambda$ & $0$ & $0$ & $0$ & $0$ & $0$ & $\lambda+\lambda^2$ & $0$&  $0$\\
		$\const{cb}$   & $0$ & $0$ & $0$ & $0$ & $0$ & $0$ & $0$ & $\lambda$& $0$ \\
		\bottomrule
	\end{tabularx}
}
\vspace{-0.8cm}
\end{center}
\end{table}
\begin{example}\label{ex:wheredotiszero} %\small
Consider tasks $\tasks=\Set{a,b,c}$. The possible 2-grams over $\tasks$ are $\tasks^2=\Set{\const{aa},\const{ab},\const{ac},\const{ba},\const{bb},\const{bc},\const{ca},\const{cb},\const{cc}}$. Table~\ref{tb:embedding} shows the embeddings of some traces. Being a 2-gram, trace $\const{cb}$ has only one nonzero component, namely that corresponding to itself, with $\trembed_{\const{cb}}(\const{cb})=\lambda$. Trace $\const{caa}$ has the 2-gram $\const{ca}$ occurring with length $1$ ($\const{\underline{ca}a}$) and $2$ ($\const{\underline{c}a\underline{a}}$), and the 2-gram $\const{aa}$ with occurring length $1$ ($\const{c\underline{aa}}$). Hence: $\trembed_{\const{ca}}(\const{caa})=\lambda+\lambda^2$ and  $\trembed_{\const{aa}}(\const{caa})=\lambda$.  Similar considerations can be carried out for the other traces in the table.
We now want to compute the similarity between the first trace $\const{caba}$ and the other two traces. To do so, we sum, column by column (that is, 2-gram by 2-gram) the product of the embeddings for each pair of traces. We then get $k_{\trembed}(\const{caba},\const{caa})=\lambda^3+(\lambda+\lambda^3)(\lambda+\lambda^2)$ and $k_{\trembed}(\const{caba},\const{cb})=\lambda^3
$,
%{\footnotesize
%\[
%k_{\trembed}(\const{caba},\const{caaa})=\lambda(\lambda+\lambda^2+\lambda^3)
%~~
%k_{\trembed}(\const{caba},\const{caa})=\lambda(\lambda+\lambda^2)
%~~
%k_{\trembed}(\const{caba},\const{cb})=\lambda(\lambda+\lambda^3)
%\]}
which induces ranking $
k_{\trembed}(\const{caba},\const{caa})>
k_{\trembed}(\const{caba},\const{cb})
$.
\end{example}

\endinput
\subsection{Graph Embedding}\label{ssec:ge}
Graph kernels allow mapping graph data structures to feature spaces (usually an Euclidean space in $\mathbb{R}^n$ for $n\in \mathbb{N}_{>0}$) \cite{Samatova} so to express graph similarity functions that can then be adopted for both classification \cite{TsudaS10} and clustering algorithms. One of the first approaches used in literature involved the definition of topological description vectors \cite{Sidere} for each graph in a graph database, for then defining the graph similarity function as an inner product of their associated vectors. One inconvenience of such a technique is that it is required to perform NP-complete subgraph isomorphisms among a collection of graphs. It has been already proved that the definition of a graph kernel function fully recognizing the structure the graph always boils down to solving such NP-Complete problem \cite{GartnerFW03}, as exact embeddings generable in polynomial can be inferred just for loop-free Direct Acyclic Graphs \cite{BergamiBM20}.


Consequently, most recent literature focused on extracting relevant features of such graphs, that are then used to define a graph similarity function. The most common approach adopted in the kernel to extract such features is called \textit{propositionalization}: we might extract all the possible features (e.g., subsequences), and then define a kernel function based on the occurrence and similarity of these features \cite{Gartner03}.

%\section{LTL over Finite Traces and the Declare Framework}
%\label{sec:preliminaries}
%As a formal basis for specifying crisp (temporal) business constraints, we adopt the customary choice of Linear Temporal Logic over finite traces (\LTLf \cite{DeVa13,DDGM14}). This logic is at the basis of the well-known \declare \cite{PeSV07} constraint-based process modeling language.
%We provide here a gentle introduction to this logic and to the \declare framework.
%
%\subsection{Linear Temporal Logic over Finite Traces}
%
%$\LTLf$ has exactly the same syntax as standard $\LTL$, but, differently from $\LTL$, it interprets formulae over an unbounded, yet finite linear sequence of states. Given an alphabet $\Sigma$ of atomic propositions (in our setting, representing activities), an \LTLf formula $\varphi$ is built by extending propositional logic with temporal operators:
%\[\varphi ::= a \mid \lnot \varphi \mid \varphi_1\lor \varphi_2
% \mid \Next\varphi \mid \varphi_1\Until\varphi_2 \quad \text{ where $a \in \Sigma$.}\]
%
%
%%The semantics of \LTLf is given in terms of \emph{finite traces}
%%denoting finite, \emph{possibly empty}, sequences
%%$\tau=\tau_0,\ldots,\tau_n$ of elements from the alphabet $\Sigma$. The evaluation of a formula is done in a given state (i.e., position) of the trace.
%
%
%The semantics of \LTLf is given in terms of \emph{finite traces} denoting finite, \emph{possibly empty} sequences $\tau=\tup{\tau_0, \ldots, \tau_n}$ of elements of $2^\Sigma$, containing all possible propositional interpretations of the propositional symbols in $\Sigma$. In the context of this paper, consistently with the literature on business process execution traces, we make the simplifying assumption that in each point of the sequence, one and only one element from $\Sigma$ holds. Under this assumption, $\tau$ becomes a total sequence of activity occurrences from $\Sigma$, matching the standard notion of (process) execution trace. We indicate with $\tasks^*$ the set of all traces over $\tasks$. The evaluation of a formula is done in a given state (i.e., position) of the trace, and we use the notation $\tau,i\models \varphi$ to express that $\varphi$ holds in the position $i$ of $\tau$. We also use $\tau \models \varphi$ as a shortcut notation for $\tau,0\models\varphi$. This denotes that $\varphi$ holds over the entire trace $\tau$ starting from the very beginning and, consequently, logically captures the notion of \emph{conformance} of $\tau$ against $\varphi$. We also say that $\varphi$ is \emph{satisfiable} if it admits at least one conforming trace.
%
%%We start by giving an intuitive account of the resulting semantics. In the syntax above, operator $\Next$ denotes the \emph{next state} operator, and $\Next \varphi$ is true if $\varphi$ is true is true now if there exists a next state (i.e., the current state is not at the end of the trace), and in the next state $\varphi$ holds. Operator $\Until$ instead is the \emph{until} operator, and $\varphi_1\Until\varphi_2$ is true if $\varphi_1$ holds now and continues to hold until eventually, in a future state, $\varphi_2$ holds. From the given syntax we can derive the usual boolean operators $\land$ and $\rightarrow$, the two formulae $\true$ and $\false$, as well also additional temporal operators. We consider in particular the following three:
%%\begin{compactitem}[$\bullet$]
%%\item (eventually) $\Diamond \varphi = \true \Until \varphi$ is true if there is a future state where $\varphi$ holds;
%%\item (globally) $\Box \varphi = \neg \Diamond \neg \varphi$ is true if now and in all future sates $\varphi$ holds;
%%\item (weak until) $\varphi_1 \Wntil \varphi_2 = \varphi_1\Until\varphi_2 \lor \Box \varphi_1$ relaxes the until operator by admitting the possibility that $\varphi_2$ never becomes true, in this case by requiring that is true if $\varphi_1$ holds now and in all future states.
%%\end{compactitem}
%% To define the semantics formally, we denote the length of trace $\tau$ as $\length(\tau) =  n+1$.
%
%
%In the syntax above, operator $\Next$ denotes the \emph{next state} operator, and $\Next \varphi$ is true if there exists a next state (i.e., the current state is not at the end of the trace), and in the next state $\varphi$ holds. Operator $\Until$ instead is the \emph{until} operator, and $\varphi_1\Until\varphi_2$ is true if $\varphi_1$ holds now and continues to hold until eventually, in a future state, $\varphi_2$ holds. From these operators, we can derive the usual boolean operators $\land$ and $\rightarrow$, the two formulae $\true$ and $\false$, as well as additional temporal operators. We consider, in particular, the following three:
%\begin{compactitem}[$\bullet$]
%\item (eventually) $\Diamond \varphi = \true \Until \varphi$ is true if there is a future state where $\varphi$ holds;
%\item (globally) $\Box \varphi = \neg \Diamond \neg \varphi$ is true if now and in all future states $\varphi$ holds;
%\item (weak until) $\varphi_1 \Wntil \varphi_2 = \varphi_1\Until\varphi_2 \lor \Box \varphi_1$ relaxes the until operator by admitting the possibility that $\varphi_2$ never becomes true, in this case by requiring that $\varphi_1$ holds now and in all future states.
%\end{compactitem}
%%We write $\tau \models \varphi$ as a shortcut notation for $\tau,0\models \varphi$, and say that formula $\varphi$ is \emph{satisfiable}, if there exists a trace $\tau$ such that $\tau \models \varphi$.
%
%\begin{example}
%The $\LTLf$ formula $\Box(\activity{accept} \rightarrow \Diamond\activity{pay})$ models that, whenever an order is accepted, then it is eventually paid. The structure of the formula follows what is called \emph{response template} in \declare.
%\end{example}
%
%%Every $\LTLf$ formula $\varphi$ can be translated into a corresponding standard finite-state automaton $\aut_\varphi$ that accepts all and only those finite traces that satisfy $\varphi$ \cite{DeVa13,DDGM14}. Although the complexity of reasoning with $\LTLf$ is the same as that of $\LTL$, finite-state automata are much easier to manipulate in comparison with B\"uchi automata, which are necessary when formulae are interpreted over infinite traces.
%
%\subsection{Declare}
%\begin{table}[t]
\caption{Some \declare templates, their textual and graphical representation, the corresponding \LTLf formalization and the \LTLf formula capturing their complement (i.e., their logical negation).
\label{tab:constraints}}
\centering
\begin{adjustbox}{width=0.9\textwidth,center}
\begin{tikzpicture}
  \matrix[  nodes={node distance=\nodedist,minimum height=6mm},
            rectangle, draw,
            nodes in empty cells,
            row sep=2mm,column sep=3mm,
            very thick,
            column 1/.style={anchor=west,font=\footnotesize},
            column 2/.style={anchor=west,xshift=1mm,font=\footnotesize},
            column 3/.style={anchor=west,xshift=1mm,font=\footnotesize},
            column 4/.style={anchor=west},
            >=latex,->,
          ] (declarematrix) {
    \node {\textsc{text}};
    &
    \node[yshift=.5mm] {\textsc{notation}};
  &
    \node {\textsc{\LTLf\ formula ($\varphi$)}};
    &
    \node[yshift=.5mm] {\textsc{complement ($\neg\varphi$)}};
    \\
    \node {
      \begin{tabular}{@{}l@{}}
      \constraint{existence($\mathit{a}$)} \\
      \end{tabular}
    };
    &
    \node[smalltask,xshift=1.5mm] (a) {$\mathit{a}$};
    \node[taskfg,above=-1mm of a,xshift=-.3mm]{\footnotesize $1..\ast$};
    &
    \node {$\Diamond {a}$};
    &
    \node {$\Box \neg {a}$};

    \\
     \node {
      \begin{tabular}{@{}l@{}}
        \constraint{absence2($\mathit{a}$)}\\
      \end{tabular}
    };
    &
    \node[smalltask,xshift=1.5mm] (a) {$\mathit{a}$};
    \node[taskfg,above=-1mm of a,xshift=-.3mm]{\footnotesize $0..1$};

    &
    \node {$\neg \Diamond ({a} \land \Next \Diamond {a})$};
    &
    \node {$\Diamond ({a} \land \Next \Diamond {a})$};
    \\
    \node {
      \begin{tabular}{@{}l@{}}
        \constraint{response($\mathit{a}$,$\mathit{b}$)}
      \end{tabular}
    };
    &
    \node[smalltask,xshift=1.5mm] (a) {$\mathit{a}$};
    \node[above=-1mm of a,xshift=-.3mm]{\footnotesize $~$};
    \node[smalltask,right=\taskdist of a] (b) {$\mathit{b}$};
    \path[response,very thick] (a) -- (b);
    &
    \node {$\Box ({a} \rightarrow \Diamond {b})$};
    &
    \node {$\Diamond ({a} \land \Box \neg {b})$};
    \\
    \node {
      \begin{tabular}{@{}l@{}}
        \constraint{precedence($\mathit{a}$,$\mathit{b}$)}\\
      \end{tabular}
    };
    &
    \node[smalltask,xshift=1.5mm] (a) {$\mathit{a}$};
    \node[smalltask,right=\taskdist of a] (b) {$\mathit{b}$};
    \path[precedence,very thick] (b) -- (a);
    &
    \node {$\neg {b} \Wuntil {a}$};
    &
    \node {$\neg {a} \Until {b}$};
    \\
    \node {
      \begin{tabular}{@{}l@{}}
        \constraint{not-coexistence($\mathit{a}$,$\mathit{a}$)}\\

      \end{tabular}
    };
    &
    \node[smalltask,xshift=1.5mm] (a) {$\mathit{a}$};
    \node[smalltask,right=\taskdist of a] (b) {$\mathit{b}$};
    \path[notcoexistence,very thick] (a) -- (b);
    &
    \node {$\neg(\Diamond {a} \land \Diamond {b})$};
    &
    \node {$\Diamond {a} \land \Diamond {b}$};
    \\
%    \node {
%      \begin{tabular}{@{}l@{}}
%        \constraint{neg-response(\activity{a},\activity{b})}\\
%        $\Box( \activity{a} \limp \neg \bigcirc\Diamond \activity{b})$
%      \end{tabular}
%    };
%    &
%    \node[smalltask,xshift=1.5mm] (a) {\activity{a}};
%    \node[smalltask,right=\taskdist of a] (b) {\activity{b}};
%    \path[negationresponse,very thick] (a) -- (b);
%    \\
    };
\end{tikzpicture}
\end{adjustbox}
\end{table} 
%\declare\ \cite{PeSV07} is a declarative process modeling language based on \LTLf. More specifically, a \declare model fixes a set of activities, and a set of constraints over such activities, formalized using \LTLf formulae. The overall model is then formalized as the conjunction of the \LTLf formulae of its constraints.
%
%Among all possible \LTLf formulae, \declare selects some pre-defined patterns. Each pattern is represented as a \declare template, i.e., a formula with placeholders to be substituted by concrete activities to obtain a constraint. Constraints and templates have a graphical representation; Table~\ref{tab:constraints} lists the \declare templates used in this paper. A \declare model is then graphically represented by showing its activities, and the application of templates to such activities (which indicates how the template placeholders have to be substituted to obtain the corresponding constraint).
%
%%Automata-based techniques for $\LTLf$ have been adopted to tackle fundamental tasks within the lifecycle of \declare processes, such as consistency checking \cite{PeSV07,MPVC11}, enactment and monitoring \cite{PeSV07,MMWV11,DDGM14}, and discovery support \cite{MaCV12}.
%
%
%
%
%\begin{example}
%\label{ex:inconsistency}
%Consider the following \declare model, constituting a (failed) attempt of capturing a fragment of an order-to-shipment process:
%
%\begin{center}
%  \resizebox{3.2cm}{!}{
%        \begin{tikzpicture}
%        \node[task] (accept) {\accept};
%        \node[task,right=of accept] (reject) {\reject};
%        \node[left=0mm of accept,taskfg] {1..*};
%        \node[right=0mm of reject,taskfg] {1..*};
%        \draw[notcoexistence] (accept) -- (reject);
%    \end{tikzpicture}
%  }
%\end{center}
%
%The model indicates that there are two activities to accept or reject an order, that these two activities are mutually exclusive, and that both of them have to be executed.
%%  \begin{wrapfigure}[13]{l}{42mm}
%%  \end{wrapfigure}
%These constraints are obviously contradictory and, in fact, the model is inconsistent, since its \LTLf formula
%$
%\Diamond \accept \land \Diamond \reject \land \neg (\Diamond \accept \land \Diamond \reject)
%$
%is unsatisfiable.
%\end{example}
%
%
%
%\endinput
%
%\smallskip\noindent\textbf{\declare} is a constraint-based process modeling language based on \LTLf. Differently from imperative process modeling languages,
%\declare models a process by fixing a set of activities, and defining a set of
%\emph{temporal constraints} over them, accepting every execution trace that satisfies all constraints.
%Constraints are specified via pre-defined \LTLf templates, which come with a corresponding
%graphical representation (see Table~\ref{tab:constraints} for the \declare patterns we use in this paper).
%For the sake of generality, in this paper we consider arbitrary \LTLf formulae as constraints. However, in the examples we consider formulae whose templates can be represented graphically in \declare.
%
%
%
%Automata-based techniques for $\LTLf$ have been adopted in \declare to tackle fundamental tasks within the lifecycle of Declare processes, such as consistency checking \cite{PeSV07,MPVC11}, enactment and monitoring \cite{PeSV07,MMWV11,DDGM14}, and discovery support \cite{MaCV12}.

% !TeX root=../main.tex
\begin{figure}[!t]
	\hspace*{-4cm}\includegraphics[width=1.7\textwidth]{images/pipeline}
	\caption{Proposed pipeline to assess the Probabilistic Trace Alignment}\label{fig:pipe}
\end{figure}


\section{Probabilistic Trace Alignment Pipeline}
We propose the pipeline from Figure \ref{fig:pipe} connecting several existing formalisations via intermediate processing steps. 
The input of this pipeline is a query trace to be aligned, an USWN, and a minimum probability threshold $p_\theta$. Its output 
is a set of model traces satisfying $p_\theta$ with an alignment ranking.
%
The pipeline has the following phases: after representing the USWN as a graph of all the sequentially scheduled transitions 
(\S\ref{sec:seqZ}), we shift the labels from the edges towards the nodes while preserving the set of probabilistic traces 
(\S\ref{sec:LSift}) and minimize the graph representation by removing the $\tau$-labelled nodes while preserving the 
trace probability (\S\ref{sec:clos}). We extract the set of all traces having probability above $p_\theta$ threshold (\S\ref{sec:unrav}) and  apply two different alignment strategies; one exact  (\S\ref{subsec:eta}), and one approximated. 


We later discuss how to rank traces in the exact and approximated scenarios by reducing the alignment process to a k-nearest 
neighbour problem. While the exact trace alignment requires to perform the alignment process each time a novel trace $\sigma^*$ is 
introduced (\S\ref{subsec:exbkptap}), the approximated alignment can split the alignment into a preliminary loading phase and a 
query phase. In the former, each stochastic trace from the USWN is represented as a vector (\S\ref{subsec:ate}), and in the latter the to-be-aligned trace $\sigma^*$ is first represented as a vector and then compared to all the other vectorial representations.  

\subsection{Sequentialization}\label{sec:seqZ}
The sequentialization step transforms a USWN with an initial marking $M$ into a Reachability Graph $(\mathcal{M},\mathcal{E})$ 
generated by a sequentialization process, where potentially concurrent firing transitions are represented via a sequential scheduling. 

\begin{definition}[Reachability Graph]
	Given an initial marking $M$ for a USWN $\mathcal{U}$,  the \textit{Reachability Graph} for $\mathcal{U}$ is a graph 
	$(\mathcal{M},\mathcal{E})$ where the nodes  $\mathcal{M}$ are composed of all the reachable markings from $M$, 
	and the edges $\mathcal{E}$ are induced by the aforementioned relation $M\overset{t}{\to}M'$ among the 
	nodes. To each edge $M\overset{t}{\to}M'$, we associate a transition probability $\mathbb{P}\left(M\overset{t}{\to}M'\right)=\frac{W(t)}{\sum_{t'\in E(M)}W(t')}$ \cite{spdwe}. 
\end{definition}

\begin{example}
From the USWN in Figure \ref{fig:spn}, the sequentialization process generates the reachability graph depicted in 
Figure \ref{fig:rg}. Each node represents a marking $M$ as a vector, and the edges are labelled with the firing transitions. 
The edges associated to this graph describe potentially concurrent firing transitions sequentially. While visiting the graph from 
$M$, the chaining of the edge labels generates a trace produced from the Untimed Workflow Net, and the product of the edge 
weights provides the probability associated to the trace.
\end{example}



\subsection{Label Shifter}\label{sec:LSift}
Reachability graphs obtained via sequentialization cannot be directly embedded using existing methods:  Reachability graphs 
associated to stochastic workflow nets are edge labelled, but TGs are node labelled. To represent the former as the latter, we 
shift the labels from the edges to the nodes  while preserving the set of traces and their associated probability. 
Such transformation is defined next.

\begin{definition}[Label Shifter]\label{def:transf}
The reachability graph $(\mathcal{M},\mathcal{E})$ generated from an initial marking $M$, is transformed into the TG $(s,t,L,R,1)$, where:
\begin{itemize}
	\item If is a single edge $M_1\overset{t}{\to}M_2\in\mathcal{E}$ where $M_1=M$, then $s=M\overset{t}{\to}M_2$; otherwise, define a new node $\textbf{i}$ and set it as the initial node for TG: $s=\textbf{i}$.
	\item If is a single edge $M_1\overset{t}{\to}M_2\in\mathcal{E}$ without outgoing edges in the reachability graph, then $t=M_1\overset{t}{\to}M_2$; otherwise, define a new node $\textbf{f}$ and set it as the accepting node for TG:  $t=\textbf{f}$.
	\item $[L]_{\lambda(t),\;M\overset{t}{\to} M'}=1$ for each $M\overset{t}{\to} M'\in\mathcal{E}$; if $\textbf{i}$ is defined then $[L]_{\tau\textbf{i}}=1$; if $\textbf{f}$ is defined, then $[L]_{\tau\textbf{f}}=1$; $[L]_{ij}=0$ otherwise.
	\item $[R]_{M\overset{t}{\to} M',\;M'\overset{t'}{\to} M''}=\frac{W(t')}{\sum_{\textbf{t}\in E(M')}W(\textbf{t})}$ for each $M\overset{t}{\to} M',M'\overset{t'}{\to} M''\in\mathcal{E}$; if $\textbf{i}$ is defined, $[R]_{\textbf{i},\;M\overset{t}{\to}M'}=\frac{W(t)}{\sum_{\textbf{t}\in E(M)}W(\textbf{t})}$; if $\textbf{f}$ is defined, then $[R]_{M\overset{t}{\to}M',\;\textbf{i}}=1$ for each $M'$ without outgoing edges in the reachability graph; $[R]_{ij}=0$ ow.
\end{itemize}
\end{definition}

\endinput
%
We can show that the TG obtained in Definition \ref{def:transf} preserves the same set of probabilistic traces associated by the reachability graph. The proof is omitted due to the lack of space.

\begin{example}
Figure \ref{fig:lmc} shows the TG obtained from the reachability graph in Figure \ref{fig:rg}. Nodes are labelled with the firing 
transition labels (in green), and edges preserve the probabilistic information from the reachability graph (in red). Intuitively, when a 
new initial node \textit{\textbf{i}} is inserted, we preserve all the initial probabilistic choices that a transition is fired from an initial 
marking $M$, while the intermediate edges inherit the probabilisitc choice of the firing transition from the subsequent choices. When 
a new final node \textit{\textbf{f}} is added, such edges always have probability $1$, and thus do not interfere with the 
initial traces' probability.
\end{example}

\subsection{$\tau$-closure}\label{sec:clos}
The $\tau$-closure process has two main purposes: first, reduce the size of the previously generated TG by removing all 
$\tau$-labelled nodes \texttt{\color{blue}w} and preserving the connection between  the nodes \texttt{\color{blue}u} 
from its ingoing edges   $\texttt{\color{blue}u}\xrightarrow{\color{violet}\rho}\texttt{\color{blue}w}$ with the nodes \texttt{\color{blue}v} from its ingoing edges   $\texttt{\color{blue}w}\xrightarrow{\color{violet}\rho'}\texttt{\color{blue}v}$ by establishing new edges $\texttt{\color{blue}u}\xrightarrow{\color{violet}\rho\rho'}\texttt{\color{blue}v}$. $\tau$-labelled initial (or accepting) nodes are removed iff they have only one outgoing (ingoing) edge with probability $1$.

\begin{example}\todo{Is it now ok?}
	The $\tau$-closure removes the non-initial and non-accepting nodes within an automaton, while preserving the probabilistic trace equivalence of the two automata. Let us suppose to apply the $\tau$-closure to the automata in Figure \ref{fig:orig}: node \texttt{\color{blue}10} is removed alongside its associated edges, and new edges $\texttt{\color{blue}3}\xrightarrow{\color{violet}\rho_{65}}\texttt{\color{blue}4}$ and $\texttt{\color{blue}3}\xrightarrow{\color{violet}\rho_{6f}}\texttt{\color{blue}5}$ are introduced. The resulting TG $P$ is represented with the same graphical depiction Figure \ref{fig:closed}.
\end{example}
%
Consequently, it is always possible to minimize a TG  via $\tau$-closure, so that the only nodes labelled as $\tau$ 
are the source and the target nodes and the set of weighted traces is preserved. From now on, we consider only minimised TGs. 

\subsection{Unraveller}\label{sec:unrav}
%Being that both the graph isomorphism problem is NP-Complete and the 
Since TGs are fully characterized by the set of the probabilistic traces that they generate,  we say that two TGs are 
(probabilistic-trace) equivalent iff they share the same set of weighted traces. In particular, we denote as $\mathcal{W}_p^n(P)$ the set of all the weighted traces in $P$ having at least probability $p$ and maximum length $n$. Under these assumptions, the probabilistic trace equivalence is deterministic.

\begin{example}
	The set $\mathcal{W}_0^{\aleph_0}(P^*)$ of weighted traces of the TG in Figure \ref{fig:orig} is
%	The TG in Figure \ref{fig:orig} has the following set $\mathcal{W}_0^{\aleph_0}(P^*)$ of weighted traces:
$$\set{\braket{\underbrace{\color{green}a\dots a}_{n},{\color{violet}\pa\pc^n\pf}}|n\in \mathbb{N}_{>0}}\cup \set{\braket{{\color{green}c}\underbrace{\color{green}a\dots a}_{n},{\color{violet}\pb\pd\pc^n\pf}}|n\in \mathbb{N}_{>0}}\cup\{\braket{{\color{green}cb},{\color{violet}\pb\pe}}\}$$
After the $\tau$-closure process, $\mathcal{W}_0^{\aleph_0}(P^*)=\mathcal{W}_0^{\aleph_0}(P)$, so the two TGs are (probabilistic-trace) equivalent.
\end{example}


\subsection{Alignment Strategy}

\subsubsection{Optimal-Ranking Trace Aligner}\label{subsec:eta}
{A possible way to carry out probabilistic trace alignment is to reuse existing trace aligners such as \cite{LeoniM17}, where the notion of distance $d(\sigma,\sigma^*)$ between two traces $\sigma$ and $\sigma^*$ (introduced in \S\ref{subsec:katk}) is the cost for aligning one of the two traces with respect to the other when all the possible moves have cost equal to 1.\footnote{This distance is slightly different from the Levenshtein distance since here the replacement is obtained as a concatenation of a deletion and an addition.} We can then exploit decision theory \cite{dectheor} and express the statistical decision of aligning a weighted trace $\braket{\sigma,w_\sigma}\in\mathcal{W}_{p_\theta}(P)$ with $\sigma^*$ as the $w_\sigma$-weighted distance between the two traces, i.e., as $w_\sigma d(\sigma,\sigma^*)$, and rank the alignments based on this metric (this produces what we call an \emph{optimal ranking}).

Note that if we want to represent the same intuition of such a weighted distance as a kernel function (\S\ref{subsec:katk}) so to compare the optimal-ranking strategy to our proposed approximate-ranking trace embedder, we need to transform it as a similarity function returning $1$ when $\sigma=\sigma^*$ and has probability $w_\sigma=1$. In order to do so, we express $d$ as}
%\xout{We can express our probabilistic trace alignment as finding the trace that maximizes both the trace's probability and its similarity with the query trace $\sigma^*$. Still, the trace alignments problems are usually expressed via trace alignments cost functions, and not via trace similarities \cite{LeoniM17}. Given a generic trace cost function $d(\sigma,\sigma')$, it is always possible to convert it into}
a normalized similarity score $s_d(\sigma,\sigma'):=1/(d(\sigma,\sigma')/c+1)$ where $c\in\mathbb{N}_{\neq 0}$ is a constant, so that the maximum similarity of $1$ is reached when the distance is $0$ and the similarity decreases while the distance increases \cite{BergamiBM20}. %\xout{As a consequence, the exact trace aligner will find the weighted trace $\braket{\sigma,\mathbb{P}(\sigma)}$ in $\mathcal{W}_{p_\theta}(P)$ maximising $s_d(\sigma,\sigma^*)$, and use $s_d$ as a ranking function.} \ADD
{Considering that if we interpret the trace to be aligned $\sigma^*$ as a TG, it is reasonable to assume that $w_{\sigma^*}=1$, we can express the golden kernel function (i.e., the one producing the optimal ranking) as $k_\star(\sigma,\sigma^*)=w_\sigma w_{\sigma^*} s_d(\sigma,\sigma^*)$. At this stage, the computation of $\underset{\braket{\sigma,w_\sigma}\in \mathcal{W}_p^n(P)}{\max\arg} k_\star(\sigma, {\sigma^*})$ returns the best approximated trace alignment $\sigma$ for a query trace $\sigma^*$. Therefore, $k_\star$ can be used for reduce the optimal-rank alignment problem as }% \xout{Section \ref{subsec:exbkptap} is going to discuss how such problem might be}
the k-nearest neighbor one { (\S\ref{sec:topk}), and to compare this ranking to the one induced by the approximated vectorial embedding (\S\ref{sec:exp}) proposed in the incoming subsection.}

\subsubsection{Approximate-Ranking Trace Embedder}\label{subsec:ate}
{We want to define an embedding $\phi_\mathcal{P}$ for a dot-product-based kernel $k_{\phi_\mathcal{P}}$ so that it  performs weakly-ideally (\S\ref{subsec:katk} and \cite{Gartner03}). Given that the embedding of a (weighted) trace $\braket{\sigma,w_\sigma}\in\mathcal{W}_{p_\theta}^n(P)$ requires an intermediate $TG$ representation, we can map each of these traces to the subgraph of $P$ generating $\braket{\sigma,w_\sigma}$ with at most $|\sigma|$ steps:}
	\begin{definition}[Trace Embedding for TGs]
		Given a minimum probability\\ threshold $p_\theta$, a maximum path length $n$, and a TG $P=(s,t,L,R,w)$, we generate the set of the trace embeddings for $P$ as follows: for each weighted trace $\braket{\sigma,w_\sigma}\in\mathcal{W}_{p_\theta}^n(P)$ generated from a path $\pi_\sigma=s\to n_2\rightsquigarrow n_m\to t$ over $R$, we generate a TG $P_\sigma=(s',t',L_\sigma,R_\sigma,w_\sigma)$, where \begin{alphalist}
			\item $s'=s$ if $\textit{label}(s)\neq \tau$ and $t'=n_2$ otherwise,
			\item $t'=t$ if $\textit{label}(t)\neq \tau$ and $t'=n_m$ otherwise,
			\item $L_\sigma$ (and $R_\sigma$) is the submatrix of $L$ (and $R$) over the non-$\tau$ labelled notes in $\pi_\sigma$ and the labels from $\sigma$,
			\item $w'$ is initialised by $w$ and then multiplied by $[R]_{s,n_2}$ (and also $[R]_{n_m,t}$) if $\textit{label}(s)=\tau$ (and  $\textit{label}(t)=\tau$).
		\end{alphalist}
	\end{definition}

\begin{table}[!t]
	\caption{Generating the sub-TGs $P_\sigma\in \mathbf{T}_0^4(P)$ from the TG $P$ via the set of weighted traces $\mathcal{W}_0^4(P)$. $l$ and $w$ respectively represent the desired parameter for $\phi_{\mathcal{P}}(P_\sigma)$ and the weight associated to $P_\sigma$.}\label{tab:proj}
	\centering
	\resizebox{.45\textwidth}{!}{\begin{tabular}{>{\centering\arraybackslash} m{1cm}| >{\centering\arraybackslash} m{4cm} >{\centering\arraybackslash} m{1cm} >{\centering\arraybackslash} m{1cm} }
			\toprule
			$\sigma$&$P_\sigma$&$l$&$\omega$\\
			\midrule
			a & \includegraphics{images/trace_a} & $1$ & $\color{red}p_1p_6$\\
			cb & \includegraphics{images/trace_cb} & $2$ & $\color{red}p_2$\\
			aaa & \includegraphics{images/trace_a_loop} & $3$ & $\color{red}p_1p_6$\\
			caa & \includegraphics{images/trace_caa} & $3$ & $\color{red}p_2p_6$\\
			\bottomrule
	\end{tabular}}\quad 	\resizebox{.45\textwidth}{!}{\begin{tabular}{>{\centering\arraybackslash} m{1cm}| >{\centering\arraybackslash} m{4cm} >{\centering\arraybackslash} m{1cm} >{\centering\arraybackslash} m{1cm} }
	\toprule
	$\sigma$&$P_\sigma$&$l$&$\omega$\\
	\midrule
	aa & \includegraphics{images/trace_aa} & $2$ & $\color{red}p_1p_6$\\
	ca & \includegraphics{images/trace_ca} & $2$ & $\color{red}p_2p_6$\\
	\begin{tabular}{l}aaaa\end{tabular} & \includegraphics{images/trace_a_loop} & $4$ & $\color{red}p_1p_6$\\
	caaa & \includegraphics{images/trace_ca_loop} & $4$ & $\color{red}p_2p_6$\\
	\bottomrule
\end{tabular}}
\end{table}
\begin{example}\label{ex:neue}
Given the TG $P$ in Figure \ref{fig:closed}, we assign the following probability values to the edges: $p_1=0.8$, $p_2=0.2$, $p_3=p_6=0.5$, $p_4=0.7$, and $p_5=0.3$. If we limit our analysis to all the traces of maximum length $4$, we generate the following probabilistic traces set:
$$\begin{aligned}
\mathcal{W}^4_0(P)=\{&\braket{a,0.4},\braket{aa,0.2},\braket{aaa,0.1},\braket{ca,0.07},\\
&\braket{cb,0.06},\braket{aaaa,0.05},\braket{caa,0.035},\braket{caaa,0.0175}\}\\
\end{aligned}$$
\medskip

Table \ref{tab:proj} represents the TGs $P_\sigma\in\mathbf{T}_p^n(P)$ with weight $\omega$ generated for each weighted path $\braket{\sigma,w_\sigma}\in \mathcal{W}_0^4(P)$ of length $l=|\sigma|$ from the TG $P$ in Figure \ref{fig:closed}. The associated weight $\omega$ derives from the source's outgoing edges (and target's ingoing edges) when such node is labelled as $\tau$: given that both \textbf{(i)} the embedding strategy from current literature (\S\ref{subsec:katk}) allows trace embedding just for visible (i.e., non-$\tau$) transitions, and that \textbf{(ii)} such trace extraction process discards the $\tau$ information but we want to preserve the transition probabilities from and to $\tau$ transitions, we exploit an additional parameter $\omega$ to preserve such information.
%\xout{selection of a trace $\sigma$ implies performing a set of specific probabilistic choices, we can remove all the edges starting (and arriving) at nodes that are $\tau$-labelled and move their score as the weight of the TG.}
\end{example}
	
{We can now focus on the embedding $\phi_{\mathcal{P}}$ for each TG associated to our selected weighted traces. The main goal is to use $k_{\phi_\mathcal{P}}$ for ranking all the traces generated by the Unravelling step. Walking on such footsteps, we  want to extend  previous work on embeddings (\S\ref{subsec:katk}) by both including the traces' associated probability and by making the ranking induced by $k_{\phi_\mathcal{P}}$ the inverse of the sum of the following distances: the transitions correlations $\epsilon$ and the transitions label frequency $\nu$ (Lemma \ref{lem:approxRank}). }
%\xout{Given that we previously observed that a TGs $P$ can be fully characterised (read, similarity) by their associated set of traces $\mathcal{W}_p^n(P)$  and that the trace embedding can be described as an embedding over a TG, we can characterise a TG embedding as a transition matrix embedding. In addition to that, when two Workflow Nets share similar node labellings but no ${\color{green}\alpha}\rightsquigarrow{\color{green}\beta}$ paths for any ${\color{green}\alpha}$ and ${\color{green}\beta}$, we should combine the former embedding with an embedding characterising the frequency on how the nodes' labels appear in the generated traces.} \ADD
{We would also require that the desired properties of $\phi_{\mathcal{P}}$ are independent from the characterization of $\epsilon$ and $\nu$, which would only provide different embedding strategies. } We can then provide the following definition:


%\begin{table}[!t]
%	\centering
%	\caption{Embedding representation for the TG $P$ in Figure \ref{fig:closed} and the trace $\sigma^*=\textup{caba}$ after representing it as in Figure \ref{fig:sigmastar}. Please note that we restrict $\Sigma_\tau^2$ to the one from $P$.}\label{tab:emb1}
%		\begin{tabular}{l|l|l|l|l|l|l|}
%	\toprule
%	& a    & b                                                   & c    & aa   & ca   & cb   \\
%	\midrule
%	$\phi_{\mathcal{P}}(P)$ & $9.94\cdot10^{-25}$ & $1.18\cdot 10^{-26}$ & $1.04\cdot10^{-25}$ & $4.45\cdot 10^{-25}$ & $6.22\cdot10^{-25}$ & $8.29\cdot10^{-26}$\\
%	$\phi_{\mathcal{P}}(P_{\sigma^*})$ & $8.16\cdot10^{-17}$ & $4.08\cdot 10^{-17}$ & $4.08\cdot10^{-17}$ & $4.37\cdot 10^{-17}$ & $1.03\cdot10^{-16}$ & $4.37\cdot10^{-17}$\\
%	\bottomrule
%\end{tabular}
%\end{table}
\begin{definition}[TG Embedding]\label{def:ppne}
Given a finite set of non-empty labels $\Sigma_\tau =\Sigma\backslash\{\tau\}$, $\Sigma_\tau^2$ denotes all the possible pair of labels associated to paths ${\color{green}\alpha}\rightsquigarrow{\color{green}\beta}$ and $\Sigma_\tau$ denotes the set of all the possible non-$\tau$ node labels. Therefore, it is always possible to enumerate $\Sigma_\tau^2\cup\Sigma_\tau$ via an enumeration by a bijection $\iota\colon \Sigma_\tau^2\cup\Sigma_\tau\to  N$, where $N\subset \mathbb{N}_{\neq 0}$ and $\max N=|N|$.
	
Given a TG $P=(s,t,L,R,w)$ resulting from a $\tau$-closure and a tuning parameter $t_f\in[0,1]\subseteq\mathbb{R}$, the associated embedding is defined as follows:
$$\phi_{\mathcal{P}}(P)_i=\begin{cases}
	\frac{\epsilon(P)_{\color{green}\alpha\beta}}{\|\epsilon\|_2}\omega t_f^{|R>0|} & i={\color{green}\alpha\beta}\\
	\frac{\nu(P)_{\color{green}\alpha}}{\|\nu\|_2}t_f^{|R>0|} & i={\color{green}\alpha}\\
\end{cases}$$
where $\epsilon$ and $\nu$ respectively represent the non-negatively defined embedding associated to the TG's transition matrix and nodes.
\end{definition}

Albeit we might pick different characterizations for $\epsilon$ and $\nu$, {in our experiment section} we choose two possible interchangeable definitions ($x=1$ and $x=2$) for both sub-embeddings (Table \ref{tab:embedstrat}): $l$ is the path length indicated in Table \ref{tab:proj}, and $N$ for $\nu^1$ is a normalization factor such that $\sum_{{\color{green}\alpha}\in\Sigma_\tau}\nu^1(P)_{\color{green}\alpha}=1$. In Example \ref{ex:cmpexample}, we will see that the given definition of $\epsilon^1$ provides a better approximation than the edge embedding $\epsilon^2$.


%When the transition matrix is ergodic \cite{StocasticCC},  the transition matrix embedding converges to $\epsilon(R)_{\color{green}\alpha\beta}=[(\mathbf{I}-\lambda\Lambda)^{-1}]_{\color{green}\alpha\beta}$ \cite{GartnerFW03} for $n\to+\infty$.

\begin{table}[!t]
	\centering
	\caption{Embedding for both the traces $\sigma$ of maximum length $4$ from $P$ and  $\sigma^*$.}\label{tab:emb2}\label{tab:embsitar}
	\begin{tabular}{l|l|l|l|l|l|l|}
		\toprule
		& a    & b                                                   & c    & aa   & ca   & cb   \\
		\midrule
		
		
		$\phi_{\mathcal{P}}(P_\textup{aaaa})$ & $1.00\cdot 10^{-24}$ & $0.00\cdot 10^{0}$ & $0.00\cdot 10^{0}$& $6.44\cdot 10^{-26}$& $0.00\cdot 10^{0}$& $0.00\cdot 10^{0}$\\
		$\phi_{\mathcal{P}}(P_\textup{aaa})$ & $1.00\cdot 10^{-24}$ & $0.00\cdot 10^{0}$ & $0.00\cdot 10^{0}$& $1.29\cdot 10^{-25}$& $0.00\cdot 10^{0}$& $0.00\cdot 10^{0}$\\
		$\phi_{\mathcal{P}}(P_\textup{aa})$ & $1.00\cdot 10^{-24}$ & $0.00\cdot 10^{0}$ & $0.00\cdot 10^{0}$& $2.57\cdot 10^{-25}$& $0.00\cdot 10^{0}$& $0.00\cdot 10^{0}$\\
		$\phi_{\mathcal{P}}(P_\textup{a})$ & $1.00\cdot 10^{-4}$ & $0.00\cdot 10^{0}$ & $0.00\cdot 10^{0}$& $0.00\cdot 10^{0}$& $0.00\cdot 10^{0}$& $0.00\cdot 10^{0}$\\
		$\phi_{\mathcal{P}}(P_\textup{caa})$ & $7.07\cdot 10^{-25}$ & $0.00\cdot 10^{0}$ & $7.07\cdot 10^{-25}$& $1.46\cdot 10^{-25}$& $2.05\cdot 10^{-25}$& $0.00\cdot 10^{0}$\\
		$\phi_{\mathcal{P}}(P_\textup{ca})$ & $7.07\cdot 10^{-25}$ & $0.00\cdot 10^{0}$ & $7.07\cdot 10^{-25}$& $0.00\cdot 10^{0}$& $1.00\cdot 10^{-8}$& $0.00\cdot 10^{0}$\\
		$\phi_{\mathcal{P}}(P_\textup{cb})$ &  $0.00\cdot 10^{0}$ & $7.07\cdot 10^{-25}$ & $7.07\cdot 10^{-25}$& $0.00\cdot 10^{0}$&  $0.00\cdot 10^{0}$ & $4.29\cdot 10^{-9}$\\
		$\phi_{\mathcal{P}}(P_\textup{caaa})$  & $7.07\cdot 10^{-25}$ &  $0.00\cdot 10^{0}$ & $7.07\cdot 10^{-25}$& $1.03\cdot 10^{-25}$&  $7.20\cdot 10^{-26}$ & $0.00\cdot 10^{0}$\\
		\bottomrule 
				$\phi_{\mathcal{P}}(P_{\sigma^*})$ & $8.16\cdot10^{-17}$ & $4.08\cdot 10^{-17}$ & $4.08\cdot10^{-17}$ & $4.37\cdot 10^{-17}$ & $1.03\cdot10^{-16}$ & $4.37\cdot10^{-17}$\\
		\bottomrule
	\end{tabular}
\end{table}  \begin{table}[!t]
\caption{Different sub-embedding definitions ($\epsilon^1$, $\epsilon^2$, $\nu^1$, and $\nu^2$) for $\phi_{\mathcal{P}}$.}\label{tab:embedstrat}
\centering
\begin{tabular}{c|c|c}
	\toprule
	& $x=1$ & $x=2$ \\
	\midrule
	$\epsilon^x(P)_{\color{green}\alpha\beta}:=$ & $\label{eq:epsilon}
	\sum_{i=1}^l{\lambda^i}\frac{[LR^iL^t]_{\color{green}\alpha\beta}}{\sum_{\color{green}\alpha'\beta'}R^i_{\color{green}\alpha'\beta'}}$ & $
	\sum_{i=1}^l\lambda^i[\Lambda^i]_{\color{green}\alpha\beta}$\\
	$\nu^x(P)_{\color{green}\alpha}:=$ & $\frac{1}{N}\sum_{\braket{\sigma,w}\in\mathcal{W}_p^l(P)}\frac{|\Set{\sigma_i\in\sigma|\sigma_i\neq\tau\wedge \sigma_i={\color{green}\alpha}}|}{|\sigma|}$ & $0$ \\
	\bottomrule
\end{tabular}
\end{table} 
\begin{example}\label{ex:withpaths} After generating all the TGs for the unravelled traces (Example \ref{ex:neue}), we further associate a sub-embedding $\epsilon$ using the $2$-grams included in the model traces.\footnoteref{fn:caveat}
Given $\Sigma_\tau=\Set{a,b,c}$, the embedding space is of size $6$: three features (computed using $\epsilon$) correspond to the labels of the non-$\tau$ vertices in $\Sigma_\tau$, i.e., $\{a,b,c\}$, and other three features (computed using $\nu$) correspond to the $2$-grams subsequences that are also traces in $W^4_0(P)$, i.e., $\{aa,ca,cb\}$. Therefore, $\{a,b,c,aa,ca,cb\}\subset \Sigma_\tau^2\cup\Sigma_\tau$ is the whole set of features describing both the nodes and the transition matrix, so $\phi_{\mathcal{P}}$ is a vector with 6 dimensions.
$t_f\in [0,1]\subset\mathbb{R}^+_{\geq 0}$ and $\lambda\in [0,1]\subset\mathbb{R}^+_{\geq 0}$ are tuning parameters that can be inferred from the available data \cite{DriessensRG06}. While the latter describes the previously mentioned decay factor, $t_f$ represents the relevance of our embedding representation when the number of the edges within TG increases. In both our experiments and examples, we chose $t_f=0.0001$ and $\lambda=0.07$.
%\xout{The embedding associated to $P$ is described in Table \ref{tab:emb1} as $\phi_{\mathcal{P}}(P)$: it shows that doing ${\color{green}a}\rightsquigarrow{\color{green}a}$ is more probable than doing  ${\color{green}c}\rightsquigarrow{\color{green}a}$. Also, given that both the probability of performing ${\color{green}c}\overset{1}{\rightsquigarrow}{\color{green}b}$ is relatively low and trace $\color{green}cb$ is relatively infrequent, ${\color{green}c}{\rightsquigarrow}{\color{green}b}$ is less probable than any other subtrace. If we now consider the single nodes, $\color{green}c$ shares a subset of traces with $\color{green}a$ where $\color{green}a$ is more frequent than $\color{green}c$, and therefore the score of the former is higher than the one of the latter. Also, the score associated to the single node $\color{green}b$ is lower than the one for the single node $\color{green}c$ because $\color{green}b$ is less frequent and appears in less probable traces than $\color{green}c$: in particular, $\color{green}c$ appears in \textit{ca}, which is more probable than \textit{cb}.}
Table \ref{tab:emb2} represents the embeddings $\phi_{\mathcal{P}}(P_\sigma)$ generated from the sub-TGs generated in Example \ref{ex:neue}, where the $l=|\sigma|$ for each trace $\sigma$ associated to the sub-TG. This representation is completely independent from the representation associated with a trace to be aligned. Therefore it doesn't have to be recomputed at each alignment with a different $\sigma^*$.
{After representing trace $\sigma^*=\textup{caba}$ to be aligned as a graph (see Example \ref{ex:tracembed}), we can represent its}
%\xout{Similar considerations can be also drawn from the}
embedding $\phi_{\mathcal{P}}(P_{\sigma^*})$ with strategies $\epsilon^1$ and $\nu^1$ %\xout{associated to the trace $\sigma^*=\textup{caba}$ (also in Table --):} \ADD
{as} in Table \ref{tab:embsitar}: ${\color{green}a}$ is clearly the most frequent label ${\color{green}b}$ and ${\color{green}c}$ are equiprobable, as well as the path ${\color{green}c}\rightsquigarrow {\color{green}a}$ appears twice in the trace set and then, it is more frequent than the other subtraces. 
\end{example}

%\xout{After defining the embedding, we can show that this embedding establishes some desired features that are independent of the definition of $\epsilon$ and $\nu$, and that $\epsilon$ and $\nu$ only depend on the characterization of both the labelling $L$ and the transition matrix $R$. We provide a rewriting proposition that is going to be used in the incoming subsection to provide the aforementioned characterizing properties.} \ADD

Given that kernel functions $k_\phi$ are defined as the dot product between the embedding $\phi$ of distinct objects $x$ (\S\ref{subsec:katk}), then we can express the kernel $k_{\phi_{\mathcal{P}}}$ as such dot product and, after normalizing $\epsilon$ and $\nu$, we can rewrite such kernel
{as a function of distance  $\|\hat{\epsilon}-\hat{\epsilon}'\|_2^2$ and $\|\hat{\nu}-\hat{\nu}'\|_2^2$ for both traces $\sigma$ and $\sigma'$ to be aligned:}

\begin{proposition}\label{lem:rewritinglemma}
Given two TGs $P=(s,t,L,R,w)$ and $P'=(s',t',L',R',w')$, the TG Kernel is defined as follows:
$$\begin{aligned}
k_{\phi_{\mathcal{P}}}(P,P')=&\omega\omega't_f^{|R>0|+|R'>0|}\left(1-\frac{\norm{\hat{\epsilon}-\hat{\epsilon}'}{2}^2}{2}\right)+\\
	&+t_f^{|R>0|+|R'>0|}\left(1-\frac{\norm{\hat{\nu}-\hat{\nu}'}{2}^2}{2}\right)\\
\end{aligned}$$
\end{proposition}
\begin{proof} We can close the goal by definition of $k_{\phi}$ as a vector dot product for any embedding $\phi$ and by  $\|\hat{\mathbf{x}}-\hat{\mathbf{x}}'\|_2^2=(2-1\braket{\hat{\mathbf{x}},\hat{\mathbf{x}}'})$ (\S\ref{subsec:katk}).	
%	
%	 for normalized vectors (\S\ref{subsec:katk}), we can expand the former definition as follows:
%$$\begin{aligned}
%{k_{\phi_{\mathcal{P}}}(P,P')}&{=\Braket{\phi_{\mathcal{P}}(P),\phi_{\mathcal{P}}(P')}}\\
%	&{=\sum_{\alpha\beta\in \Sigma_\tau^2}\frac{\epsilon_{\color{green}\alpha\beta}}{\|\epsilon\|_2}\frac{{\epsilon'}_{\color{green}\alpha\beta}}{\|\epsilon'\|_2}\omega\omega't_f^{|R>0|+|R'>0|}\quad+\quad \sum_{\alpha\in \Sigma_\tau}\frac{\nu_{\color{green}\alpha}}{\|\nu\|_2}\frac{{\nu'}_{\color{green}\alpha}}{\|\nu'\|_2}t_f^{|R>0|+|R'>0|}}\\
%	&{=ww'\sigma^{|Rb>0|+|R'>0|}\sum_{\alpha\beta\in \Sigma_\tau^2}\frac{\epsilon_{\color{green}\alpha\beta}}{\|\epsilon\|_2}\frac{{\epsilon'}_{\color{green}\alpha\beta}}{\|\epsilon'\|_2}\quad+\quad t_f^{|R>0|+|R'>0|}\sum_{\alpha\in \Sigma_\tau}\frac{\nu_{\color{green}\alpha}}{\|\nu\|_2}\frac{{\nu'}_{\color{green}\alpha}}{\|\nu'\|_2}}\\
%	&{=\omega\omega't_f^{|R>0|+|R'>0|}\Braket{\hat{\epsilon}, \hat{\epsilon}'}+ t_f^{|R>0|+|R'>0|}\Braket{\hat{\nu}, \hat{\nu}'}}\\
%	&{=\omega\omega't_f^{|R>0|+|R'>0|}\left(1-\frac{\norm{\hat{\epsilon}- \hat{\epsilon}'}{2}^2}{2}\right)+ t_f^{|R>0|+|R'>0|}\left(1-\frac{\norm{\hat{\nu}- \hat{\nu}'}{2}^2}{2}\right)}\\
%\end{aligned}$$
\end{proof}

%\xout{Given that we can now follow Definition \ref{def:ppne} for representing a trace $\sigma$ as a proper embedding after transforming it as a TG $P_{\sigma^*}$ (\S\ref{subsec:katk}), we can find the TG $P$ providing the best approximate match with  a trace $\sigma$ as follows:}
%\[\Rcancel{\underset{{P}}{\max\arg}\;k_{\phi_{\mathcal{P}}}(P,T)}\]
%\xout{Still, this TG matching strategy does not allow to find the trace maximizing such score.} %To assess such problem, the next section is going to determine both an exact (\S\ref{subsec:exbkptap}) and an approximated strategy (\S\ref{subsec:akptap}) for probabilistically matching one single trace from the TG.
%
%\xout{Given the characterization of a TG as in \S\ref{subsec:ppn} and the embedding strategy proposed in Definition \ref{def:ppne}, We can \ADD{now} generate an embedding for each possible weighted trace $\braket{\sigma,w_\sigma}\in\mathcal{W}_p^n(P)$ for a given TG $P$ as described in the following definition:}





\begin{table}[!t]
	\caption{Comparison between the ranking induced by the expected ranking $k_\star$ and the proposed kernel $k_{\phi_{\mathcal{P}}}$ with embedding strategies $\epsilon^1$ and $\nu^1$: arrows $\boldsymbol{\downarrow}$ remark the column of choice under which we sort the rows (i.e., ranking).}\label{tab:rank3}
	\centering
	%	\begin{tabular}{l|c|ll}
	%		\toprule
	%		$\sigma$ & $k_{\phi_{\mathcal{P}}}(\sigma,\sigma^*)$ & \textit{kernel ranking} & expected ranking\\
	%		\midrule
	%		a & $8.16\cdot 10^{-21}$ & \textbf{1} & \textbf{\color{blue}1}\\
	%		ca & $1.89\cdot 10^{-24}$ & \textbf{2} & \textbf{\color{blue}4}\\
	%		cb & $7.64\cdot 10^{-25}$ & \textbf{3} & \textbf{\color{blue}5}\\
	%		caa & $1.14\cdot 10^{-40}$ & \textbf{4} & \textbf{\color{blue}7}\\
	%		caaa & $9.84\cdot 10^{-41}$ & \textbf{5} & \textbf{\color{blue}8}\\
	%		aa & $9.28\cdot 10^{-41}$ & \textbf{6} & \textbf{\color{red}2}\\
	%		aaa & $8.72\cdot 10^{-41}$ & \textbf{7} & \textbf{\color{red}3}\\
	%		aaaa & $8.44\cdot 10^{-41}$ & \textbf{8} & \textbf{\color{red}6}\\
	%		
	%		\bottomrule
	%	\end{tabular}
	
	\resizebox{.9\textwidth}{!}{\begin{tabular}{l|ll|cc}
		\toprule
		
		{$\sigma$} &
		%\multirow{2}{*}{$d(\sigma,\sigma^*)$} &
		%\multicolumn{2}{c|}{$\mu_{\sigma^*}$} &
		$( w_\sigma$ &  $,\,\boldsymbol{\downarrow} s_d(\sigma,\sigma^*)) $ &
		{$=k_\star(\sigma,\sigma^*)$} &
		{$k_{\phi_{\mathcal{P}}}(P_\sigma,P_{\sigma^*})$} \\
		
		
		\midrule
		{caa}  & $0.035$ & $\;\; 0.8333$ & $0.0292$ & $1.14\cdot 10^{-40}$\\
		{caaa}  &  $0.0175$ & $\;\; 0.8333$ & $0.0145$ & $9.84\cdot 10^{-41}$\\
		{a}  & $0.4$ & $\;\; 0.6250$  & $0.2500$ & $8.16\cdot 10^{-21}$ \\
		{aaaa}  & $0.05$ & $\;\; 0.6250$ & $0.0357$ & $8.44\cdot 10^{-41}$\\
		{aa}  & $0.2$ & $\;\; 0.7142$ & $0.1428$ & $9.28\cdot 10^{-41}$ \\
		{aaa}  & $0.1$ & $\;\; 0.7142$ & $0.0714$ & $8.72\cdot 10^{-41}$\\
		{ca}  &  $0.07$ & $\;\; 0.7142$ & $0.0500$ & $1.89\cdot 10^{-24}$\\
		{cb}  &  $0.06$ & $\;\; 0.7142$ & $0.0428$ & $7.64\cdot 10^{-25}$\\
		\bottomrule
	\end{tabular}\quad \begin{tabular}{l|c}
	\toprule
	
	{$\sigma$} &
	{$\boldsymbol{\downarrow}k_\star(\sigma,\sigma^*)$} \\
	
	
	\midrule
	{a}  &  $0.2500$ \\
	{aa}  &  $0.1428$  \\
	{aaa}  & $0.0714$ \\
	{ca}  &   $0.0500$\\
	{cb}  & $0.0428$ \\
	{aaaa}  &  $0.0357$ \\
	{caa}  &  $0.0292$ \\
	{caaa}  &   $0.0145$ \\
	\bottomrule
\end{tabular}\quad	\begin{tabular}{l|c}
	\toprule
	
	{$\sigma$} &
	{$\boldsymbol{\downarrow}k_{\phi_{\mathcal{P}}}(P_\sigma,P_{\sigma^*})$} \\

	
	\midrule
	{a}  & $8.16\cdot 10^{-21}$ \\
	{ca}  &   $1.89\cdot 10^{-24}$\\
	{cb}  &   $7.64\cdot 10^{-25}$\\
	{caa}  &$1.14\cdot 10^{-40}$\\
	{caaa}  &  $9.84\cdot 10^{-41}$\\
	{aa}  &  $9.28\cdot 10^{-41}$ \\
	{aaaa}  & $8.44\cdot 10^{-41}$\\
	{aaa}  &  $8.72\cdot 10^{-41}$\\
	\bottomrule
\end{tabular}}
\end{table}


At this stage, the computation of $\underset{\braket{\sigma,w_\sigma}\in \mathcal{W}_p^n(P), P_\sigma\in\mathbf{P}_p^n(P)}{\max\arg} k_{\phi_\mathcal{P}}(P_\sigma, P_{\sigma^*})$ returns the best approximated trace alignment $\sigma$ for a query trace represented as $P_{\sigma^*}$. %\xout{Similarly, we can provide the TG $P\in\mathbf{P}$ providing the best approximated alignment for $P_{\sigma^*}$ as $\underset{P}{\max\arg}\underset{ P_\sigma\in\mathbf{P}_p^n(P)}{\max} k_{\phi_\mathcal{P}}(P_\sigma, P_{\sigma^*})$.}¯

\begin{example}\label{ex:11}
	Given that $k_{\phi_{\mathcal{P}}}(\sigma,\sigma^*)=\braket{\phi_{\mathcal{P}}(P_\sigma),\;\phi_{\mathcal{P}}(P_{\sigma^*})}$ for each weighted trace $\braket{\sigma,w_\sigma}\in\mathcal{W}_0^4(P)$, then the dot product between the resulting similarity ranking is represented in Table \ref{tab:rank3}, where the expected ranking $k_\star$ is also showed. This similarity score approximates the expected ranking  and tends to rank in similar ways the paths generated from the same subgraph of the TG.
\end{example}


%\begin{example}\label{ex:moreskew}
%	\xout{Let us suppose to change the probability distribution associated with the $P$'s edges, so that it becomes more skewed and that some traces are relatively more probable than others. Let us set $p_1=p_2=0.5$, $p_3=0.9$, $p_6=0.1$, $p_4=0.3$, and $p_5=0.7$, so that the initial choice is equiprobable but performing a loop is more probable than terminating the path. We keep the other tuning parameters as in Example \ref{ex:withpaths}. In this case, we generate the following set of weighted traces:}
%	$$\begin{aligned}
%	\Rcancel{\mathcal{W}_0^4(P)=\{}&\Rcancel{\braket{cb,0.35},\braket{a,0.05},\braket{aa,0.045},\braket{aaa,0.0405},}\\
%	&\Rcancel{\braket{aaaa,0.03645},\braket{ca,0.015},\braket{caa,0.0135},\braket{caaa,0.01215}\}}\\
%	\end{aligned}$$
%	\xout{Let us also assume that we want to align these traces in a probabilistic way with the query $\sigma^*=\textup{caba}$: the distance ($d$) and similarity ($s_d$) scores will be still the same, while the associated probabilities will vary. The expected ranking by multiplying weight with similarity is represented in Table \ref{tab:witherror}.}
%	
%	%\xout{As a consequence of the different probability distribution associated to the edges, a different set of embedding will be generated for each trace of interest while the TG $T$ associated to $\sigma^*$ will be kept the same. Table \ref{tab:witherror} represents the ranking induced by the kernel $k_{\phi_{\mathcal{P}}}$ over this different set of vectors by ranking the traces in descendant order of $k_{\phi_{\mathcal{P}}}$. As we might notice, the more skewed edge probability distribution introduced more errors in the ranking result: while the largest ranking subsequence (marked in blue) always starts from the best-expected trace \textit{cb}, this element now appears in the third position, and the position of traces \textit{caa} and \textit{aaaa} is swapped.}
%	
%\end{example}
%\begin{table}[!t]
%	\centering
%	\caption{Expected ranking of the paths from Example \ref{ex:moreskew} with the trace $\sigma^*=\textup{caba}$. The cost function is the one from \cite{LeoniM17} and its normalized similarity score has $c=5$. Traces are ranked by decreasing kernel $k_{\phi_{\mathcal{P}}}$ value: slight changes in the expected expected order are circled, the others are marked in red.}\label{tab:witherror}
%	\begin{tabular}{lc|ll|cc|l}
%		\toprule
%		
%		\multirow{2}{*}{$\sigma$} &
%		\multirow{2}{*}{$d(\sigma,\sigma^*)$} &
%		\multicolumn{2}{c|}{$\mu_{\sigma^*}$} &
%		\multirow{2}{*}{$\approx s_d(\sigma,\sigma^*)\cdot w_\sigma$} &
%		\multirow{2}{*}{$k_{\phi_{\mathcal{P}}}(\sigma,\sigma^*)$}&
%		\multirow{2}{*}{\textit{expected ranking}}\\
%		
%		\cline{3-4} &&  $\langle w_\sigma$ &  $,\,s_d(\sigma,\sigma^*)\rangle $ && \\
%		
%		\midrule
%		{a}  & $3$ & $0.05$ & $\;\; 0.6250$  & $0.03125$ & $8.16497\cdot 10^{-16}$ & \textbf{\color{red}3}\\
%		{ca}  & $2$ & $0.015$ & $\;\; 0.7142$ & $0.01071$ & $1.30623\cdot 10^{-18}$ & \textbf{\color{red}7}\\
%		{cb}  & $2$ & $0.35$ & $\;\; 0.7142$ & $0.25000$ & $1.01399\cdot10^{-18}$ & \textbf{\color{blue}1}\\
%		{aa}  & $2$ & $0.045$ & $\;\; 0.7142$ & $0.03214$ & $1.01894\cdot10^{-30}$ & \textbf{\color{blue}2}\\
%		{aaa}  & $2$ & $0.0405$ & $\;\; 0.7142$ & $0.02893$ & $9.98696\cdot10^{-31}$ & \textbf{\color{blue}4}\\
%		{caa}  & $1$ & $0.0135$ & $\;\; 0.8333$ & $0.01125$ & $9.96052\cdot10^{-31}$ & \textbf{\color{blue}\ding{177}}\\
%		{aaaa}  & $3$ & $0.03645$ & $\;\; 0.7142$ & $0.02603$ & $9.80476\cdot10^{-31}$ & \textbf{\color{blue}\ding{176}}\\
%		{caaa}  & $1$  & $0.01215$ & $\;\; 0.8333$ & $0.01012$ & $9.52398\cdot 10^{-31}$ & \textbf{\color{blue}8}\\
%		\bottomrule
%	\end{tabular}
%\end{table}
\begin{table}[!t]
	\caption{Comparison between the ranking induced by the expected ranking $k_\star$ and the proposed kernel $k_{\phi_{\mathcal{P}}}$ with embedding strategies $\epsilon^2$ and $\nu^1$: arrows $\boldsymbol{\downarrow}$ remark the column of choice under which we sort the rows (i.e., ranking).}\label{tab:compLit}
	\centering
	\resizebox{.9\textwidth}{!}{\begin{tabular}{l|ll|cc}
			\toprule
			
			{$\sigma$} &
			%\multirow{2}{*}{$d(\sigma,\sigma^*)$} &
			%\multicolumn{2}{c|}{$\mu_{\sigma^*}$} &
			$( w_\sigma$ &  $,\,\boldsymbol{\downarrow} s_d(\sigma,\sigma^*)) $ &
			{$=k_\star(\sigma,\sigma^*)$} &
			{$k_{\phi_{\mathcal{P}}}(P_\sigma,P_{\sigma^*})$} \\
			
			
			\midrule
			{caa}  & $0.035$ & $\;\; 0.8333$ & $0.0292$ & $1.03498\cdot10^{-40}$\\
			{caaa}  &  $0.0175$ & $\;\; 0.8333$ & $0.0145$ & $8.94997\cdot10^{-41}$ \\
			{a}  & $0.4$ & $\;\; 0.6250$  & $0.2500$ & $8.16497\cdot10^{-21}$\\
			{aaaa}  & $0.05$ & $\;\; 0.6250$ & $0.0357$ & $8.20640\cdot10^{-41}$\\
			{aa}  & $0.2$ & $\;\; 0.7142$ & $0.1428$ & $9.96007\cdot10^{-41}$ \\
			{aaa}  & $0.1$ & $\;\; 0.7142$ & $0.0714$ & $8.41263\cdot10^{-41}$\\
			{ca}  &  $0.07$ & $\;\; 0.7142$ & $0.0500$ & $1.45079\cdot10^{-24}$\\
			{cb}  &  $0.06$ & $\;\; 0.7142$ & $0.0428$ & $8.52070\cdot10^{-25}$\\
			\bottomrule
		\end{tabular}\quad \begin{tabular}{l|c}
			\toprule
			
			{$\sigma$} &
			{$\boldsymbol{\downarrow}k_\star(\sigma,\sigma^*)$} \\
			
			
			\midrule
			{a}  &  $0.2500$ \\
			{aa}  &  $0.1428$  \\
			{aaa}  & $0.0714$ \\
			{ca}  &   $0.0500$\\
			{cb}  & $0.0428$ \\
			{aaaa}  &  $0.0357$ \\
			{caa}  &  $0.0292$ \\
			{caaa}  &   $0.0145$ \\
			\bottomrule
		\end{tabular}\quad	\begin{tabular}{l|c}
			\toprule
			
			{$\sigma$} &
			{$\boldsymbol{\downarrow}k_{\phi_{\mathcal{P}}}(P_\sigma,P_{\sigma^*})$} \\
			
			
			\midrule
			{a}  & $8.16497\cdot10^{-21}$ \\
			{ca}  &   $1.45079\cdot10^{-24}$\\
			{cb}  &   $8.52070\cdot10^{-25}$\\
			{caa}  & $1.03498\cdot10^{-40}$\\
			{aa}  &  $9.96007\cdot10^{-41}$ \\
			{caaa}  &  $8.94997\cdot10^{-41}$ \\
			{aaa}  &  $8.41263\cdot10^{-41}$\\
			{aaaa}  & $8.20640\cdot10^{-41}$\\
			\bottomrule
	\end{tabular}}
%	
%	\centering
%	\begin{tabular}{lc|l}
%		\toprule
%		%\multicolumn{3}{c||}{Example \ref{ex:withpaths}} %&
%		%\multicolumn{3}{c}{Example \ref{ex:moreskew}}\\
%		%\hline
%		$\sigma$ &  $k_{\phi_{\mathcal{P}}}(P_\sigma,P_{\sigma^*})$ & $k_\star(\sigma,\sigma^*)$\\ %&
%		%$\sigma$ &  $k_{\phi_{\mathcal{P}}}(\sigma,\sigma^*)$ & \textit{exp. ranking}\\
%		\midrule
%		
%		a & $\;8.16497\cdot10^{-21}$ & \textbf{\color{blue}1} \\%& a & $\;8.16497\cdot 10^{-21}$ & \textbf{\color{red}3} \\
%		ca & $\;1.45079\cdot10^{-24}$ & \textbf{\color{blue}4} \\%& ca &  $\;1.45079\cdot 10^{-24}$ & \textbf{\color{red}7}\\
%		cb & $\;8.52070\cdot10^{-25}$ & \textbf{\color{blue}5} \\%& cb & $\;8.52070\cdot10^{-25}$& \textbf{\color{blue}1}\\
%		caa & $\;1.03498\cdot10^{-40}$ & \textbf{\color{blue}7} \\%& aa & $\;9.29342\cdot10^{-41}$ & \textbf{\color{blue}2}\\
%		aa & $\;9.96007\cdot10^{-41}$ & \textbf{\color{red}2} \\%& caa & $\;9.18112\cdot10^{-41}$ & \textbf{\color{blue}6}\\
%		caaa & $\;8.94997\cdot10^{-41}$ & \textbf{\color{red}8} \\%& caaa & $\;8.71867\cdot10^{-41}$ & \textbf{\color{blue}8}\\
%		aaa & $\;8.41263\cdot10^{-41}$ &  \textbf{\color{red}3} \\%& aaa & $\;8.31269\cdot10^{-41}$ & \textbf{\color{red}4}\\
%		aaaa & $\;8.20640\cdot10^{-41}$ &  \textbf{\color{red}6}\\% & aaaa & $\;8.19352\cdot10^{-41}$ & \textbf{\color{red}5}\\
%		
%		\bottomrule
%	\end{tabular}
\end{table}

\begin{example}\label{ex:cmpexample}
	Let us compare the ranking results by replacing in $\phi_{\mathcal{P}}$ the edge embedding $\epsilon^1$ with $\epsilon^2$. If we re-run the computations performed in  Example \ref{ex:11}, we obtain   Table \ref{tab:compLit}:  $\epsilon^1$ provides longer approximated subsequences if compared to $\epsilon^2$. Both embedding proposals tend to favor sequences containing one single node or one single subtrace due to the normalization of both  the edge and the nodes' distribution, but $\epsilon^2$ seems to be less influenced than $\epsilon^1$ in the change of the edge distribution. Therefore, $\epsilon^1$ proposal is to be preferred to $\epsilon^2$.
\end{example}


\subsubsection{Properties}
As a first property, we want to show that when the two traces are equivalent,  there exists an embedding configuration for which the kernel computation reduces to the two traces' weight product, i.e. $ww'$. The kernel will   represent the probability that both traces are valid contemporaneously and, when both weights are $1$, the kernel returns $1$. We will call this condition  ``weak equality'' because we cannot possibly prove that $k_{\phi_{\mathcal{P}}}(P,P')=ww'\Leftrightarrow \mathcal{W}_p^n(P)=\mathcal{W}_p^n(P')$, as there could be similar embeddings coming from Workflow Nets sharing a different weighted traces set ($\mathcal{W}_p^n(P)\neq\mathcal{W}_p^n(P')$).

\begin{figure}[!t]
	\centering
	\includegraphics[scale=1.1]{images/counterexample.pdf}
	\caption{Two TGs, $Q$ (left) and $R$ (right), having a different set of traces but the same embedding representation.}\label{fig:counterexample}
\end{figure}
\begin{example}
	If we use $\epsilon$ (or $\epsilon^2$) and $\nu$ (or $\nu^2$) for $\phi_{\mathcal{P}}$, we might have a false positive for ``weak equality'' if $Q=(s,s,L,R,w)$ and $R=(s',s',L,R,w)$ are both cycle graphs with $s\neq s'$, $\textit{label}(s)\neq\textit{label}(s')$, $\textit{label}(s)\neq\tau$, and $\textit{label}(s')\neq\tau$. An intuitive example of such situation is presented in Figure \ref{fig:counterexample}: both graphs will have the same frequency for both subtraces and nodes, and therefore have the same  $\epsilon$ and $\nu$ by construction. By having different initial and accepting node with  different labels, we have $\mathcal{W}_0^{\aleph_0}(Q)=\Set{\textup{a(bca)}^n|n\in\mathbb{N}}$ and $\mathcal{W}_0^{\aleph_0}(R)=\Set{\textup{c(abc)}^n|n\in\mathbb{N}}$, thus implying $\mathcal{W}_0^{\aleph_0}(Q)\neq\mathcal{W}_0^{\aleph_0}(R)$ but $k_{\phi_{\mathcal{P}}}(Q,R)=1$ for $t_f=1$.
\end{example}


\begin{lemma}[Weak Equality]
	Given two TGs $P=(s,t,L,R,w)$ and $P'=(s',t',L',R',w')$ providing the same set of weighted traces, then $k_{\phi_{\mathcal{P}}}(P,P')=ww'$ for $t_f=1$.
\end{lemma}
%\begin{proof}
%	\xout{Given Proposition \ref{lem:rewritinglemma} and the positive definition of $\epsilon$ and $\nu$,  we have that $\norm{\hat{\epsilon}-\hat{\epsilon}'}{2}\to 0$ as well as $\norm{\hat{\nu}-\hat{\nu}'}{2}\to 0$, for which we can immediately close the goal.}
%\end{proof}

%\xout{As per previous observations, we know that}
Two TGs should have the maximum dissimilarity when all the non $\tau$-nodes have different labels, thus making it impossible to find an alignment, thus implying that they share an utterly dissimilar set of weighted traces:

\begin{lemma}[Strong Dissimilarity]
	Given two TGs $P=(s,t,L,R,w)$ and $P'=(s',t',L',R',w')$, $k_{\phi_{\mathcal{P}}}(P,P')=0$ iff. $P$ and $P'$ have a different set of labels with $t_f,w,w'>0$.
\end{lemma}
%\begin{proof}
%	\xout{If we exclude the trivial conditions $t_f=0$, $w=0$ or $w'=0$, the only condition when the kernel returns zero is when  $\Braket{\hat{\epsilon},\hat{\epsilon}'}=0$ and $\Braket{\hat{\nu},\hat{\nu}'}=0$. This implies that, when a component of $\epsilon$ (or $\nu$) is non-zero, the same component of $\epsilon'$ (or $\nu'$) is zero and viceversa. This directly requires that there is a different set of labels associated to the nodes. }
%\end{proof}

As a corollary of the two lemmas, we have that the proposed embedding performs weakly-ideally as defined in \S\ref{subsec:katk}, as equality condition holds in a relaxed form.

Last, under the assumption that a TG is approximately characterized by $\epsilon$ and $\nu$, we might expect that the TG similarity is characterized by the sum of the distance of both embeddings. Therefore, we show that an increase in both distance embeddings approximately corresponds to a decrease in the kernel output and vice-versa.

\begin{lemma}\label{lem:approxRank}
	Given two TGs $P=(s,t,L,R,w)$ and $P'=(s',t',L',R',w')$ having respectively the embeddings $(\epsilon,\nu)$ and $(\epsilon',\nu')$, we have that the kernel $k_{\phi_{\mathcal{T}}}$ induces an inverse ranking of $\norm{\hat{\epsilon}-\hat{\epsilon}'}{2}+\norm{\hat{\nu}-\hat{\nu}'}{2}$:
	$$k_{\phi_{\mathcal{T}}}(P,P')\appropto 2-(\norm{\hat{\epsilon}-\hat{\epsilon}'}{2}+\norm{\hat{\nu}-\hat{\nu}'}{2})$$
\end{lemma}
%\begin{proof}
%	\xout{Let us use $T=t_f^{|R>0|+|R'>0|}$, $\omega=ww'$, $V=\norm{\hat{\nu}-\hat{\nu}'}{2}$, and $E=\norm{\hat{\epsilon}-\hat{\epsilon}'}{2}$ as shorthands. The goal can be rewritten as $k_{\phi_{\mathcal{T}}}(P,P')\appropto 2-(E+V)$. Given that the embeddings $(\epsilon,\nu)$ and $(\epsilon',\nu')$ are normalized kernel function $k_{\phi_{\mathcal{P}}}$ and that they are always positive definite, then we have that $0\leq E +V\leq 2$, so $0\leq 2-(E+V)\leq 2$. Using Proposition \ref{lem:rewritinglemma}, we can write $k_{\phi_{\mathcal{P}}}(P,P')$ as follows:}
%	$$\Rcancel{\left(1-\frac{E}{2}\right)\omega T+\left(1-\frac{V}{2}\right)T=T\left(\omega+1-\frac{E\omega+V}{2}\right)}$$
%	\xout{Given that the embeddings $(\epsilon,\nu)$ and $(\epsilon',\nu')$ are normalized in $k$ and that they are always positive definite,  we also have that $0\leq E\omega +V\leq 2$ where $0\leq \omega\leq 1$. We can also write  $0\leq \omega+1-\frac{E\omega+V}{2}\leq \frac{2}{T}$. For $\omega,T=1$, we have that $k_{\phi_{\mathcal{P}}}(P,P')=2-\frac{(E+V)}{2}$. Thus, $0<\omega,T<1$ approximates the expected ranking. }
%\end{proof}

%\ADD{Such lemma is going to be empirically evaluated in our experiment section.}

\begin{table}[!t]
\caption{Distinct USWNs and associated sets of unravelled traces from a single Sepsis Cases Event Log \cite{mannhardt_2016}.}\label{tab:dataset}
 \begin{adjustbox}{max width=\textwidth}
	\begin{tabular}{crl||cl|c}
		\toprule
		\textbf{Experiment Conf.} $(\mathcal{U})$ & \textit{Model} & $+$\textit{W. Estimator} & $n$ & $p_\theta$& $\;\;|\mathcal{W}_{p_\theta}^n(P_{\mathcal{U}})|$ \\
		\midrule
		
		\textbf{SM\_CONS\_20} &SplitMiner 2.0  \cite{AugustoCDRP19}       & +\texttt{Constant} & $20$ & $\;\;0$ & $157$  \\
		
		\textbf{SM\_FORK\_20} & SplitMiner 2.0  \cite{AugustoCDRP19}      & +Fork \cite{spdwe} & $20$ & $\;\;0$ & $32$  \\
		
		
		\textbf{SM\_PAIR\_20} & SplitMiner 2.0  \cite{AugustoCDRP19}      & +PairScale \cite{spdwe} & $20$ & $\;\;0$ & $157$ \\
		
		\textbf{SM\_PETRI\_20} & \multicolumn{2}{c||}{Rogge-Solti \cite{RoggeSoltiAW13}} & $20$ & $10^{-5}$ & $1612$ \\
		\bottomrule
	\end{tabular}
\end{adjustbox}
\end{table}
\section{Experimental Results}\label{sec:exp}
\subsection{Dataset}
For our experiments, we took the Sepsis Cases Event Log \cite{mannhardt_2016}, splitted the dataset into the ``\textit{happy traces}'' occurring approximately near to the trace average length ($\leq 2.3\cdot 10^{7}$ ms), and used this dataset to generate either a USWN via ProM \cite{RoggeSoltiAW13} and a BPMN with only exclusive gates using Split Miner 2.0 \cite{AugustoCDRP19} that was then converted into a Petri Net \cite{PPNFromLog}. Such Petri Net was later on converted into a USWN by using a firing weight estimator: we choose the Fork and the PairScale estimators from \cite{spdwe} and we denote as \texttt{Constant} a naïve estimator assuming that each all the transition probabilities from a given marking are equiprobable. No estimator was used for the USWN generated via ProM, as such engine already estimates the firing weights. From such USWNs, we generate distinct sets of unravelled traces. All these steps are summarised in Table \ref{tab:dataset}. The following experiments have the aim of evaluating the benefits of performing the Approximate-Ranking strategy over the Optimal-Ranking one.

\begin{figure*}[!t]
\begin{minipage}{.49\textwidth}

	\includegraphics[width=1.1\textwidth]{images/Prec.pdf}
	\caption{Approximation comparison.}\label{fig:app}
\end{minipage}\hfill \begin{minipage}{.49\textwidth}

	\includegraphics[width=1.1\textwidth]{images/kronos.pdf}
	\caption{$k$NN alignment benchmark.}\label{fig:kronos}
\end{minipage}
\end{figure*}
\subsection{Approximation}\label{subsec:apprp}
In order to assess how well the proposed Approximate-Ranking strategy approximates the Optimal-Ranking one, we use the Spearman Correlation Index \cite{} to express the correlation between each sub-embedding strategies for $\phi_{\mathcal{P}}$ ($\color{ggplotGreen}\epsilon^1\&\nu^1$, $\color{ggplotRed}\epsilon^2\&\nu^1$, $\color{ggplotPurple}\epsilon^1\&\nu^2$, and $\color{ggplotBlue}\epsilon^2\&\nu^2$) and the Optimal-Ranking one.
In general, we can see form the plots that when aligning longer traces, the correlation of the approximate rankings with the optimal one is lower. This is due to the fact that, in this case, the embedding representing the trace to align is longer and this implies also a larger approximation error. We can also observe that the sub-embeddings considering only information about the edges (i.e., the one where the features corresponding to the $\nu$ dimension are set to zero) have in general a higher correlation with the Optimal-Ranking strategy, but their correlation values are less stable with respect to the length of the trace to be aligned. In the case of \textbf{SM\_FORK\_20}, the correlation is maximum for all sub-embedding strategies.

%set of unravelled traces in Table \ref{tab:dataset} and the subset of the Sepsis Cases Event Log that was not used to generate the USWNs. For each of this log trace $\sigma^*$ we added controlled noise (transition addition, deletion, or swap) at either $20\%$ ($\tilde{\sigma}^*$) or $30\%$ ($\tilde{\tilde{{\sigma}}}^*$) of the log trace as for \cite{LeoniM17}. Then, we found the correlation between the ranking $R_\star$ induced by $k_{\phi_{\mathcal{P}}}(\sigma,\sigma^*)$ to the ranking induced by replacing $\sigma^*$ with one of the two noised traces (either a ranking $R_{20}$ induced by $k_{\phi_{\mathcal{P}}}(\sigma,\tilde{\sigma}^*)$ or $R_{30}$ induced by $k_{\phi_{\mathcal{P}}}(\sigma,\tilde{\tilde{\sigma}}^*)$). The correlation $\rho$ between these two rankings ($\rho(R_\star,R_{20})$ and $\rho(R_\star,R_{30})$) is performed via Spearman Correlation Index $\rho$: such index will return near-$1$ on increasing monotonic trend, near-$(-1)$ values on decreasing monotonic trend, and near-$0$ values where the two rankings are almost uncorrelated. Figure \ref{fig:app} shows the outcome of such experiments for all the possible combinations of $\epsilon$ and $\nu$ sub-embeddings for $\phi_{\mathcal{P}}$ while varying the log trace length. We can observe that strategies including traces' frequencies ($\nu^1$) are more stable if compared to strategies where such information is completely ignored ($\nu^2$). Furthermore, such approximation never reaches zero values, while that occurrence might happen for $\nu^2$-based strategies.

\subsection{Efficiency}\label{subsec:efficio}
With reference to the plots in Figure \ref{fig:kronos}, 
we evaluated the efficiency of computing the trace alignment over both Optimal- ({\color{ggplotPurple}Purple} and {\color{ggplotGreen}Green}) and Approximated-ranking ({\color{ggplotBlue}Blue} and {\color{ggplotRed}Red}) strategies over two different data structures enabling $k$NN queries, i.e., VP-Trees ({\color{ggplotBlue}Blue} and {\color{ggplotPurple}Purple}) and KD-Trees ({\color{ggplotRed}Red} and {\color{ggplotGreen}Green}). We conduct our experiments for $k=20$\ADD{, and we use the Levenshtein distance as a baseline distance for the alignment cost function}. While the query time for the Optimal-Ranking  includes the additional \textit{indexing time} for generating all the vectors to the neighbourhood search, the Approximated-Ranking  adds the neighbourhood search time with the embedding transformation of $x$; in particular, we consider the average embedding time for all the possible embedding strategies described in the previous experiment setting. Figure \ref{fig:kronos} plots the result of such experiments: the time required to generate and load all the $\phi_\star$ truly dominates the cost of generating the embedding $\phi_{\mathcal{P}}(x)$ for bigger datasets such as \textbf{STPETRI\_20}, while the cost for $\phi_{\mathcal{P}}(x)$ becomes non-negligible when we have just a few traces to align (\textbf{SM\_FORK\_20}). Last, while the $k$NN alignments over $\phi_{\mathcal{P}}$ always provide the best timing results via KD-Trees, we cannot elect a best data structure $\phi_\star$ that is valid for all datasets and all trace lengths. 
% !TeX root=../main.tex

\section{Conclusions}
%\texttt{\color{red}[TODO]}
\label{sec:conclusion}

In this paper, we have presented an approach to tackle the probabilistic trace alignment as a $k$NN problem.
The approach balances between the likelihood of the aligned trace and the cost of the alignment by providing the top-k alignments instead of a single alignment as output. The experimentation shows that the approximated top-k ranking provides a good trade-off between accuracy and efficiency especially when the reference stochastic net generates several model traces.
Future works will investigate the probabilistic alignment over fuzzy-labeled nodes and declarative process models, thus allowing to generalize the proposed approach to noisy walks in plan recognition for non-rational agents \cite{RamirezG10}. Also, we will try to improve the performance (in terms of efficiency and accuracy) of the proposed approach by intervening both on the embedding and the algorithmic strategies, such as extending the Viterbi Algorithm  to perform $k$ candidates instead of the one maximizing probability and similarity value. Last, we will also investigate the possibility of representing POMDPs as Transition Graphs when the reward function is completely determined by the distance between traces' actions.

%\section*{Acknowledgements}
%This research has been partially supported by the project IDEE (FESR1133) funded by the Eur.\ Reg.\ Development Fund (ERDF) 
%Investment for Growth and Jobs Programme 2014-2020. 


\bibliographystyle{splncs04}
\bibliography{biblio3}
%,bibliography2,references,libraryFiltered,main,Bibliography-CDC,DiCiccio,biblio}




\end{document}
