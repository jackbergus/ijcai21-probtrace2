% This is samplepaper.tex, a sample chapter demonstrating the
% LLNCS macro package for Springer Computer Science proceedings;
% Version 2.20 of 2017/10/04
%
\documentclass[runningheads,final]{llncs}
%
%\documentclass[runningheads]{llncs}

%\setlength{\paperheight}{232.8mm}
%\setlength{\paperwidth}{151.5mm}
%\setlength\voffset     {-23mm}
%\setlength\hoffset     {-34mm}


\usepackage{hyperref}
%\usepackage[framemethod=TikZ]{mdframed}
\usepackage{braket}
\usepackage{multicol}
\usepackage{xcolor}
\usepackage[final]{changes}
\usepackage{graphicx}
\usepackage{url}
\usepackage{amsmath}
\usepackage{bm}
\usepackage{multirow}
\usepackage{enumitem}
\usepackage{tcolorbox}
\usepackage{booktabs}
\usepackage{tabularx}
\usepackage{rotating}
\usepackage{subfig}
\usepackage{latexsym,amssymb,amsmath}

\usepackage{makecell}
\usepackage{xspace}
%\usepackage{times}
%\usepackage[obeyFinal]{todonotes}

\usepackage{scalerel}
\usepackage[ruled,noend]{algorithm2e}
\usepackage{paralist}
\usepackage{wrapfig}
\usepackage{adjustbox}



\newcommand{\DEBUG}{}
\ifdefined\DEBUG
\input{with_edits.tex}
\else
%
\usepackage{wrapfig}

%\usepackage{lineno}
%\linenumbers
\newcommand{\linerange}[2]{%
	\ifthenelse{\equal{\getrefnumber{#1}}{\getrefnumber{#2}}}{%
		line \ref{#1}%
	}{%
		lines \ref{#1}--\ref{#2}%
	}%
}
\usepackage{pifont}
\usepackage{hyperref}
\usepackage{multirow}
\usepackage{booktabs}
\usepackage[utf8]{inputenc}
\usepackage[pdftex]{graphicx}
\usepackage{subfig}
\usepackage{multirow,bigdelim} % multirow in table and big bracket
\usepackage{amsmath}
\newtheorem{lemma}{Lemma}
\usepackage{array}
\usepackage{xcolor}
%\usepackage{comment}
\usepackage{varioref}
\usepackage{xfrac}
\usepackage{adjustbox}

\usepackage{algorithm}
\usepackage[noend]{algpseudocode}

\newcommand{\ignore}[1]{}
\usepackage{float}
\usepackage{listings}
%\usepackage[new]{old-arrows}
\lstdefinelanguage{sparql}{
	morecomment=[l][]{\#},
	morestring=[b][]\",
	morekeywords={BIND,URI,CONCAT,SELECT,CONSTRUCT,DESCRIBE,ASK,WHERE,FROM,NAMED,PREFIX,BASE,OPTIONAL,FILTER,GRAPH,LIMIT,OFFSET,SERVICE,UNION,EXISTS,NOT,BINDINGS,MINUS,a},
	sensitive=true
}
\lstdefinelanguage{cypher}{
	morekeywords={MATCH,RETURN,WHERE,DISTINCT,WITH,CREATE,COUNT,AS,UNION,ALL,is,null,NOT,AND,OR},
	sensitive=true,
	morecomment=[l]{//}, % l is for line comment
}
\definecolor{eclipseBlue}{RGB}{42,0.0,255}
\definecolor{eclipseGreen}{RGB}{63,127,95}
\definecolor{eclipsePurple}{RGB}{127,0,85}
\lstset{basicstyle=\ttfamily,%
	%backgroundcolor=\color[rgb]{0.85,0.85,0.86},%
	frame=single,
	framerule=0pt,
	xleftmargin=\fboxsep,
	xrightmargin=\fboxsep,
	commentstyle=\color{eclipseGreen}, % style of comments
	keywordstyle=\color{eclipsePurple}\bfseries, % style of keywords
	stringstyle=\color{eclipseBlue},
	breaklines=true,
	postbreak=\raisebox{0ex}[0ex][0ex]{\ensuremath{\color{red}\hookrightarrow\space}}
}
\captionsetup[lstlisting]{position=top}

\newtheorem{example}{Example}
\newtheorem{definition}{Definition}

%\ifCLASSOPTIONcompsoc
%\usepackage[nocompress]{cite}
%\else
\usepackage[noadjust,nocompress]{cite}
\usepackage[author={Giacomo Bergami}]{pdfcomment}
\makeatletter
\makeatother
%\fi


\usepackage{amssymb}
\usepackage{ifsym}
\usepackage{paralist}
\usepackage[inline]{enumitem}
\newlist{myalist}{enumerate*}{1}
\setlist[myalist]{label=\textbf{(\arabic*)}}
\newlist{mylist}{enumerate*}{1}
\setlist[mylist]{label=\textbf{(\roman*)}}
\newlist{alphalist}{enumerate*}{1}
\setlist[alphalist]{label=\textbf{(\alph*)}}

\usepackage{braket}

\DeclareMathOperator{\dom}{dom}
\DeclareMathOperator{\cod}{cod}

\makeatletter
\newcommand\notsotiny{\@setfontsize\notsotiny{6.31415}{7.1828}}
\makeatother 

\newenvironment{reviewerextension}{}{}

\received{xxx}
\revised{xxx}
\accepted{xxx}

\pubyear{2000}
\pagerange{\pageref{firstpage}--\pageref{lastpage}}
\volume{xxx}



%\usepackage[printwatermark]{xwatermark}
%\newwatermark[allpages,color=red!50,angle=45,scale=3,xpos=30,ypos=-70]{Under Review}


\usepackage{changes}
\usepackage{lipsum}% <- For dummy text
\definechangesauthor[name={Giacomo Bergami}, color=red]{del}
%\setremarkmarkup{(#2)}
%\setremarkmarkup{(#2)}
%DIF PREAMBLE EXTENSION ADDED BY LATEXDIFF
%DIF UNDERLINE PREAMBLE %DIF PREAMBLE
\RequirePackage[normalem]{ulem} %DIF PREAMBLE
\RequirePackage{color}\definecolor{RED}{rgb}{1,0,0}\definecolor{BLUE}{rgb}{0,0,1} %DIF PREAMBLE
\providecommand{\DIFaddtex}[1]{{\protect\color{blue}\uwave{#1}}} %DIF PREAMBLE
\providecommand{\DIFdeltex}[1]{{\protect\color{red}\sout{#1}}}                      %DIF PREAMBLE
%DIF SAFE PREAMBLE %DIF PREAMBLE
\providecommand{\DIFaddbegin}{} %DIF PREAMBLE
\providecommand{\DIFaddend}{} %DIF PREAMBLE
\providecommand{\DIFdelbegin}{} %DIF PREAMBLE
\providecommand{\DIFdelend}{} %DIF PREAMBLE
\providecommand{\DIFmodbegin}{} %DIF PREAMBLE
\providecommand{\DIFmodend}{} %DIF PREAMBLE
%DIF FLOATSAFE PREAMBLE %DIF PREAMBLE
\providecommand{\DIFaddFL}[1]{\DIFadd{#1}} %DIF PREAMBLE
\providecommand{\DIFdelFL}[1]{\DIFdel{#1}} %DIF PREAMBLE
\providecommand{\DIFaddbeginFL}{} %DIF PREAMBLE
\providecommand{\DIFaddendFL}{} %DIF PREAMBLE
\providecommand{\DIFdelbeginFL}{} %DIF PREAMBLE
\providecommand{\DIFdelendFL}{} %DIF PREAMBLE
%DIF HYPERREF PREAMBLE %DIF PREAMBLE
\providecommand{\DIFadd}[1]{#1} %DIF PREAMBLE
\providecommand{\DIFdel}[1]{#1} %DIF PREAMBLE
\newcommand{\DIFscaledelfig}{0.5}
%DIF HIGHLIGHTGRAPHICS PREAMBLE %DIF PREAMBLE
\RequirePackage{settobox} %DIF PREAMBLE
\RequirePackage{letltxmacro} %DIF PREAMBLE
\newsavebox{\DIFdelgraphicsbox} %DIF PREAMBLE
\newlength{\DIFdelgraphicswidth} %DIF PREAMBLE
\newlength{\DIFdelgraphicsheight} %DIF PREAMBLE
% store original definition of \includegraphics %DIF PREAMBLE
\LetLtxMacro{\DIFOincludegraphics}{\includegraphics} %DIF PREAMBLE
\newcommand{\DIFaddincludegraphics}[2][]{{\color{blue}\fbox{\DIFOincludegraphics[#1]{#2}}}} %DIF PREAMBLE
\newcommand{\DIFdelincludegraphics}[2][]{% %DIF PREAMBLE
	\sbox{\DIFdelgraphicsbox}{\DIFOincludegraphics[#1]{#2}}% %DIF PREAMBLE
	\settoboxwidth{\DIFdelgraphicswidth}{\DIFdelgraphicsbox} %DIF PREAMBLE
	\settoboxtotalheight{\DIFdelgraphicsheight}{\DIFdelgraphicsbox} %DIF PREAMBLE
	\scalebox{\DIFscaledelfig}{% %DIF PREAMBLE
		\parbox[b]{\DIFdelgraphicswidth}{\usebox{\DIFdelgraphicsbox}\\[-\baselineskip] \rule{\DIFdelgraphicswidth}{0em}}\llap{\resizebox{\DIFdelgraphicswidth}{\DIFdelgraphicsheight}{% %DIF PREAMBLE
				\setlength{\unitlength}{\DIFdelgraphicswidth}% %DIF PREAMBLE
				\begin{picture}(1,1)% %DIF PREAMBLE
				\thicklines\linethickness{2pt} %DIF PREAMBLE
				{\color[rgb]{1,0,0}\put(0,0){\framebox(1,1){}}}% %DIF PREAMBLE
				{\color[rgb]{1,0,0}\put(0,0){\line( 1,1){1}}}% %DIF PREAMBLE
				{\color[rgb]{1,0,0}\put(0,1){\line(1,-1){1}}}% %DIF PREAMBLE
				\end{picture}% %DIF PREAMBLE
			}\hspace*{3pt}}} %DIF PREAMBLE
} %DIF PREAMBLE
\LetLtxMacro{\DIFOaddbegin}{\DIFaddbegin} %DIF PREAMBLE
\LetLtxMacro{\DIFOaddend}{\DIFaddend} %DIF PREAMBLE
\LetLtxMacro{\DIFOdelbegin}{\DIFdelbegin} %DIF PREAMBLE
\LetLtxMacro{\DIFOdelend}{\DIFdelend} %DIF PREAMBLE
\DeclareRobustCommand{\DIFaddbegin}{\DIFOaddbegin \let\includegraphics\DIFaddincludegraphics} %DIF PREAMBLE
\DeclareRobustCommand{\DIFaddend}{\DIFOaddend \let\includegraphics\DIFOincludegraphics} %DIF PREAMBLE
\DeclareRobustCommand{\DIFdelbegin}{\DIFOdelbegin \let\includegraphics\DIFdelincludegraphics} %DIF PREAMBLE
\DeclareRobustCommand{\DIFdelend}{\DIFOaddend \let\includegraphics\DIFOincludegraphics} %DIF PREAMBLE
\LetLtxMacro{\DIFOaddbeginFL}{\DIFaddbeginFL} %DIF PREAMBLE
\LetLtxMacro{\DIFOaddendFL}{\DIFaddendFL} %DIF PREAMBLE
\LetLtxMacro{\DIFOdelbeginFL}{\DIFdelbeginFL} %DIF PREAMBLE
\LetLtxMacro{\DIFOdelendFL}{\DIFdelendFL} %DIF PREAMBLE
\DeclareRobustCommand{\DIFaddbeginFL}{\DIFOaddbeginFL \let\includegraphics\DIFaddincludegraphics} %DIF PREAMBLE
\DeclareRobustCommand{\DIFaddendFL}{\DIFOaddendFL \let\includegraphics\DIFOincludegraphics} %DIF PREAMBLE
\DeclareRobustCommand{\DIFdelbeginFL}{\DIFOdelbeginFL \let\includegraphics\DIFdelincludegraphics} %DIF PREAMBLE
\DeclareRobustCommand{\DIFdelendFL}{\DIFOaddendFL \let\includegraphics\DIFOincludegraphics} %DIF PREAMBLE
%DIF LISTINGS PREAMBLE %DIF PREAMBLE
\RequirePackage{listings} %DIF PREAMBLE
\RequirePackage{color} %DIF PREAMBLE
\lstdefinelanguage{DIFcode}{ %DIF PREAMBLE
	%DIF DIFCODE_UNDERLINE %DIF PREAMBLE
	moredelim=[il][\color{red}\sout]{\%DIF\ <\ }, %DIF PREAMBLE
	moredelim=[il][\color{blue}\uwave]{\%DIF\ >\ } %DIF PREAMBLE
} %DIF PREAMBLE
\lstdefinestyle{DIFverbatimstyle}{ %DIF PREAMBLE
	language=DIFcode, %DIF PREAMBLE
	basicstyle=\ttfamily, %DIF PREAMBLE
	columns=fullflexible, %DIF PREAMBLE
	keepspaces=true %DIF PREAMBLE
} %DIF PREAMBLE
\lstnewenvironment{DIFverbatim}{\lstset{style=DIFverbatimstyle}}{} %DIF PREAMBLE
\lstnewenvironment{DIFverbatim*}{\lstset{style=DIFverbatimstyle,showspaces=true}}{} %DIF PREAMBLE
%DIF END PREAMBLE EXTENSION ADDED BY LATEXDIFF
\usepackage{stackengine}
\usepackage{MnSymbol}
\usepackage[many]{tcolorbox}
\newtcolorbox{cross}{blank,breakable,parbox=false,
	overlay={\draw[red,line width=5pt] (interior.south west)--(interior.north east);
		\draw[red,line width=5pt] (interior.north west)--(interior.south east);}}


\newcommand*\mnote[3][0pt]{%
	\if l#2\reversemarginpar\def\pointer{\filledmedtriangleright}%
	\def\stackalignment{r}\fi%
	\if r#2\normalmarginpar\def\pointer{\filledmedtriangleleft}%
	\def\stackalignment{l}\fi%
	\marginpar{%
		\topinset{%
			\scalebox{1.5}{\textcolor{blue}{$\pointer$}}}{%
			\belowbaseline[-1.5\baselineskip-#1]{%
				\stackengine%
				{-5pt}%
				{\fcolorbox{blue}{white}{\parbox{4cm}%
						{\vspace{5pt}\raggedright#3}}}%
				{~\colorbox{white}{\sffamily Note}}%
				{O}%
				{l}%
				{F}%
				{F}%
				{S}%
			}%
		}{%
			3ex+#1}{%
			-2ex}%
	}%
}

\newcommand{\lnote}[1]{}
\newcommand{\rnote}[1]{}
\newcommand{\addswig}[1]{#1}
\newcommand{\delstrike}[1]{}
\newcommand{\strikepic}[1]{}


\usepackage{zref-savepos}
\usepackage{changepage}
\newcounter{mycomment}
\newcounter{myinterp}
\newenvironment{Reply}
{\unskip\stepcounter{mycomment}\zsavepos{endpar\the\value{mycomment}}%
	\par\bigskip%
	%\bfseries%
	\begin{adjustwidth}{1cm}{0cm}\textbf{[Reply \the\value{mycomment}.]}}
	{\end{adjustwidth}
	\par\bigskip
	\noindent\hspace*{\dimexpr\zposx{endpar\the\value{mycomment}}sp-\oddsidemargin-1in}\ignorespaces}
\newenvironment{Interp}
{\unskip\stepcounter{myinterp}\zsavepos{endpar\the\value{myinterp}}%
	\par\bigskip%
	%\bfseries%
	\begin{adjustwidth}{1cm}{0cm}\textbf{[Interpretazione \the\value{myinterp}.]}}
	{\end{adjustwidth}
	\par\bigskip
	\noindent\hspace*{\dimexpr\zposx{endpar\the\value{myinterp}}sp-\oddsidemargin-1in}\ignorespaces}


\usepackage{cancel}
\newcommand\Rcancel[1]{ }
\renewcommand{\replaced}[2]{#1}
\renewcommand{\added}[1]{#1}
\renewcommand{\deleted}[2][]{}
\fi
\usepackage{kbordermatrix}
\usepackage{thmtools}

%%%%%%%%%%%%%%%%%%%%%%%%%%%%%%%%%%%%%%%%%%%%%%%%%%%%%%%%%%%%%%%%%%%%%%%%%%%%%%
%%% Time-stamp: "2018-09-07 18:35:03 calvanese"
%%%%%%%%%%%%%%%%%%%%%%%%%%%%%%%%%%%%%%%%%%%%%%%%%%%%%%%%%%%%%%%%%%%%%%%%%%%%%%

%%%%%%%%%%%%%%%%%%%%%%%%%% General Math

\newcommand{\A}{\ensuremath{\mathcal{A}}}
\newcommand{\B}{\ensuremath{\mathcal{B}}}
%\newcommand{\C}{\ensuremath{\mathcal{C}}}
\newcommand{\D}{\ensuremath{\mathcal{D}}}
\newcommand{\E}{\ensuremath{\mathcal{E}}}
\newcommand{\F}{\ensuremath{\mathcal{F}}}
%\newcommand{\G}{\ensuremath{\mathcal{G}}}
\renewcommand{\H}{\ensuremath{\mathcal{H}}}
\newcommand{\I}{\ensuremath{\mathcal{I}}}
\newcommand{\J}{\ensuremath{\mathcal{J}}}
\newcommand{\K}{\ensuremath{\mathcal{K}}}
\renewcommand{\L}{\ensuremath{\mathcal{L}}}
\newcommand{\M}{\ensuremath{\mathcal{M}}}
\newcommand{\N}{\ensuremath{\mathcal{N}}}
\renewcommand{\O}{\ensuremath{\mathcal{O}}}
\renewcommand{\P}{\ensuremath{\mathcal{P}}}
\newcommand{\Q}{\ensuremath{\mathcal{Q}}}
\newcommand{\R}{\ensuremath{\mathcal{R}}}
%\renewcommand{\S}{\ensuremath{\mathcal{S}}}
\newcommand{\T}{\ensuremath{\mathcal{T}}}
%\newcommand{\U}{\ensuremath{\mathcal{U}}}
\newcommand{\V}{\ensuremath{\mathcal{V}}}
\newcommand{\W}{\ensuremath{\mathcal{W}}}
\newcommand{\X}{\ensuremath{\mathcal{X}}}
\newcommand{\Y}{\ensuremath{\mathcal{Y}}}
\newcommand{\Z}{\ensuremath{\mathcal{Z}}}

%%%%%%%%%%%%%%%%%%%%%%%%%% Abbreviations

%%\newcommand{\eset}{\emptyset}
%%\newcommand{\col}{\colon}
\newcommand{\ol}[1]{\overline{#1}}                % overline
%\newcommand{\ul}[1]{\underline{#1}}               % underline
%%\newcommand{\uls}[1]{\underline{\raisebox{0pt}[0pt][0.45ex]{}#1}}
%% ul with space between text and line

\newcommand{\ra}{\rightarrow}
\newcommand{\Ra}{\Rightarrow}
\newcommand{\la}{\leftarrow}
\newcommand{\La}{\Leftarrow}
%\newcommand{\lra}{\leftrightarrow}
\newcommand{\Lra}{\Leftrightarrow}
\newcommand{\lora}{\longrightarrow}
\newcommand{\Lora}{\Longrightarrow}
\newcommand{\lola}{\longleftarrow}
\newcommand{\Lola}{\Longleftarrow}
\newcommand{\lolra}{\longleftrightarrow}
\newcommand{\Lolra}{\Longleftrightarrow}
%\newcommand{\ua}{\uparrow}
\newcommand{\Ua}{\Uparrow}
\newcommand{\da}{\downarrow}
\newcommand{\Da}{\Downarrow}
\newcommand{\uda}{\updownarrow}
\newcommand{\Uda}{\Updownarrow}

%%%%%%%%%%%%%%%%%%%%%%%%%% Relations

%%\newcommand{\incl}{\subseteq}
%%\newcommand{\imp}{\rightarrow}
\newcommand{\per}{\mbox{\bf .}}                  % period

%%%%%%%%%%%%%%%%%%%%%%%%%% Delimiters

%%\newcommand{\quotes}[1]{{\lq\lq #1\rq\rq}}
%\newcommand{\set}[1]{\{#1\}}                      % set
%\newcommand{\Set}[1]{\left\{#1\right\}}
\newcommand{\bigset}[1]{\Bigl\{#1\Bigr\}}
\newcommand{\bigmid}{\Big|}
\newcommand{\size}[1]{|{#1}|}                     % cardinality of a set
%%\newcommand{\Card}[1]{\left| #1\right|}
\newcommand{\card}[1]{\sharp #1}
\newcommand{\tup}[1]{\langle #1\rangle}            % tuple
\newcommand{\Tup}[1]{\Braket{#1}}
\newcommand{\norm}[2]{\|#1\|_{#2}}
\newcommand{\setone}[2][1]{\set{#1\cld #2}}

%%%%%%%%%%%%%%%%%%%%%%%%%% STYLING AND SPACING

%\newcommand{\inlinetitle}[1]{\smallskip\noindent\textbf{#1.}\xspace}





\newcolumntype{C}{>{\centering\arraybackslash}X}

%\makeatletter
%\g@addto@macro\normalsize{%
%\setlength{\abovecaptionskip}{-2pt}
%\setlength{\belowcaptionskip}{12pt}
%\setlength\abovedisplayskip{3pt}
%\setlength\belowdisplayskip{3pt}
%\setlength\abovedisplayshortskip{3pt}
%\setlength\belowdisplayshortskip{3pt}
%}
%\makeatother

\newcounter{dummy} 
\newcounter{dummy1} 
\newcounter{dummy2}
\newcounter{dummy3} 
\newcounter{dummy4}
\newcounter{dummy5} 
\newcounter{dummy6}
\newcounter{dummy7}
%\numberwithin{dummy}{section}

\usepackage[thmmarks,amsmath]{ntheorem}
%\theorempreskip{1pt}
%\theorempostskip{1pt}

%\theoremstyle{plain}
%\theorembodyfont{\normalfont}
%\theoremseparator{.}
%\let\definition\relax
%\theoremsymbol{\ensuremath{\square}}
%\newtheorem{definition}{Definition}


\let\proposition\relax
\let\theorem\relax
\let\lemma\relax
\let\definition\relax
\theoremseparator{.}
\theorembodyfont{\itshape}
\theoremsymbol{$\triangleleft$}
\newtheorem{theorem}[dummy]{Theorem}
\newtheorem{lemma}[dummy1]{Lemma}
\newtheorem{definition}[dummy2]{Definition}
\newtheorem{proposition}[dummy3]{Proposition}

\let\remark\relax
\let\example\relax
%\let\example*\relax
\theorembodyfont{\normalfont}
\newtheorem{example}[dummy4]{Example}
\newtheorem{remark}[dummy5]{Remark}
%\newtheorem{example*}[dummy4]{Example}


\theoremstyle{nonumberplain}
\theoremheaderfont{\itshape}
\theorembodyfont{\normalfont}
\let\proof\relax
\theoremseparator{.}
\theoremsymbol{\ensuremath{\dashv}}
\newtheorem{proof}[dummy6]{Proof}


\qedsymbol{\ensuremath{\dashv}}


%%% Local Variables:
%%% mode: latex
%%% TeX-master: "main"
%%% save-place: t
%%% End:

\usepackage{tkz-graph}
\usetikzlibrary{matrix,automata,arrows,calc,fit}
\usetikzlibrary{decorations.pathreplacing,positioning}


\definecolor{deepblue}{HTML}{0C3B80}
\definecolor{deepgreen}{HTML}{2EA601}
%\definecolor{deepblue}{HTML}{232F3E}
%\definecolor{chaptercolor}{HTML}{232F3E}{0.8}
\definecolor{lightOrange}{HTML}{FFA03C}
\definecolor{darkOrange}{HTML}{F1800A}
\definecolor{lightBlue}{HTML}{0174CD}
\definecolor{greenF}{HTML}{2CBB5C}
\definecolor{cyan}{HTML}{86A6D5}
\definecolor{darkred}{HTML}{8B0000}

\usetikzlibrary{automata,positioning}
\usetikzlibrary{arrows}
\usetikzlibrary{decorations.markings}
\usetikzlibrary{shapes.misc, positioning}
\usetikzlibrary{arrows.meta,decorations.markings}



%%%%%%%%%%%%%%%%%%%%% TIKZS MACROS
\def\DTZU {2ex}

\tikzstyle{taskfg} = [
  text=deepblue,
]

\tikzstyle{taskbg} = [
  fill=cyan!50,
]

\tikzstyle{taskline} = [
  draw=deepblue,
]

\tikzstyle{taskstyle} = [
  ultra thick,
  taskfg,
  taskbg,
  taskline
]

\tikzstyle{task} = [
  rectangle,
  minimum width=12mm,
  minimum height=10mm,
  taskstyle
]

\tikzstyle{smalltask} = [
  rectangle,
  minimum width=8mm,
  minimum height=6mm,
  taskstyle,
  very thick
]

\tikzstyle{constraint} = [
  ultra thick,
  taskline,
  taskbg
]

\tikzstyle{response} = [
  constraint,
  *-triangle 60
]

\tikzstyle{precedence} = [
  constraint,
  *triangle 60-
]
\tikzstyle{succession} = [
  constraint,
  *triangle 60-*
]


\tikzstyle{respondedexistence} = [
  constraint,
  *-
]

\tikzstyle{coexistence} = [
  constraint,
  *-*
]

\tikzstyle{negationconstraint} = [
  constraint,
  postaction={decorate,decoration={markings,
   mark=at position .5 with {\arrow[xshift=0.15*\DTZU]{Bar[width=1.5*\DTZU]}},
   mark=at position .5 with {\arrow[xshift=-0.15*\DTZU]{Bar[width=1.5*\DTZU]}}
  }}
]

\tikzstyle{notcoexistence} = [
  negationconstraint,
  coexistence,
]

\tikzstyle{negationresponse} = [
  constraint,
  response,
  postaction={decorate,decoration={markings,
   mark=at position .5 with {\arrow[xshift=0.15*\DTZU]{Bar[width=1.5*\DTZU]}},
   mark=at position .5 with {\arrow[xshift=-0.15*\DTZU]{Bar[width=1.5*\DTZU]}}
  }}
]

\tikzstyle{negationsuccession} = [
  constraint,
  succession,
  postaction={decorate,decoration={markings,
   mark=at position .65 with {\arrow[xshift=0.15*\DTZU]{Bar[width=1.5*\DTZU]}},
   mark=at position .65 with {\arrow[xshift=-0.15*\DTZU]{Bar[width=1.5*\DTZU]}}
  }}
]



\tikzstyle{exclusivechoice} = [
  constraint,
  -,
  postaction={decorate,decoration={markings,
   mark=at position .5 with {\arrow[xshift=0.5*\DTZU]{Diamond[width=1*\DTZU]}} 
  }}
]

\tikzstyle{choice} = [
  constraint,
  -,
  postaction={decorate,decoration={markings,
   mark=at position .5 with {\arrow[xshift=0.5*\DTZU]{Diamond[width=1*\DTZU]}},
    mark=at position .5 with {\arrow[xshift=0.28*\DTZU,white,scale=.6]{Diamond[width=1*\DTZU]}} 
  }}
]







\tikzstyle{circ} = [
  solid,
  text=black
]

\tikzstyle{timeline} = [
  -,
  thin,
  ]
  
\tikzstyle{snapshot} = [
  -,
  thin,
  densely dotted,
  ]
  

\tikzstyle{objnode} = [
  inner sep=2pt,
  circle,
  ultra thick,
]

\tikzstyle{tobj} = [
  objnode,
  taskfg,
  taskbg,
  taskline
]

\tikzstyle{cobj} = [
  objnode,
  classfg,
  classbg,
  classline
]

\tikzstyle{link} = [
  solid,
  -angle 60
]
%%%%%%%%%%%%%%%%%%%%%
\newcommand{\LTL}{\ensuremath{\text{LTL}}\xspace}
\newcommand{\LTLf}{\ensuremath{\text{LTL}_f}\xspace}
\newcommand{\PLTL}{\ensuremath{\text{PLTL}_f}\xspace}
\newcommand{\PLTLz}{\ensuremath{\text{PLTL}_f^0}\xspace}

\newcommand{\Wuntil}{\mathcal{W}}

\newcommand{\At}{\ensuremath{\mathsf{At}}\xspace}

\newcommand{\nats}{\ensuremath{\mathbb{N}}\xspace}
\newcommand{\acc}{\ensuremath{\mathsf{acc}}\xspace}
\newcommand{\branch}{\ensuremath{\mathsf{b}}\xspace}
\newcommand{\exa}{\ensuremath{\mathsf{exa}}\xspace}
\newcommand{\ini}{\ensuremath{\mathsf{in}}\xspace}
\newcommand{\ms}{\ensuremath{\mathsf{m}}\xspace}
\newcommand{\wt}{\ensuremath{\mathsf{wt}}\xspace}
\newcommand{\run}{\ensuremath{\mathsf{run}}\xspace}
\newcommand{\rnk}{\ensuremath{\mathsf{rk}}\xspace}
\newcommand{\swt}{\ensuremath{\mathsf{w}}\xspace}
\newcommand{\csub}{\ensuremath{\mathsf{csub}}\xspace}
\newcommand{\END}{\ensuremath{\mathsf{END}}\xspace}

\newcommand{\atom}{\ensuremath{\mathbf{a}}\xspace}

\newcommand{\Amc}{\ensuremath{\mathcal{A}}\xspace}
\newcommand{\Bmc}{\ensuremath{\mathcal{B}}\xspace}
\newcommand{\Imc}{\ensuremath{\mathcal{I}}\xspace}
\newcommand{\Lmc}{\ensuremath{\mathcal{L}}\xspace}
\newcommand{\Omc}{\ensuremath{\mathcal{O}}\xspace}
\newcommand{\Pmc}{\ensuremath{\mathcal{P}}\xspace}
\newcommand{\Qmc}{\ensuremath{\mathcal{Q}}\xspace}
\newcommand{\Smc}{\ensuremath{\mathcal{S}}\xspace}
\newcommand{\Wmc}{\ensuremath{\mathcal{W}}\xspace}

\newcommand{\Imf}{\ensuremath{\mathfrak{I}}\xspace}

\newcommand{\PS}{\text{\sc{PSpace}}\xspace}
\newcommand{\ET}{\text{\sc{ExpTime}}\xspace}

\newcommand{\prob}[1]{\ensuremath{\scaleobj{1.2}{\circledcirc}_{#1}}\xspace}

\newcommand{\argmax}{\text{argmax}\xspace}
\newcommand{\declare}{Declare\xspace}
\newcommand{\pdeclare}{ProbDeclare\xspace}
\newcommand{\constraint}[1]{\texttt{#1}}
\newcommand{\activity}[1]{\ensuremath{\mathsf{#1}}}
%\newtheorem{theorem}{Theorem}
%\newtheorem{lemma}[theorem]{Lemma}
%\newtheorem{claim}[theorem]{Claim}
%\newtheorem{proposition}[theorem]{Proposition}
%\newtheorem{corollary}[theorem]{Corollary}
%\theoremstyle{definition}
%\newtheorem{example}[theorem]{Example}
%\newtheorem{definition}[theorem]{Definition}



%% temporal operators
\newcommand{\Next}{{\ensuremath\raisebox{0.25ex}{\text{\scriptsize$\bigcirc$}}}}
\newcommand{\Until}{\ensuremath{\mathbin{\mathcal{U}}}}
\newcommand{\Wntil}{\ensuremath{\mathbin{\mathcal{W}}}}









\allowdisplaybreaks

%%%%%%%%%%%%%%%%%%%%%%%%%%%%%
%%%%%%%  Conference / arXiv Versions  %%%%%%%
\newif\iffull\fulltrue
%\fullfalse		%%% uncomment to make conference version
\newcommand{\FullT}[1]{\iffull #1 \fi}
\newcommand{\FullF}[1]{\iffull \else #1 \fi}
\newcommand{\FullTF}[2]{\iffull #1 \else #2 \fi}
\usepackage{tkz-graph}
\usetikzlibrary{matrix,automata,arrows,calc,fit}
\usetikzlibrary{decorations.pathreplacing,positioning}


\definecolor{deepblue}{HTML}{0C3B80}
\definecolor{deepgreen}{HTML}{2EA601}
%\definecolor{deepblue}{HTML}{232F3E}
%\definecolor{chaptercolor}{HTML}{232F3E}{0.8}
\definecolor{lightOrange}{HTML}{FFA03C}
\definecolor{darkOrange}{HTML}{F1800A}
\definecolor{lightBlue}{HTML}{0174CD}
\definecolor{greenF}{HTML}{2CBB5C}
\definecolor{cyan}{HTML}{86A6D5}
\definecolor{darkred}{HTML}{8B0000}

\usetikzlibrary{automata,positioning}
\usetikzlibrary{arrows}
\usetikzlibrary{decorations.markings}
\usetikzlibrary{shapes.misc, positioning}
\usetikzlibrary{arrows.meta,decorations.markings}



%%%%%%%%%%%%%%%%%%%%% TIKZS MACROS
\def\DTZU {2ex}

\tikzstyle{taskfg} = [
  text=deepblue,
]

\tikzstyle{taskbg} = [
  fill=cyan!50,
]

\tikzstyle{taskline} = [
  draw=deepblue,
]

\tikzstyle{taskstyle} = [
  ultra thick,
  taskfg,
  taskbg,
  taskline
]

\tikzstyle{task} = [
  rectangle,
  minimum width=12mm,
  minimum height=10mm,
  taskstyle
]

\tikzstyle{smalltask} = [
  rectangle,
  minimum width=8mm,
  minimum height=6mm,
  taskstyle,
  very thick
]

\tikzstyle{constraint} = [
  ultra thick,
  taskline,
  taskbg
]

\tikzstyle{response} = [
  constraint,
  *-triangle 60
]

\tikzstyle{precedence} = [
  constraint,
  *triangle 60-
]
\tikzstyle{succession} = [
  constraint,
  *triangle 60-*
]


\tikzstyle{respondedexistence} = [
  constraint,
  *-
]

\tikzstyle{coexistence} = [
  constraint,
  *-*
]

\tikzstyle{negationconstraint} = [
  constraint,
  postaction={decorate,decoration={markings,
   mark=at position .5 with {\arrow[xshift=0.15*\DTZU]{Bar[width=1.5*\DTZU]}},
   mark=at position .5 with {\arrow[xshift=-0.15*\DTZU]{Bar[width=1.5*\DTZU]}}
  }}
]

\tikzstyle{notcoexistence} = [
  negationconstraint,
  coexistence,
]

\tikzstyle{negationresponse} = [
  constraint,
  response,
  postaction={decorate,decoration={markings,
   mark=at position .5 with {\arrow[xshift=0.15*\DTZU]{Bar[width=1.5*\DTZU]}},
   mark=at position .5 with {\arrow[xshift=-0.15*\DTZU]{Bar[width=1.5*\DTZU]}}
  }}
]

\tikzstyle{negationsuccession} = [
  constraint,
  succession,
  postaction={decorate,decoration={markings,
   mark=at position .65 with {\arrow[xshift=0.15*\DTZU]{Bar[width=1.5*\DTZU]}},
   mark=at position .65 with {\arrow[xshift=-0.15*\DTZU]{Bar[width=1.5*\DTZU]}}
  }}
]



\tikzstyle{exclusivechoice} = [
  constraint,
  -,
  postaction={decorate,decoration={markings,
   mark=at position .5 with {\arrow[xshift=0.5*\DTZU]{Diamond[width=1*\DTZU]}} 
  }}
]

\tikzstyle{choice} = [
  constraint,
  -,
  postaction={decorate,decoration={markings,
   mark=at position .5 with {\arrow[xshift=0.5*\DTZU]{Diamond[width=1*\DTZU]}},
    mark=at position .5 with {\arrow[xshift=0.28*\DTZU,white,scale=.6]{Diamond[width=1*\DTZU]}} 
  }}
]







\tikzstyle{circ} = [
  solid,
  text=black
]

\tikzstyle{timeline} = [
  -,
  thin,
  ]
  
\tikzstyle{snapshot} = [
  -,
  thin,
  densely dotted,
  ]
  

\tikzstyle{objnode} = [
  inner sep=2pt,
  circle,
  ultra thick,
]

\tikzstyle{tobj} = [
  objnode,
  taskfg,
  taskbg,
  taskline
]

\tikzstyle{cobj} = [
  objnode,
  classfg,
  classbg,
  classline
]

\tikzstyle{link} = [
  solid,
  -angle 60
]
%%%%%%%%%%%%%%%%%%%%%

\makeatletter
\g@addto@macro\normalsize{%
\setlength{\abovecaptionskip}{3pt}
\setlength{\belowcaptionskip}{-10pt}
\setlength\abovedisplayskip{3pt}
\setlength\belowdisplayskip{3pt}
\setlength\abovedisplayshortskip{3pt}
\setlength\belowdisplayshortskip{3pt}
}
\makeatother

%\theorempreskip{2pt}
%\theorempostskip{2pt}

%%%%%%%%%%%%%%%%%%%%%%%%%%%%%






\newcommand*\circled[1]{\tikz[baseline=(char.base)]{
            \node[shape=circle,draw,inner sep=2pt] (char) {#1};}}


\newcommand{\nodedist}{6mm}
\newcommand{\textsize}{80mm}
\newcommand{\taskdist}{12mm}
\newcommand{\autshift}{5mm}

\tikzstyle{pconstraint} = [
  draw=lightBlue!80,
]

\tikzstyle{pwords} = [
  text=lightBlue!80
]

\newcommand{\exclose}{\ensuremath{\Diamond \activity{close}}}
\newcommand{\negexclose}{\ensuremath{\neg \Diamond \activity{close}}}


\setlength{\parskip}{0ex}
\setlength{\parindent}{0ex}

\newlength{\yellownotewidth}
\setlength{\yellownotewidth}{2.5cm}
\newlength{\yellownoteheight}
\setlength{\yellownoteheight}{2.5cm}
%   -   -   -   -   -   -   -   -   -   -   -   -
% Yellow note...
%   -   -   -   -   -   -   -   -   -   -   -   -
\newcommand{\yellownote}[1]{
	\marginpar{
		\vspace{-0.5\yellownoteheight}
		\begin{center}
			\begin{tikzpicture}
			\draw[white,fill=gray!25,opacity=0.75,shift={(-0.125,-0.125)}] 
			(0,0) rectangle (\yellownotewidth,\yellownoteheight);
			\draw[fill=yellow!35] (0,0) rectangle (\yellownotewidth,\yellownoteheight);
			\draw[opacity=0.45,fill=gray!50] (0.7\yellownotewidth,0) -- 
			(0.9\yellownotewidth,0.45) -- (\yellownotewidth,0.4) -- cycle;
			\node[blue,below] at (0.5\yellownotewidth,\yellownoteheight) {
				\begin{minipage}{\yellownotewidth-1em}
				\scriptsize\sf#1
				\end{minipage}
			};
			\end{tikzpicture}
		\end{center}
		\vspace{0.5\yellownoteheight}
	}
}

%   -   -   -   -   -   -   -   -   -   -   -   -
% Resizeable - Yellow note...
%   -   -   -   -   -   -   -   -   -   -   -   -
\newcommand{\resizeableyellownote}[3]{
	\setlength{\yellownotewidth}{#1cm}
	\setlength{\yellownoteheight}{#2cm}
	\marginpar{
		\vspace{-0.5\yellownoteheight}
		\begin{center}
			\begin{tikzpicture}
			\draw[white,fill=gray!25,opacity=0.75,shift={(-0.125,-0.125)}] 
			(0,0) rectangle (\yellownotewidth,\yellownoteheight);
			\draw[fill=yellow!35] (0,0) rectangle (\yellownotewidth,\yellownoteheight);
			\draw[opacity=0.45,fill=gray!50] (0.7\yellownotewidth,0) -- 
			(0.9\yellownotewidth,0.45) -- (\yellownotewidth,0.4) -- cycle;
			\node[blue,below] at (0.5\yellownotewidth,\yellownoteheight) {
				\begin{minipage}{\yellownotewidth-1em}
				\scriptsize\sf#3
				\end{minipage}
			};
			\end{tikzpicture}
		\end{center}
		\vspace{0.5\yellownoteheight}
	}
}

\newcommand{\approptoinn}[2]{\mathrel{\vcenter{
			\offinterlineskip\halign{\hfil$##$\cr
				#1\propto\cr\noalign{\kern2pt}#1\sim\cr\noalign{\kern-2pt}}}}}

\newcommand{\appropto}{\mathpalette\approptoinn\relax}


\sloppy
\begin{document}
%
\title{Probabilistic Trace Alignment}
%\subtitle{\emph{Vision Paper}}
%
%\titlerunning{Abbreviated paper title}
% If the paper title is too long for the running head, you can set
% an abbreviated paper title here
%
\author{
Giacomo Bergami\textsuperscript{\textdagger} \and
Fabrizio Maria Maggi\inst{1} \and
Marco Montali\inst{1} \and
Rafael Pe\~naloza\inst{2}}

\authorrunning{G.~Bergami, F.M.~Maggi, M.~Montali and R.~Pe\~naloza}
% First names are abbreviated in the running head.
% If there are more than two authors, 'et al.' is used.
%
\institute{
Free University of Bozen-Bolzano, Italy \\\email{\{maggi,montali\}@inf.unibz.it}
\and
University of Milano-Bicocca \\\email{rafael.penaloza@unimib.it}
}

%\vspace{1.5cm}
%
%
\maketitle              % typeset the header of the contribution
%\linespread{0.98}
%\vspace{-0.5cm}

\begin{abstract}[TODO]
\end{abstract}

\keywords{Declarative Process Models, Probabilistic Temporal Logics, Conformance Checking}

%
%
%%

\section{Introduction}\label{introduction}


In the existing literature on conformance checking, one of the most common approach is based on trace alignments. However, the proposed approaches based on trace alignments use crisp process models as reference models. On the other hand, recently, probabilistic conformance checking approaches are gaining momentum, but the existing approaches are used to check the degree of conformance of and event log with respect to a stochastic process model instead of finding trace alignments.
In this paper, for the first time, we provide a conformance checking approach based on trace alignments using stochastic reference models. Conceptually, this requires to handle the two possibly contrasting forces of the cost of the alignment on the one hand and the likelihood of the model trace with respect to which the alignment is computed.
Balancing between the likelihood of the model trace with respect to which we are computing the alignment and the cost of the alignment (if the cost of the alignment is too high even if the model trace is very likely applying too many changes in the original trace is in turn not very likely).




%%%%% Proposed part as the last part of the introduction: 
\texttt{\color{red}[TODO]}
We perform experiments to empirically evaluate some properties of the proposed embedding strategy:
\begin{enumerate}
	\item {We're going to informally assess the degree of the trace alignment approximation induced by the vector kernel $k_{\phi_\mathcal{P}}$  if compared to the exact probabilistic trace alignment while introducing such embedding (\S\ref{subsec:eta}).}
	\item We're going to assess the proposed kernel's appropriateness (\S\ref{subsec:katk}) by taking all the traces $\tau$ generated by a USWN $\mathcal{U}$, add some noise to $\tau$ thus causing noised traces $\tilde{\tau}$, and evaluating the distance between the trace ranking of $\tau$ and $\tilde{\tau}$ over the traces of $P$ (\S\ref{subsec:apprp}). 
	\item Last, we're going to compare the time required to compute the exact trace alignment approach against the embedding-based approach (\S\ref{subsec:efficio}).
\end{enumerate}
\texttt{\color{red}[TODO]}

Implementation\footnote{\url{https://github.com/jackbergus/approxProbTraceAlign}}

\section{Modeling Probabilistic Dynamic Systems}
\label{sec:models}
In this section, we introduce the different models and techniques that will constitute the basis for representing and computing probabilistic trace alignments.

\subsection{Stochastic Workflow Nets}\label{subsec:spn}
As customary in probabilistic conformance checking \cite{DBLP:conf/bpm/LeemansSA19,DBLP:conf/icpm/PolyvyanyyK19,DBLP:journals/tosem/PolyvyanyySWCM20}, we adopt stochastic Petri nets \cite{MarsanCB84,RoggeSoltiAW13} as the underlying formal basis to represent processes. More specifically, we consider an interesting class of stochastic Petri nets with only immediate transitions (i.e., no timed ones), namely untimed Stochastic Workflow Nets (\uswn for short).
We assume to have a set $\alphabet = \tasks \cup \set{\tau}$ of labels, where labels in $\tasks$ indicate process tasks, whereas $\tau$ indicates an invisible execution step ($\tau$-transition). A \emph{trace} is a finite sequence of labels from $\tasks$.

\begin{definition} An \emph{untimed Stochastic Workflow Net (\uswn)}
is a tuple $\net = (P,T,F,\ell,W)$ where:
\begin{compactitem}
\item $(P,T,F)$ is a standard \emph{Workflow net} with places $P$, transitions $T$, and flow relation $F$ such that there is exactly one \emph{input place} with no incoming arc, and exactly one \emph{output place} with no outgoing arcs;
\item $\ell: T \rightarrow \alphabet$ is a \emph{labeling function} mapping each transition $t \in T$ into a label $\ell(t) \in \alphabet$ - this either indicates the task executed upon firing $t$, or the fact that $t$ is an invisible transition (in the latter case, $\ell(t) = \tau$);
\item $W\colon T\to \mathbb{R}^+$ is a \emph{weight function} assigning a positive firing weight to each transition of the net.
\end{compactitem}
\end{definition}
Given an \uswn $\net$, we use dot notation to extract its constitutive components (e.g., $\net.P$ denotes its places). \emph{The same dot notation will be used for the other structures introduced in the paper}. We also use $\net.in$ and $\net.out$ to respectively denote the input and output place of $\net$.

As usual, the current state of execution is captured using a marking of the net, that is, a multiset over places $P$ indicating how many tokens populate each place.
%As pointed out above, \emph{we always assume, as customary in BPM, that the input \uswn is \underline{bounded}}, that is, in every state the number of tokens associated to each place cannot exceed a maximum, fixed threshold.
The notions of transition enablement and firing are also the standard ones, since they do not depend on the weight function, which provides the basis for capturing the stochastic behavior of the net. We use the following notation: given a marking $\marking$ over \uswn $\net$, we denote by $\enaset{\marking}{\net}$ the set of enabled transitions in $\marking$; given transition $t \in \enaset{\marking}{\net}$, we write $\fire{\marking}{t}{\marking'}{\net}$ to capture the fact that, within $\net$, firing $t$ in $\marking$ results in the new marking $\marking'$. A \emph{firing sequence of $\net$ starting from marking $\marking_0$} is a sequence $t_1\cdots t_n$ of transitions from $\net.T$ so that, for every $i \in \set{1,\ldots,n}$, we have that $\fire{\marking_{i-1}}{t_i}{\marking_{i}}{\net}$. We say that the firing sequence results in $\marking_{n}$.

As customary in Workflow nets, we consider two special markings: the \emph{input} (resp.~\emph{output}) marking $m_{in}^\net$ (resp.~$m_{out}^\net$) that assigns a single token to the input (resp.~output) place $\net.in$ (resp.~$\net.out$) of $\net$, and no token elsewhere. A \emph{valid sequence} $\seq = t_1\cdots t_n$ of $\net$ is a firing sequence of $\net$ starting from $m_{in}^\net$ and resulting in $m_{out}^\net$. A sequence of labels $\run = \alpha_1 \cdots \alpha_n$ from $\alphabet$ is a \emph{run} of $\net$ if there exists a valid underlying sequence $\seq = t_1\cdots t_n$ of $\net$  such that, for every $i \in \set{1,\ldots,n}$, we have $\net.\ell(t_i) = \alpha_i$. Run $\run$ may have different underlying valid sequences in $\net$, which we collectively refer to as $\seqs{\run}{\net}$. A trace $\trace$ is a \emph{model trace} of $\net$ (or $\net$-trace for short) if there exists an underlying run $\run$ of $\net$ that corresponds to $\trace$ once all occurrences of $\tau$ are removed. There may be multiple runs underlying an $\net$-trace $\trace$, and we collectively refer to them as $\runs{\trace}{\net}$. Finally, we denote the (possibly infinite) set of $\net$-traces as $\traces{\net}$.
\begin{example} %\small
\label{ex:net}
\figurename~\ref{fig:spn} shows an example of an \uswn with input place $p_1$ and output place $p_7$. One run of the net is $\const{\tau c \tau a a \tau}$, which corresponds to trace $\const{caa}$. Overall, the net supports infinitely many finite traces of the form (represented using regular expressions):
\begin{inparaenum}[\it (i)]
\item $\const{aa^*}$,
\item $\const{cb}$,
\item $\const{caa^*}$.
\end{inparaenum}
\end{example}

When executing an \uswn, the crucial addition to the standard execution semantics of Workflow nets is that, being the net stochastic, in each marking the set of enabled transitions gets associated to a discrete probability distribution. This is defined as follows: given a marking $\marking$ of $N$ and an enabled transition $t \in \enaset{\marking}{\net}$, the \emph{firing probability} of $t$ in $\marking$ is $\probt{t}{\marking}{\net} = \frac{\net.W(t)}{\sum_{t'\in \enaset{\marking}{\net}}\net.W(t')}$. As required, the probabilities associated to all enabled transitions in a marking always add up to 1.

Using firing probabilities as a basic building block, we define the probability $\prob{\seq}{\net}$ of a valid sequence $\seq = t_1\cdots t_n$ of $\net$ as the product of the probabilities associated to each transition: $\prob{\seq}{\net} = \prod_{i \in \set{1,\ldots,n}}\prob{t_i}{\marking_{i-1}}{\net}$. %For a run $\run$ of $\net$, its probability $\prob{\seq}{\net}$ is then obtained by summing up the probabilities of all valid sequences corresponding to $\run$: $\prob{\run}{\net} = \sum_{\seq \in \seqs{\run}{\net}} \prob{\seq}{\net}$. Likewise, for a trace $\trace$ of $\net$, its probability is obtained by summing up
For a trace $\trace$ of $\net$, its probability $\prob{\trace}{\net}$ is then obtained by collecting all its underlying runs, in turn collecting all their underlying valid sequences, and summing up their respective probabilities: $\prob{\trace}{\net} = \sum_{\run \in \runs{\trace}{\net}} \sum_{\seq \in \seqs{\run}{\net}} \prob{\seq}{\net}$. This corresponds to the intuition that, to observe $\trace$, one can equivalently pick any of its underlying valid sequences. Notably, if a trace is not an $\net$-trace (i.e., it does not conform with $\net$), then its probability is 0. For convenience, when needed, we represent an $\net$-trace as a pair $\tup{\trace,\prob{\trace}{\net}}$, where the probability assigned to $\trace$ by $\net$ is retained.


By interpreting concurrency by interleaving, we can represent all transition firings of an \uswn, together with their probabilities, in a reachability graph.

\begin{definition}
The \emph{Reachability Graph} $\rg{\net}$ of \uswn \net is a triple $(M,E,P)$ where:
\begin{compactitem}[$\bullet$]
\item $M$ is the set of all reachable markings from $\marking_0^\net$ (including $\marking_0^\net$ itself).
\item $E \subseteq M \times \alphabet \times M$ is a $\alphabet$-\emph{labeled transition relation} induced by $\net$, that is, for $\marking,\marking' \in M$, we have edge $(\marking,a,\marking') \in E$ if and only if there exists transition $t$ in $\net$ with label $\ell(t) = a$ and such that $\fire{\marking}{t}{\marking'}{\net}$.
\item $P:E \rightarrow [0,1]$ is the \emph{transition probability} function assigning to each transition $(\marking,a,\marking') \in E$ its corresponding probability, obtained from the firing probability of the \uswn transition(s) that lead from $\marking$ to $\marking'$ and are labeled by $a$: $P(\marking,a,\marking') = \sum_{t_i \in \enaset{\marking}{\net} \text{ s.t.~} \net.\ell(t) = a \text{ and } \fire{\marking}{t}{\marking'}{\net}} \prob{t}{\marking}{\net}$.
\end{compactitem}
\end{definition}
Notice that, in the definition, we have to account for the possible case where, in a given state, distinct net transitions with the same label produce the same consequent state. In this case, they are indistinguishable when observing the execution traces of the net, and in fact they collapse into a single edge of the reachability graph. This is why, in this case, we accumulate all their firing probabilities into a single value.


In the remainder of the paper, given an \uswn $\net$, we always assume that it satisfies two structural assumptions that are natural in the BPM setting:
\begin{compactitem}
\item $\net$ is \emph{bounded}, that is, every marking in $\rg{\net}$ assigns at most a pre-defined number of tokens to each place;
\item $\rg{\net}$ does not contain loops where all edges are labeled with $\tau$.
\end{compactitem}
The first assumption indicates that a case of the process does not generate unboundedly many parallel threads, and guarantees in turn that the reachability graph contains finitely many states. The second assumption naturally corresponds to how $\tau$-transitions are used when modeling business processes, where they are essential in representing gateways (such as exclusive and parallel splits/joins), cascaded gateways without tasks in between, and skippable tasks.  In all these cases, multiple $\tau$-transitions may be used, but never creating completely invisible loops. Under this assumption, $\net$ enjoys a very interesting property: given a trace $\trace$, there are only boundedly many valid sequences that can produce it. Hence, the probability of $\trace$ can be computed by:
\begin{inparaenum}[\it (i)]
\item exhaustively enumerating all its valid sequences;
\item calculating the probability of each such sequence;
\item summing up all the so-obtained probabilities.
\end{inparaenum}
\figurename~\ref{fig:rg} shows an example of a reachability graph.
\begin{example} %\small
  \label{ex:trace}
Consider the \uswn \net of Example~\ref{ex:net}. Considering trace $\const{caa}$, it is easy to see that it has only one underlying run, namely $\const{\tau c \tau a a \tau}$, in turn produced by a single underlying valid sequence, and that
%The firing probability of picking the first $\tau$-transition starting from the input marking is $1$, as there are no alternatives. In the new marking, where only one token is assigned to $p_2$, the firing probability of choosing the $\tau$-transition above is $\rho_{23} = \frac{v_{\tau_2}}{v_{\tau_2}+v_c}$, whereas that of choosing the $c$-transition below is $\rho_{24} = \frac{v_c}{v_{\tau_2}+v_c}$. Upon choosing the transition below, the new marking assigns only to $p_4$ one token, leaving just one choice to continue by moving that token to $p_6$. In that marking, the probability of choosing the $a$-transition above is $\rho_{65} = \frac{v_{a_3}}{v_{a_3}+v_b}$, resulting in the token being moved to $p_5$. In this new marking, the probability of iterating over the $a$-transition above is $\rho_{55} = \frac{v_{\tau_3}}{v_{\tau_3}+v_{a_2}}$, while that of completing in the output marking via the enabled $\tau$-transition is $\rho_{57} = \frac{v_{\tau_3}}{v_{\tau_3}+v_{a_2}}$. Hence, all in all
$\prob{\const{caa}}{\net} = 1 \cdot \rho_{24} \cdot 1 \cdot \rho_{65} \cdot \rho_{55} \cdot \rho_{57}$.
\end{example}



%Technically:
%\begin{compactitem}
%\item $P$ is a finite set of \textit{places}.
%\item $T$ is a finite set of \textit{transitions}, each of which is associate to a label. Each label either denotes a task executed upon transition firing, or indicates an invisible transition; in the latter case, we employ the special label $\varepsilon$.\footnotesize{This corresponds to the standard notion of $\tau$-transitions in Petri nets, but we use $\varepsilon$ since in the remainder of the paper $\tau$ is used to refer to an execution trace.}
%%to which we associate a label $\lambda(t)\in\Sigma$, where $\Sigma$ also includes the empty string\footnote{Given that we are going to denote the traces as $\tau$ and $t$ as the Petri net Transitions, we choose to denote the empty string as such instead of $\tau$ as in current literature from Petri nets.} $\varepsilon$.
%\item $F\subseteq (P\times T)\cup (T\times P)$ is the flow relation, representing arcs linking places to transitions and transitions to places.
%%to which we associate a \textit{firing cost} $\omega\colon F\to\mathbb{N}$.
%\item The initial place $i\in P$ has no ingoing edges ($\not\exists t\in T. (t,i)\in F$).
%\item The final place $f\in P$ has no outgoing edges ($\not\exists t\in T. (f,t)\in F$).
%\item $W\colon T\to \mathbb{R}^+_{>0}$ defines a \textit{firing weight} associated to each transition.
%\end{compactitem}

%A \textit{marking} is an assignment of a given amount of indistinguishable tokens to places described by a vector $M\colon P\to \mathbb{N}$. We say that a given transition $t$ is \textit{enabled} if $M(p)\geq 1$ for each ingoing $p$ to $t$ ($(p,t)\in F$). If such transition is enabled, then it can \textit{fire} a token. The \textit{enabling transitions} $E(M)$ for a given marking $M$ are all the $t$ reachable from $p$ ($(p,t)\in F$) with $M(p)\neq 0$ where $t$ is enabled. When $t$ can fire a token for a marking $M$, we can generate a novel marking $M'$ from $M$ by moving the tokens from the ingoing places towards the outgoing places as follows:
%\[\forall p\in P.\; M'(p)=M(p)-\mathbf{1}_{(p,t)\in F}+\mathbf{1}_{(t,p)\in F}\]
%We denote the transition from marking $M$ to marking $M'$ via an enabling $t$ as a relation $M\overset{t}{\to}M'$. We say that an \uswn with initial marking $M$ is $k$-\textit{bounded} if each of the markings $M'$ reachable from $M$, $M$ included, have $\forall p\in P.\; M(p)\leq k$\\

\begin{figure*}[!t]
	\begin{minipage}{.49\textwidth}
		\includegraphics[width=.9\textwidth]{images/petri.pdf}
		\caption{A sample \uswn. Labels are shown in green, $\tau$ transitions in grey, weights in magenta.}\label{fig:spn}
	\end{minipage}\hfill \begin{minipage}{.49\textwidth}
		\includegraphics[width=.9\textwidth]{images/rg.pdf}
		\caption{Reachability graph $RG(N)$ of the \uswn $N$. Probabilities are shown in violet.}\label{fig:rg}
	\end{minipage}
\end{figure*} \begin{figure*}[!t]
	\begin{minipage}{.49\textwidth} \includegraphics[width=.9\textwidth]{images/running_example.pdf}
	\caption{Transition graph $G_{RG(N)}$ encoding the reachability graph $RG(N)$.}\label{fig:lmc}\label{fig:orig}
\end{minipage}\hfill \begin{minipage}{.49\textwidth} \includegraphics[width=.9\textwidth]{images/closed_example.pdf}
	\caption{Transition graph $\closed{G_{RG(N)}}$ resulting from the transition graph in $G_{RG(N)}$ after $\tau$-closure.}\label{fig:closed}
\end{minipage}
\end{figure*}


%Figures \ref{fig:spn} and \ref{fig:rg} respectively show a sample \uswn and its corresponding reachability graph. This net will be our running example throughout the paper.


%\begin{example}
%Figure \ref{fig:spn} provides a sample \uswn defined as such, and \ref{fig:rg} provides its associated Reachability Graph. This representation can be beneficial when such \uswns are inferred and extracted from log files \cite{PPNFromLog} for extracting the set of the probabilistic traces associated to the \uswn.
%\end{example}
%
%
%We use \uswns for modelling business processes: in fact, it can be shown \cite{RaedtsPUWGS07} that it is always possible to convert BPMNs to \uswns. Last, we also assume that a transition is enabled when all of its input places contain at least one token and that, when a transition fires, we remove one token from each of its input places and depose tokens for each of its output places.



\subsection{Transition Graphs}\label{subsec:ppn}

The graph and trace embedding techniques that we will use as the basis for computing probabilistic alignments cannot be directly defined over reachability graphs. In fact, these techniques rely on graphs where edges are only labeled by probabilities, whereas labels are attached to nodes. In addition, towards readily enabling efficient algorithmic techniques, such graphs are compactly defined using transition matrixes. We therefore take inspiration from \cite{GartnerFW03} and introduce the so-called \emph{probabilistic transition graphs}, which we will later use to encode \uswn{s} via their reachability graphs.

%For a matrix $Q$ with row set $A$ and column set $B$, notation $[Q]_{ab}$ for $a \in A$ and $b \in B$ denotes the corresponding element in the matrix. In addition $\transp{Q}$ denotes the transposed matrix where rows and columns are inverted. We employ the usual sum and product operations over matrixes and arrays, and denote, for a square matrix $Q$, the repeated multiplication of $Q$ with itself $n$ times by $Q^n$.\todo{Rimuovere questo paragrafo se serve spazio,}\todo{NOn si capisce il significato di $\omega$}
%In our technical treatment, we continue to assume the existence of a set $\alphabet$ of labels (including the special label $\tau$).
\begin{definition} A \emph{(Probabilistic) Transition Graph} is a tuple $(V,s,t,L,R)$ where:
  \begin{inparaenum}[\itshape (i)]
    \item $V \subset \mathbb{N}$ is a set of \emph{nodes};
    \item $s\in V$ is the \emph{initial node};
    \item $e\in V$ is the \emph{accepting node};
    \item $L: \alphabet \times V \rightarrow \{0,1\}$ is a \emph{label matrix} associating each node in $V$ to a single label in $\alphabet$, where for label $\alpha \in \alphabet$ and node $\ind{i} \in V$, $[L]_{\alpha\ind{i}}$ gives $1$ if $\ind{i}$ is labeled by $\alpha$, $0$ otherwise;
    \item $R: V \times V \rightarrow [0,1]$ is a \emph{(probabilistic) transition matrix} indicating, for each pair of nodes, what is the probability that executing a transition from the first node leads to the second node.
   % \item $\omega \in [0,1]$ is a \emph{graph weight} indicating an overall value associated to the entire graph.
  \end{inparaenum}
$L$ and $R$ satisfy the following well-formedness conditions:
\begin{inparaenum}[\itshape (i)]
\item for every $i \in V$ there is one and only one label $\alpha \in \alphabet$ so   that $[L]_{\alpha\ind{i}}=1$;
\item  for  every $\ind{i} \in V$, we have that $\sum_{\ind{j}\in V}[R]_{\ind{ij}}=1$.
\end{inparaenum}

\ADD{A Weigthed (Probabilistic) Transition Graph is a pair $(G,\omega)$ where $G$ is a (Probabilistic) Transition Graph and $\omega\in[0,1]$ is a weight associated to such graph.}
\end{definition}
The condition for $L$ indicates that each node is mapped by $L$ to a single label, while the same label may be used for multiple nodes. The condition for $R$ ensures that the values contained therein can be interpreted as a probability distribution when choosing which next node to pick upon executing a transition. Matrices $L$ and $R$ can be exploited to determine the probability of reaching a node labeled by $\beta\in\Sigma$ from any node labeled $\alpha\in\Sigma$ in $n$ steps with $[LR^n\transp{L}]_{\alpha\beta}/[L\transp{L}]_{\alpha\alpha}$, that we can shorthand as $[G.\Lambda^n]_{\alpha\beta}$ (see \cite{GartnerFW03} and Example \ref{ex:wheredotiszero}).

A transition graph $\tg$ can be visualized as shown in \figurename~\ref{fig:lmc} (and in \figurename~\ref{fig:closed} after $\tau$-closure). There, the various elements have the obvious interpretation, with the only important consideration that an edge from node $\ind{i}$ to node $\ind{j}$ is only shown if the transition probability $[\tg.R]_{\ind{i}\ind{j}}$ is positive.
%There, each node $\ind{i} \in \tg.V$ is  represented as a circle with its identifying number. The initial node is decorated by a small incoming edge, while the final node is double circled. The label of the node is shown close to the circle, in agreement with $\tg.L$. Finally, an edge from $\ind{i} \in \tg.V$ to $\ind{j} \in \tg.V$ is shown if the transition probability $[\tg.R]_{\ind{i}\ind{j}}$ is positive. Each edge is decorated with the positive probability assigned by $\tg.R$.

%\begin{definition}[Path, trace]
%A \emph{path} in a transition graph $\tg$ is a finite sequence of nodes $\ind{i}_1 \cdots \ind{i}_n$ (with $n > 1$) such that, for every $j \in \set{1,\ldots,n-1}$, we have that $[\tg.R]_{\ind{i}_j\ind{i}_{j+1}} > 0$. Such a path is \emph{valid} if it starts from the initial node and ends in the accepting node of $\tg$, that is, $\ind{i}_1 = \tg.s$ and $\ind{i}_n = \tg.e$.
%
%A \emph{trace} is a finite sequence of nodes that can be turned into a valid sequence by introducing in the sequence an arbitrary number of $\tau$ labels (so as to account for hidden transitions in the graph).
%\end{definition}
%From the definition, it is clear that every valid path can be straightforwardly converted into a corresponding trace by removing all $\tau$ labels from the sequence.

%$\npath{\ind{i}}{\ind{j}}$

By mirroring to definitions of \uswn{s} taking into account that now labels are on nodes, a \emph{valid sequence} of $\tg$ is a sequence $\ind{i}_0\ldots\ind{i}_n$ of nodes in $\tg.V$ that leads from the initial to the accepting node by only traversing transitions with nonzero probability:
\begin{inparaenum}[\it (i)]
\item $\ind{i}_0 = \tg.s$;
\item $\ind{i}_n = \tg.e$;
\item if the sequence contains at least two nodes, each two consecutive nodes are connected by a positive transition probability, i.e., for every $j \in \set{1,\ldots,n}$ we have $[R]_{\ind{i}_{j-1}\ind{i}_{j}} > 0$.
\end{inparaenum}
Runs and model traces of transition graphs are then defined as in \uswn{s}, and we employ the same notation to indicate the runs underlying a model trace, and the valid sequences underlying a run. The computation of probabilities for runs and traces is hence defined equivalently.

%We close this section by introducing how some  matrix operations defined in the literature \cite{GartnerFW03} are applied to matrixes $L$ and $R$ of $\tg$, towards tackling interesting probability computations. These will be instrumental later on in the paper. Given two nodes $\ind{i},\ind{j} \in \tg.V$, $[R^n]_{\ind{i}\ind{j}}$ returns the probability of having a path in $\tg$ that connects $\ind{i}$ to $\ind{j}$ and has length $n$. Given two labels $\alpha,\beta \in \alphabet$, with $[LR^n\transp{L}]_{\alpha\beta}/[L\transp{L}]_{\alpha\alpha}$, we obtain the probability that, starting in any node labeled by $\alpha$, we reach a node labeled by $\beta$ through  $n$ consecutive steps in $\tg$. As a shortcut notation, we call the result $[\tg.\Lambda^n]_{\alpha\beta}$. Since there may be different nodes labeled by $\alpha$, we need to normalize the resulting probabilities. This is obtained with the division by $L\transp{L}$, which does so by assuming a uniform distribution when picking from which specific $\alpha$-labeled node one wants to start. Notice that these calculations need to be refined so as to consider proper runs and  traces. This will be done in Section~\ref{subsec:as}. %\todo{Rimandare alla sezione giusta}



%
%$\texttt{\color{blue}i}\overset{n}{\rightsquigarrow}\texttt{\color{blue}j}$ of length $n$: therefore, $[\Lambda^n]_{\color{green}\alpha\beta}:=[LR^nL^t]_{\color{green}\alpha\beta}/[LL^t]_{\color{green}\alpha\alpha}$ denotes the probability that, having started at any node labeled $\color{green}\alpha$ and taking $n$ steps, we arrive at any node labeled $\color{green}\beta$ (${\color{green}\alpha}\overset{n}{\rightsquigarrow}{\color{green}\beta}$). We denote as ${\color{green}\alpha}{\rightsquigarrow}{\color{green}\beta}$ an aforementioned path of arbitrary length.
%We can also associate a weight $\omega\in[0,1]\subseteq\mathbb{R}$ to a TG, so to express the probability associated with the TG itself as valid.
%




%\begin{example}
%We can graphically represent such TG as in \cite{Myers1989}.
%Figure \ref{fig:orig} is a  TG $P^*=(\mathtt{\color{blue}1},\mathtt{\color{blue}8},L,R,1)$ where $\omega=1$, where the matrices $L$ and $R$ can be both defined as follows:
%$$L:=\kbordermatrix{
%             & \texttt{\color{blue}1}&\texttt{\color{blue}2}&\texttt{\color{blue}3}&\texttt{\color{blue}4}&\texttt{\color{blue}5}&\texttt{\color{blue}6}&\texttt{\color{blue}7}&\texttt{\color{blue}8}&\texttt{\color{blue}9}&\texttt{\color{blue}10}\\
%\color{green}\varepsilon  & \textbf{1}&0&0&0&0&0&\textbf{1}&\textbf{1}&\textbf{1}&\textbf{1}\\
%\color{green}a            & 0&\textbf{1}&0&\textbf{1}&0&\textbf{1}&0&0&0&0\\
%\color{green}b            & 0&0&0&0&\textbf{1}&0&0&0&0&0\\
%\color{green}c            & 0&0&\textbf{1}&0&0&0&0&0&0&0\\
%}\qquad R:=\kbordermatrix{
%& \texttt{\color{blue}1}&\texttt{\color{blue}2}&\texttt{\color{blue}3}&\texttt{\color{blue}4}&\texttt{\color{blue}5}&\texttt{\color{blue}6}&\texttt{\color{blue}7}&\texttt{\color{blue}8}&\texttt{\color{blue}9}&\texttt{\color{blue}10}\\
%\texttt{\color{blue}1}  & 0&0&{\color{red}p_2}&0&0&0&0&0&{\color{red}p_1}&0\\
%\texttt{\color{blue}2}  & 0&0&0&0&0&{\color{red}p_3}&{\color{red}p_6}&0&0&0\\
%\texttt{\color{blue}3}  & 0&0&0&0&0&0&0&0&0&{\color{red}1}\\
%\texttt{\color{blue}4}  & 0&0&0&0&0&{\color{red}p_3}&{\color{red}p_6}&0&0&0\\
%\texttt{\color{blue}5}  & 0&0&0&0&0&0&0&{\color{red}1}&0&0\\
%\texttt{\color{blue}6}  & 0&0&0&0&0&{\color{red}p_3}&{\color{red}p_6}&0&0&0\\
%\texttt{\color{blue}5}  & 0&0&0&0&0&0&0&{\color{red}1}&0&0\\
%\texttt{\color{blue}8}  & 0&0&0&0&0&0&0&0&0&0\\
%\texttt{\color{blue}9}  & 0&{\color{red}1}&0&0&0&0&0&0&0&0\\
%\texttt{\color{blue}10}  & 0&0&0&{\color{red}p_4}&{\color{red}p_5}&0&0&0&0&0\\
%}$$
%\end{example}

% Given a TG $P=(s,t,L,R,\omega)$, a trace $\tau$ is a tuple in $(\Sigma\backslash\{\varepsilon\})^*$ denoting a path always originating from $s$ and terminating in $t$.


\subsection{Kernels and Trace Kernels}\label{subsec:katk}
As a foundational basis to compute trace alignments, we adapt similarity measures from the database literature.  Given a set of data examples $\mathcal{X}$, (e.g., strings or traces, transition graphs) a (positive definite) \emph{kernel} function $k\colon \mathcal{X}\times \mathcal{X}\to \mathbb{R}$ denotes the similarity of elements in $\mathcal{X}$. If $\mathcal{X}$ is the $d$-dimensional Euclidean Space $\mathbb{R}^d$, the simplest kernel function is the inner product $\Braket{\mathbf{x},\mathbf{x}'}=\sum_{1\leq i\leq d}\mathbf{x}_i\mathbf{x}'_i$.
A kernel is said to \emph{perform ideally} \cite{Gartner03} when $k(x,x')=1$ whenever $x$ and $x'$ are the same object (\textit{strong equality}) and $k(x,x')=0$ whenever $x$ and $x'$ are distinct objects (\textit{strong dissimilarity}). A kernel is also said to be \emph{appropriate} when similar elements $x,x'\in\mathcal{X}$ are also close in the feature space. Notice that appropriateness can be only assessed  empirically \cite{Gartner03}.
A positive definite kernel induces a distance metric as:
\begin{equation}\label{eq:dofk}
d_k(\mathbf{x},\mathbf{x}'):=\sqrt{k(\mathbf{x},\mathbf{x})-2k(\mathbf{x},\mathbf{x}')+k(\mathbf{x}',\mathbf{x}')}
\end{equation}
When the kernel of choice is the inner product, the resulting distance is the Euclidean distance $\norm{\mathbf{x}-\mathbf{x}'}{2}$. A normalized vector $\hat{\mathbf{x}}$ is defined as $\mathbf{x}/\norm{\mathbf{x}}{2}$. For a normalized vector we can easily prove that: $\norm{\hat{\mathbf{x}}-\hat{\mathbf{x}}'}{2}^2=2(1-\Braket{\hat{\mathbf{x}},\hat{\mathbf{x}}'})$.

When $\mathcal{X}$ does not represent directly a $d$-dimensional Euclidean space, we can use an \emph{embedding} $\embed\colon\mathcal{X}\to \mathbb{R}^d$ to define a kernel $k_\embed\colon \mathcal{X}\times \mathcal{X}\to\mathbb{R}$ as $k_\embed(x,x'):=\Braket{\embed(x),\embed(x')}$. As a result, $k_\embed(x,x')=k_\embed(x',x)$ for each $x,x'\in\mathcal{X}$.

The literature also provides a kernel representation for strings \cite{LodhiSSCW02,GartnerFW03}, which we can directly employ for our traces. We now provide an intuition describing the desired features of this representation \cite{LodhiSSCW02}. If we associate each dimension in $\mathbb{R}^d$ to a different subtrace $\alpha\beta$ of size $2$ (i.e., $2$-grams\footnote{\label{fn:caveat}For our experiments, we choose to consider only $2$-grams, but any $p$-grams of arbitrary length $p\geq 2$ might be adopted \cite{Gartner03}. An increased size of $p$ improves precision but also incurs in a worse computational complexity, as it requires to consider all the arbitrary subtraces of length $p$ whose constitutive elements occur at any distance from each other within the trace.}), the associated coordinate should represent how frequently and ``compactly'' this subtrace is embedded in the trace $\trace$ of interest. Therefore, we introduce a \emph{decay factor} $\lambda\in[0,1]\subseteq\mathbb{R}$ that, for all $m$ subtraces where $\alpha$ and $\beta$ appear in $\trace$ at the same relative distance $L < |\trace|$, weights the resulting embedding as $\lambda^Lm$.

We need to lift this approach so as to consider all occurrences of subtraces $\alpha\beta$ at every distance between $1$ and $|\trace|-1$. To do so, we proceed in two steps. First, we encode $\trace$ into a ``linear'' transition graph $\tg_\trace$ (\figurename~\ref{fig:taustar}) in the obvious way. %\todo{Tagliare dopo i due punti se necessario.} each node in $G_\sigma.V$  corresponds to an element of the trace labeled correspondingly, and the nodes representing two consecutive elements in the trace are connected with a transition probability of 1 (whereas in all the other cases, the probability is 0).
As a second step, we rely on the matrix operations to calculate a simplified version of the embedding defined in \cite{LodhiSSCW02} as $\trembed_{\alpha\beta}(\trace)=\sum_{1\leq i\leq |\trace|}\lambda^i[(\tg_{\trace}.\Lambda)^i]_{\alpha\beta}$. %\todo{No spazio per spiegare cosa succede...}
%This value can be seen as a reward.
The kernel between two traces corresponds to the sum of the products of such values calculated 2-gram by 2-gram for the two traces.
%, namely it is equal to the \emph{kernel convolution}. %\todo{L'ho provato a scrivere intuitivamente, ma non e' chiaro da dove arrivi questo modo di calcolarlo... deriva dalle formule sopra ma la digressione in mezzo e' lunga. Come possiamo fare per chiarire? L'esempio spiega bene tutto!}
This trace kernel returns strong dissimilarity when the two traces have no shared 2-grams at any arbitrary occurring length, but does not enjoy strong equality (as the similarity of a trace with itself is at least $\lambda^2$ - returned when the trace is a 2-gram).

%
%we can represent it as a TG \cite{Myers1989} $(1,{|\tau|},L_\tau,R_\tau,1)$ having $[L_\tau]_{{\color{green}\alpha}\texttt{\color{blue}i}}=1\Leftrightarrow \tau_{\texttt{\color{blue}i}}={\color{green}\alpha}$ and $[L_\tau]_{{\color{green}\alpha}\texttt{\color{blue}i}}=0$ otherwise, and $\forall i<|\tau|.\; [R_\tau]_{\texttt{\color{blue}i(i+1)}}=1 $ and $[R_\tau]_{\texttt{\color{blue}ij}}=0$ otherwise.
%Exploiting this encoding, we can adopt a simplified version of the embedding defined in \cite{LodhiSSCW02,Raedt} as $\embed_{\mathcal{T}}(\tau)_{{\color{green}\alpha\beta}}=\sum_{1\leq i\leq |\tau|}\lambda^i[(\Lambda_\tau)^i]_{\color{green}\alpha\beta}$.
%Please note that this definition is similar to a transition matrix embedding proposed in \cite{GartnerFW03} via geometric series, that is $\sum_i\lambda^i[R^i]_{\color{green}\alpha\beta}$.

\begin{figure}[!t]
	\centering
	\includegraphics[width=.4\textwidth]{images/taustar.pdf}
	\caption{Graphical representation of the transition graph encoding trace $\const{caba}$.}\label{fig:taustar}
\end{figure}
%
%\begin{example}\label{ex:tracembed}
%	{Let us suppose that we want to align a trace $\tau^*$ to one of the traces from a transition graph: in order to carry out an approximate alignment, we need to transform it to a transition graph first.} A trace $\tau^*=\textup{caba}$ can be graphically represented in Figure \ref{fig:taustar}. The associated TG $T=(\mathtt{\color{blue}1},\mathtt{\color{blue}4},L,R,1)$ has matrices $L$ and $R$  defined as follows:
%	$$L:=\kbordermatrix{
%		& \texttt{\color{blue}1}&\texttt{\color{blue}2}&\texttt{\color{blue}3}&\texttt{\color{blue}4}\\
%		\color{green}a            & 0&\textbf{1}&0&\textbf{1}\\
%		\color{green}b            & 0&0&\textbf{1}&0\\
%		\color{green}c            & \textbf{1}&0&0&0\\
%	}\qquad R:=\kbordermatrix{
%		& \texttt{\color{blue}1}&\texttt{\color{blue}2}&\texttt{\color{blue}3}&\texttt{\color{blue}4}\\
%		\texttt{\color{blue}1}  & 0&\color{red}1&0&0\\
%		\texttt{\color{blue}2}  & 0&0&\color{red}1&0\\
%		\texttt{\color{blue}3}  & 0&0&0&\color{red}1\\
%		\texttt{\color{blue}4}  & 0& 0& 0& 0\\
%	}$$
%We can similarly represent all the traces from the USPN.
%\end{example}

%\begin{example}
%The subtrace \textit{\textbf{\uline{hi}}} is represented in \textit{\textbf{\uline{hi}}deous},   \textit{\uline{\textbf{h}}e\uline{{i}}d\textbf{i}}, and \textit{\uline{{\textbf{h}i}}nd\textbf{i}}, but with different frequencies and subtrace distances. We have $\embed_{\mathcal{T}}(\textit{hideous})_{{\color{green}hi}}=\lambda$,  $\embed_{\mathcal{T}}(\textit{heidi})_{{\color{green}hi}}=\lambda^2+\lambda^4$, and $\embed_{\mathcal{T}}(\textit{hindi})_{{\color{green}hi}}=\lambda+\lambda^4$.
%\end{example}



\begin{table}[t!]
\vspace{+0.5cm}
\caption{Embedding of traces $\const{caba}$, $\const{caa}$ and $\const{cb}$.}\label{tb:embedding}
\vspace{-0.4cm}
\begin{center}
\scalebox{0.45}{
	\begin{tabularx}{\textwidth}{
>{\hsize=.1\hsize}X
>{\hsize=.2\hsize}X
>{\hsize=.1\hsize}X
>{\hsize=.1\hsize}X
>{\hsize=.1\hsize}X
>{\hsize=.1\hsize}X
>{\hsize=.1\hsize}X
>{\hsize=.25\hsize}X
>{\hsize=.2\hsize}X
>{\hsize=.1\hsize}X
}
		\toprule
		& $\const{aa}$    & $\const{ab}$   & $\const{ac}$    & $\const{ba}$   & $\const{bb}$   & $\const{bc}$ & $\const{ca}$ & $\const{cb}$ & $\const{cc}$   \\
		\midrule
		$\const{caba}$ & $\lambda^2$ & $\lambda$ & $0$ & $\lambda$  & $0$  & $0$ & $\lambda+\lambda^3$ & $\lambda^2$ & $0$\\
		%$\const{caaa}$ & $2\lambda+\lambda^2$& $0$ & $0$ & $0$ & $0$ & $0$ & $\lambda+\lambda^2+\lambda^3$ & $0$ & $0$ \\
		$\const{caa}$  & $\lambda$ & $0$ & $0$ & $0$ & $0$ & $0$ & $\lambda+\lambda^2$ & $0$&  $0$\\
		$\const{cb}$   & $0$ & $0$ & $0$ & $0$ & $0$ & $0$ & $0$ & $\lambda$& $0$ \\
		\bottomrule
	\end{tabularx}
}
\vspace{-0.8cm}
\end{center}
\end{table}
\begin{example}\label{ex:wheredotiszero} %\small
Consider tasks $\tasks=\Set{a,b,c}$. The possible 2-grams over $\tasks$ are $\tasks^2=\Set{\const{aa},\const{ab},\const{ac},\const{ba},\const{bb},\const{bc},\const{ca},\const{cb},\const{cc}}$. Table~\ref{tb:embedding} shows the embeddings of some traces. Being a 2-gram, trace $\const{cb}$ has only one nonzero component, namely that corresponding to itself, with $\trembed_{\const{cb}}(\const{cb})=\lambda$. Trace $\const{caa}$ has the 2-gram $\const{ca}$ occurring with length $1$ ($\const{\underline{ca}a}$) and $2$ ($\const{\underline{c}a\underline{a}}$), and the 2-gram $\const{aa}$ with occurring length $1$ ($\const{c\underline{aa}}$). Hence: $\trembed_{\const{ca}}(\const{caa})=\lambda+\lambda^2$ and  $\trembed_{\const{aa}}(\const{caa})=\lambda$.  Similar considerations can be carried out for the other traces in the table.
We now want to compute the similarity between the first trace $\const{caba}$ and the other two traces. To do so, we sum, column by column (that is, 2-gram by 2-gram) the product of the embeddings for each pair of traces. We then get $k_{\trembed}(\const{caba},\const{caa})=\lambda^3+(\lambda+\lambda^3)(\lambda+\lambda^2)$ and $k_{\trembed}(\const{caba},\const{cb})=\lambda^3
$,
%{\footnotesize
%\[
%k_{\trembed}(\const{caba},\const{caaa})=\lambda(\lambda+\lambda^2+\lambda^3)
%~~
%k_{\trembed}(\const{caba},\const{caa})=\lambda(\lambda+\lambda^2)
%~~
%k_{\trembed}(\const{caba},\const{cb})=\lambda(\lambda+\lambda^3)
%\]}
which induces ranking $
k_{\trembed}(\const{caba},\const{caa})>
k_{\trembed}(\const{caba},\const{cb})
$.
\end{example}

\endinput
\subsection{Graph Embedding}\label{ssec:ge}
Graph kernels allow mapping graph data structures to feature spaces (usually an Euclidean space in $\mathbb{R}^n$ for $n\in \mathbb{N}_{>0}$) \cite{Samatova} so to express graph similarity functions that can then be adopted for both classification \cite{TsudaS10} and clustering algorithms. One of the first approaches used in literature involved the definition of topological description vectors \cite{Sidere} for each graph in a graph database, for then defining the graph similarity function as an inner product of their associated vectors. One inconvenience of such a technique is that it is required to perform NP-complete subgraph isomorphisms among a collection of graphs. It has been already proved that the definition of a graph kernel function fully recognizing the structure the graph always boils down to solving such NP-Complete problem \cite{GartnerFW03}, as exact embeddings generable in polynomial can be inferred just for loop-free Direct Acyclic Graphs \cite{BergamiBM20}.


Consequently, most recent literature focused on extracting relevant features of such graphs, that are then used to define a graph similarity function. The most common approach adopted in the kernel to extract such features is called \textit{propositionalization}: we might extract all the possible features (e.g., subsequences), and then define a kernel function based on the occurrence and similarity of these features \cite{Gartner03}.

%\section{LTL over Finite Traces and the Declare Framework}
%\label{sec:preliminaries}
%As a formal basis for specifying crisp (temporal) business constraints, we adopt the customary choice of Linear Temporal Logic over finite traces (\LTLf \cite{DeVa13,DDGM14}). This logic is at the basis of the well-known \declare \cite{PeSV07} constraint-based process modeling language.
%We provide here a gentle introduction to this logic and to the \declare framework.
%
%\subsection{Linear Temporal Logic over Finite Traces}
%
%$\LTLf$ has exactly the same syntax as standard $\LTL$, but, differently from $\LTL$, it interprets formulae over an unbounded, yet finite linear sequence of states. Given an alphabet $\Sigma$ of atomic propositions (in our setting, representing activities), an \LTLf formula $\varphi$ is built by extending propositional logic with temporal operators:
%\[\varphi ::= a \mid \lnot \varphi \mid \varphi_1\lor \varphi_2
% \mid \Next\varphi \mid \varphi_1\Until\varphi_2 \quad \text{ where $a \in \Sigma$.}\]
%
%
%%The semantics of \LTLf is given in terms of \emph{finite traces}
%%denoting finite, \emph{possibly empty}, sequences
%%$\tau=\tau_0,\ldots,\tau_n$ of elements from the alphabet $\Sigma$. The evaluation of a formula is done in a given state (i.e., position) of the trace.
%
%
%The semantics of \LTLf is given in terms of \emph{finite traces} denoting finite, \emph{possibly empty} sequences $\tau=\tup{\tau_0, \ldots, \tau_n}$ of elements of $2^\Sigma$, containing all possible propositional interpretations of the propositional symbols in $\Sigma$. In the context of this paper, consistently with the literature on business process execution traces, we make the simplifying assumption that in each point of the sequence, one and only one element from $\Sigma$ holds. Under this assumption, $\tau$ becomes a total sequence of activity occurrences from $\Sigma$, matching the standard notion of (process) execution trace. We indicate with $\tasks^*$ the set of all traces over $\tasks$. The evaluation of a formula is done in a given state (i.e., position) of the trace, and we use the notation $\tau,i\models \varphi$ to express that $\varphi$ holds in the position $i$ of $\tau$. We also use $\tau \models \varphi$ as a shortcut notation for $\tau,0\models\varphi$. This denotes that $\varphi$ holds over the entire trace $\tau$ starting from the very beginning and, consequently, logically captures the notion of \emph{conformance} of $\tau$ against $\varphi$. We also say that $\varphi$ is \emph{satisfiable} if it admits at least one conforming trace.
%
%%We start by giving an intuitive account of the resulting semantics. In the syntax above, operator $\Next$ denotes the \emph{next state} operator, and $\Next \varphi$ is true if $\varphi$ is true is true now if there exists a next state (i.e., the current state is not at the end of the trace), and in the next state $\varphi$ holds. Operator $\Until$ instead is the \emph{until} operator, and $\varphi_1\Until\varphi_2$ is true if $\varphi_1$ holds now and continues to hold until eventually, in a future state, $\varphi_2$ holds. From the given syntax we can derive the usual boolean operators $\land$ and $\rightarrow$, the two formulae $\true$ and $\false$, as well also additional temporal operators. We consider in particular the following three:
%%\begin{compactitem}[$\bullet$]
%%\item (eventually) $\Diamond \varphi = \true \Until \varphi$ is true if there is a future state where $\varphi$ holds;
%%\item (globally) $\Box \varphi = \neg \Diamond \neg \varphi$ is true if now and in all future sates $\varphi$ holds;
%%\item (weak until) $\varphi_1 \Wntil \varphi_2 = \varphi_1\Until\varphi_2 \lor \Box \varphi_1$ relaxes the until operator by admitting the possibility that $\varphi_2$ never becomes true, in this case by requiring that is true if $\varphi_1$ holds now and in all future states.
%%\end{compactitem}
%% To define the semantics formally, we denote the length of trace $\tau$ as $\length(\tau) =  n+1$.
%
%
%In the syntax above, operator $\Next$ denotes the \emph{next state} operator, and $\Next \varphi$ is true if there exists a next state (i.e., the current state is not at the end of the trace), and in the next state $\varphi$ holds. Operator $\Until$ instead is the \emph{until} operator, and $\varphi_1\Until\varphi_2$ is true if $\varphi_1$ holds now and continues to hold until eventually, in a future state, $\varphi_2$ holds. From these operators, we can derive the usual boolean operators $\land$ and $\rightarrow$, the two formulae $\true$ and $\false$, as well as additional temporal operators. We consider, in particular, the following three:
%\begin{compactitem}[$\bullet$]
%\item (eventually) $\Diamond \varphi = \true \Until \varphi$ is true if there is a future state where $\varphi$ holds;
%\item (globally) $\Box \varphi = \neg \Diamond \neg \varphi$ is true if now and in all future states $\varphi$ holds;
%\item (weak until) $\varphi_1 \Wntil \varphi_2 = \varphi_1\Until\varphi_2 \lor \Box \varphi_1$ relaxes the until operator by admitting the possibility that $\varphi_2$ never becomes true, in this case by requiring that $\varphi_1$ holds now and in all future states.
%\end{compactitem}
%%We write $\tau \models \varphi$ as a shortcut notation for $\tau,0\models \varphi$, and say that formula $\varphi$ is \emph{satisfiable}, if there exists a trace $\tau$ such that $\tau \models \varphi$.
%
%\begin{example}
%The $\LTLf$ formula $\Box(\activity{accept} \rightarrow \Diamond\activity{pay})$ models that, whenever an order is accepted, then it is eventually paid. The structure of the formula follows what is called \emph{response template} in \declare.
%\end{example}
%
%%Every $\LTLf$ formula $\varphi$ can be translated into a corresponding standard finite-state automaton $\aut_\varphi$ that accepts all and only those finite traces that satisfy $\varphi$ \cite{DeVa13,DDGM14}. Although the complexity of reasoning with $\LTLf$ is the same as that of $\LTL$, finite-state automata are much easier to manipulate in comparison with B\"uchi automata, which are necessary when formulae are interpreted over infinite traces.
%
%\subsection{Declare}
%\begin{table}[t]
\caption{Some \declare templates, their textual and graphical representation, the corresponding \LTLf formalization and the \LTLf formula capturing their complement (i.e., their logical negation).
\label{tab:constraints}}
\centering
\begin{adjustbox}{width=0.9\textwidth,center}
\begin{tikzpicture}
  \matrix[  nodes={node distance=\nodedist,minimum height=6mm},
            rectangle, draw,
            nodes in empty cells,
            row sep=2mm,column sep=3mm,
            very thick,
            column 1/.style={anchor=west,font=\footnotesize},
            column 2/.style={anchor=west,xshift=1mm,font=\footnotesize},
            column 3/.style={anchor=west,xshift=1mm,font=\footnotesize},
            column 4/.style={anchor=west},
            >=latex,->,
          ] (declarematrix) {
    \node {\textsc{text}};
    &
    \node[yshift=.5mm] {\textsc{notation}};
  &
    \node {\textsc{\LTLf\ formula ($\varphi$)}};
    &
    \node[yshift=.5mm] {\textsc{complement ($\neg\varphi$)}};
    \\
    \node {
      \begin{tabular}{@{}l@{}}
      \constraint{existence($\mathit{a}$)} \\
      \end{tabular}
    };
    &
    \node[smalltask,xshift=1.5mm] (a) {$\mathit{a}$};
    \node[taskfg,above=-1mm of a,xshift=-.3mm]{\footnotesize $1..\ast$};
    &
    \node {$\Diamond {a}$};
    &
    \node {$\Box \neg {a}$};

    \\
     \node {
      \begin{tabular}{@{}l@{}}
        \constraint{absence2($\mathit{a}$)}\\
      \end{tabular}
    };
    &
    \node[smalltask,xshift=1.5mm] (a) {$\mathit{a}$};
    \node[taskfg,above=-1mm of a,xshift=-.3mm]{\footnotesize $0..1$};

    &
    \node {$\neg \Diamond ({a} \land \Next \Diamond {a})$};
    &
    \node {$\Diamond ({a} \land \Next \Diamond {a})$};
    \\
    \node {
      \begin{tabular}{@{}l@{}}
        \constraint{response($\mathit{a}$,$\mathit{b}$)}
      \end{tabular}
    };
    &
    \node[smalltask,xshift=1.5mm] (a) {$\mathit{a}$};
    \node[above=-1mm of a,xshift=-.3mm]{\footnotesize $~$};
    \node[smalltask,right=\taskdist of a] (b) {$\mathit{b}$};
    \path[response,very thick] (a) -- (b);
    &
    \node {$\Box ({a} \rightarrow \Diamond {b})$};
    &
    \node {$\Diamond ({a} \land \Box \neg {b})$};
    \\
    \node {
      \begin{tabular}{@{}l@{}}
        \constraint{precedence($\mathit{a}$,$\mathit{b}$)}\\
      \end{tabular}
    };
    &
    \node[smalltask,xshift=1.5mm] (a) {$\mathit{a}$};
    \node[smalltask,right=\taskdist of a] (b) {$\mathit{b}$};
    \path[precedence,very thick] (b) -- (a);
    &
    \node {$\neg {b} \Wuntil {a}$};
    &
    \node {$\neg {a} \Until {b}$};
    \\
    \node {
      \begin{tabular}{@{}l@{}}
        \constraint{not-coexistence($\mathit{a}$,$\mathit{a}$)}\\

      \end{tabular}
    };
    &
    \node[smalltask,xshift=1.5mm] (a) {$\mathit{a}$};
    \node[smalltask,right=\taskdist of a] (b) {$\mathit{b}$};
    \path[notcoexistence,very thick] (a) -- (b);
    &
    \node {$\neg(\Diamond {a} \land \Diamond {b})$};
    &
    \node {$\Diamond {a} \land \Diamond {b}$};
    \\
%    \node {
%      \begin{tabular}{@{}l@{}}
%        \constraint{neg-response(\activity{a},\activity{b})}\\
%        $\Box( \activity{a} \limp \neg \bigcirc\Diamond \activity{b})$
%      \end{tabular}
%    };
%    &
%    \node[smalltask,xshift=1.5mm] (a) {\activity{a}};
%    \node[smalltask,right=\taskdist of a] (b) {\activity{b}};
%    \path[negationresponse,very thick] (a) -- (b);
%    \\
    };
\end{tikzpicture}
\end{adjustbox}
\end{table} 
%\declare\ \cite{PeSV07} is a declarative process modeling language based on \LTLf. More specifically, a \declare model fixes a set of activities, and a set of constraints over such activities, formalized using \LTLf formulae. The overall model is then formalized as the conjunction of the \LTLf formulae of its constraints.
%
%Among all possible \LTLf formulae, \declare selects some pre-defined patterns. Each pattern is represented as a \declare template, i.e., a formula with placeholders to be substituted by concrete activities to obtain a constraint. Constraints and templates have a graphical representation; Table~\ref{tab:constraints} lists the \declare templates used in this paper. A \declare model is then graphically represented by showing its activities, and the application of templates to such activities (which indicates how the template placeholders have to be substituted to obtain the corresponding constraint).
%
%%Automata-based techniques for $\LTLf$ have been adopted to tackle fundamental tasks within the lifecycle of \declare processes, such as consistency checking \cite{PeSV07,MPVC11}, enactment and monitoring \cite{PeSV07,MMWV11,DDGM14}, and discovery support \cite{MaCV12}.
%
%
%
%
%\begin{example}
%\label{ex:inconsistency}
%Consider the following \declare model, constituting a (failed) attempt of capturing a fragment of an order-to-shipment process:
%
%\begin{center}
%  \resizebox{3.2cm}{!}{
%        \begin{tikzpicture}
%        \node[task] (accept) {\accept};
%        \node[task,right=of accept] (reject) {\reject};
%        \node[left=0mm of accept,taskfg] {1..*};
%        \node[right=0mm of reject,taskfg] {1..*};
%        \draw[notcoexistence] (accept) -- (reject);
%    \end{tikzpicture}
%  }
%\end{center}
%
%The model indicates that there are two activities to accept or reject an order, that these two activities are mutually exclusive, and that both of them have to be executed.
%%  \begin{wrapfigure}[13]{l}{42mm}
%%  \end{wrapfigure}
%These constraints are obviously contradictory and, in fact, the model is inconsistent, since its \LTLf formula
%$
%\Diamond \accept \land \Diamond \reject \land \neg (\Diamond \accept \land \Diamond \reject)
%$
%is unsatisfiable.
%\end{example}
%
%
%
%\endinput
%
%\smallskip\noindent\textbf{\declare} is a constraint-based process modeling language based on \LTLf. Differently from imperative process modeling languages,
%\declare models a process by fixing a set of activities, and defining a set of
%\emph{temporal constraints} over them, accepting every execution trace that satisfies all constraints.
%Constraints are specified via pre-defined \LTLf templates, which come with a corresponding
%graphical representation (see Table~\ref{tab:constraints} for the \declare patterns we use in this paper).
%For the sake of generality, in this paper we consider arbitrary \LTLf formulae as constraints. However, in the examples we consider formulae whose templates can be represented graphically in \declare.
%
%
%
%Automata-based techniques for $\LTLf$ have been adopted in \declare to tackle fundamental tasks within the lifecycle of Declare processes, such as consistency checking \cite{PeSV07,MPVC11}, enactment and monitoring \cite{PeSV07,MMWV11,DDGM14}, and discovery support \cite{MaCV12}.

% !TeX root=../main.tex
\begin{figure}[!t]
	\hspace*{-4cm}\includegraphics[width=1.7\textwidth]{images/pipeline}
	\caption{Proposed pipeline to assess the Probabilistic Trace Alignment}\label{fig:pipe}
\end{figure}


\section{Probabilistic Trace Alignment Pipeline}
We propose the pipeline from Figure \ref{fig:pipe} connecting several existing formalisations via intermediate processing steps. 
The input of this pipeline is a query trace to be aligned, an USWN, and a minimum probability threshold $p_\theta$. Its output 
is a set of model traces satisfying $p_\theta$ with an alignment ranking.
%
The pipeline has the following phases: after representing the USWN as a graph of all the sequentially scheduled transitions 
(\S\ref{sec:seqZ}), we shift the labels from the edges towards the nodes while preserving the set of probabilistic traces 
(\S\ref{sec:LSift}) and minimize the graph representation by removing the $\tau$-labelled nodes while preserving the 
trace probability (\S\ref{sec:clos}). We extract the set of all traces having probability above $p_\theta$ threshold (\S\ref{sec:unrav}) and  apply two different alignment strategies; one exact  (\S\ref{subsec:eta}), and one approximated. 


We later discuss how to rank traces in the exact and approximated scenarios by reducing the alignment process to a k-nearest 
neighbour problem. While the exact trace alignment requires to perform the alignment process each time a novel trace $\sigma^*$ is 
introduced (\S\ref{subsec:exbkptap}), the approximated alignment can split the alignment into a preliminary loading phase and a 
query phase. In the former, each stochastic trace from the USWN is represented as a vector (\S\ref{subsec:ate}), and in the latter the to-be-aligned trace $\sigma^*$ is first represented as a vector and then compared to all the other vectorial representations.  

\subsection{Sequentialization}\label{sec:seqZ}
The sequentialization step transforms a USWN with an initial marking $M$ into a Reachability Graph $(\mathcal{M},\mathcal{E})$ 
generated by a sequentialization process, where potentially concurrent firing transitions are represented via a sequential scheduling. 

\begin{definition}[Reachability Graph]
	Given an initial marking $M$ for a USWN $\mathcal{U}$,  the \textit{Reachability Graph} for $\mathcal{U}$ is a graph 
	$(\mathcal{M},\mathcal{E})$ where the nodes  $\mathcal{M}$ are composed of all the reachable markings from $M$, 
	and the edges $\mathcal{E}$ are induced by the aforementioned relation $M\overset{t}{\to}M'$ among the 
	nodes. To each edge $M\overset{t}{\to}M'$, we associate a transition probability $\mathbb{P}\left(M\overset{t}{\to}M'\right)=\frac{W(t)}{\sum_{t'\in E(M)}W(t')}$ \cite{spdwe}. 
\end{definition}

\begin{example}
From the USWN in Figure \ref{fig:spn}, the sequentialization process generates the reachability graph depicted in 
Figure \ref{fig:rg}. Each node represents a marking $M$ as a vector, and the edges are labelled with the firing transitions. 
The edges associated to this graph describe potentially concurrent firing transitions sequentially. While visiting the graph from 
$M$, the chaining of the edge labels generates a trace produced from the Untimed Workflow Net, and the product of the edge 
weights provides the probability associated to the trace.
\end{example}



\subsection{Label Shifter}\label{sec:LSift}
Reachability graphs obtained via sequentialization cannot be directly embedded using existing methods:  Reachability graphs 
associated to stochastic workflow nets are edge labelled, but TGs are node labelled. To represent the former as the latter, we 
shift the labels from the edges to the nodes  while preserving the set of traces and their associated probability. 
Such transformation is defined next.

\begin{definition}[Label Shifter]\label{def:transf}
The reachability graph $(\mathcal{M},\mathcal{E})$ generated from an initial marking $M$, is transformed into the TG $(s,t,L,R,1)$, where:
\begin{itemize}
	\item If is a single edge $M_1\overset{t}{\to}M_2\in\mathcal{E}$ where $M_1=M$, then $s=M\overset{t}{\to}M_2$; otherwise, define a new node $\textbf{i}$ and set it as the initial node for TG: $s=\textbf{i}$.
	\item If is a single edge $M_1\overset{t}{\to}M_2\in\mathcal{E}$ without outgoing edges in the reachability graph, then $t=M_1\overset{t}{\to}M_2$; otherwise, define a new node $\textbf{f}$ and set it as the accepting node for TG:  $t=\textbf{f}$.
	\item $[L]_{\lambda(t),\;M\overset{t}{\to} M'}=1$ for each $M\overset{t}{\to} M'\in\mathcal{E}$; if $\textbf{i}$ is defined then $[L]_{\tau\textbf{i}}=1$; if $\textbf{f}$ is defined, then $[L]_{\tau\textbf{f}}=1$; $[L]_{ij}=0$ otherwise.
	\item $[R]_{M\overset{t}{\to} M',\;M'\overset{t'}{\to} M''}=\frac{W(t')}{\sum_{\textbf{t}\in E(M')}W(\textbf{t})}$ for each $M\overset{t}{\to} M',M'\overset{t'}{\to} M''\in\mathcal{E}$; if $\textbf{i}$ is defined, $[R]_{\textbf{i},\;M\overset{t}{\to}M'}=\frac{W(t)}{\sum_{\textbf{t}\in E(M)}W(\textbf{t})}$; if $\textbf{f}$ is defined, then $[R]_{M\overset{t}{\to}M',\;\textbf{i}}=1$ for each $M'$ without outgoing edges in the reachability graph; $[R]_{ij}=0$ ow.
\end{itemize}
\end{definition}

\endinput
%
We can show that the TG obtained in Definition \ref{def:transf} preserves the same set of probabilistic traces associated by the reachability graph. The proof is omitted due to the lack of space.

\begin{example}
Figure \ref{fig:lmc} shows the TG obtained from the reachability graph in Figure \ref{fig:rg}. Nodes are labelled with the firing 
transition labels (in green), and edges preserve the probabilistic information from the reachability graph (in red). Intuitively, when a 
new initial node \textit{\textbf{i}} is inserted, we preserve all the initial probabilistic choices that a transition is fired from an initial 
marking $M$, while the intermediate edges inherit the probabilisitc choice of the firing transition from the subsequent choices. When 
a new final node \textit{\textbf{f}} is added, such edges always have probability $1$, and thus do not interfere with the 
initial traces' probability.
\end{example}

\subsection{$\tau$-closure}\label{sec:clos}
The $\tau$-closure process has two main purposes: first, reduce the size of the previously generated TG by removing all 
$\tau$-labelled nodes \texttt{\color{blue}w} and preserving the connection between  the nodes \texttt{\color{blue}u} 
from its ingoing edges   $\texttt{\color{blue}u}\xrightarrow{\color{violet}\rho}\texttt{\color{blue}w}$ with the nodes \texttt{\color{blue}v} from its ingoing edges   $\texttt{\color{blue}w}\xrightarrow{\color{violet}\rho'}\texttt{\color{blue}v}$ by establishing new edges $\texttt{\color{blue}u}\xrightarrow{\color{violet}\rho\rho'}\texttt{\color{blue}v}$. $\tau$-labelled initial (or accepting) nodes are removed iff they have only one outgoing (ingoing) edge with probability $1$.

\begin{example}\todo{Is it now ok?}
	The $\tau$-closure removes the non-initial and non-accepting nodes within an automaton, while preserving the probabilistic trace equivalence of the two automata. Let us suppose to apply the $\tau$-closure to the automata in Figure \ref{fig:orig}: node \texttt{\color{blue}10} is removed alongside its associated edges, and new edges $\texttt{\color{blue}3}\xrightarrow{\color{violet}\rho_{65}}\texttt{\color{blue}4}$ and $\texttt{\color{blue}3}\xrightarrow{\color{violet}\rho_{6f}}\texttt{\color{blue}5}$ are introduced. The resulting TG $P$ is represented with the same graphical depiction Figure \ref{fig:closed}.
\end{example}
%
Consequently, it is always possible to minimize a TG  via $\tau$-closure, so that the only nodes labelled as $\tau$ 
are the source and the target nodes and the set of weighted traces is preserved. From now on, we consider only minimised TGs. 

\subsection{Unraveller}\label{sec:unrav}
%Being that both the graph isomorphism problem is NP-Complete and the 
Since TGs are fully characterized by the set of the probabilistic traces that they generate,  we say that two TGs are 
(probabilistic-trace) equivalent iff they share the same set of weighted traces. In particular, we denote as $\mathcal{W}_p^n(P)$ the set of all the weighted traces in $P$ having at least probability $p$ and maximum length $n$. Under these assumptions, the probabilistic trace equivalence is deterministic.

\begin{example}
	The set $\mathcal{W}_0^{\aleph_0}(P^*)$ of weighted traces of the TG in Figure \ref{fig:orig} is
%	The TG in Figure \ref{fig:orig} has the following set $\mathcal{W}_0^{\aleph_0}(P^*)$ of weighted traces:
$$\set{\braket{\underbrace{\color{green}a\dots a}_{n},{\color{violet}\pa\pc^n\pf}}|n\in \mathbb{N}_{>0}}\cup \set{\braket{{\color{green}c}\underbrace{\color{green}a\dots a}_{n},{\color{violet}\pb\pd\pc^n\pf}}|n\in \mathbb{N}_{>0}}\cup\{\braket{{\color{green}cb},{\color{violet}\pb\pe}}\}$$
After the $\tau$-closure process, $\mathcal{W}_0^{\aleph_0}(P^*)=\mathcal{W}_0^{\aleph_0}(P)$, so the two TGs are (probabilistic-trace) equivalent.
\end{example}

%% !TEX root =  paper.tex

\newcommand{\stoclang}{\rho}
\newcommand{\evlog}{\mathcal{L}}
\newcommand{\nop}{\activity{nop}}
\newcommand{\cancel}{\activity{cancel}}
\newcommand{\pay}{\activity{pay}}
\newcommand{\ship}{\activity{ship}}

\newcommand{\model}{\mathcal{M}}
\newcommand{\cset}{\mathcal{C}}
\newcommand{\crispc}{\cset_{\mathit{crisp}}}
\newcommand{\probc}{\cset_{\mathit{prob}}}

\section{Probabilistic Business Constraints}
\label{sec:probdeclare}

As recalled in Section~\ref{sec:preliminaries}, business constraints captured with \LTLf are interpreted in a crisp way, i.e., they are expected to hold in \emph{every} execution of the process. We now extend constraints with a natural notion of uncertainty introducing probabilistic constraints. Then, we show how this notion can be used to make \declare probabilistic and discuss informally the interplay of multiple probabilistic constraints. %This informal discussion is then mirrored into a formal counterpart in Section~\ref{sec:scenarios}.
%: we quantify what is the probability that a given execution trace of the process will satisfy the constraint. After having introduced the resulting notion

\subsection{Probabilistic Constraints: Definition and Semantics}
For simplicity, we only consider the case of \emph{exact} probability, but all the considerations we do directly carry over the more general case where the probability of a constraint is related to a given quantity with comparison operators ($\leq$, $<$, $=$, and their duals).

\begin{definition}
  A \emph{probabilistic constraint} over a set $\tasks$ of activities is a pair $\tup{\varphi,p}$, where $\varphi$ is an \LTLf formula over $\tasks$ representing the \emph{constraint formula}, and  $p$ is a rational value in $[0,1]$ representing the \emph{constraint probability}.
\end{definition}

Since a probabilistic constraint quantifies \emph{how many} traces should satisfy it, it has to be interpreted over multiple traces that, as a whole, constitute an event log for the process of interest. In particular, the \emph{constraint holds in a log if the ratio of traces in the log that satisfy the constraint formula is equal to the constraint probability}. This naturally leads to interpret the constraint probability statistically as the ratio of conforming vs non-conforming traces contained in a given log.

For simplicity , we stick here with the standard definition of event log, but we could alternatively adopt the stochastic interpretation of an event log, following  \cite{DBLP:conf/bpm/LeemansSA19}.

\begin{definition}
  An \emph{(event) log} over a set $\tasks$ of activities is a multiset of traces over $\tasks$, i.e., a multiset over $\tasks^*$.
\end{definition}
Given a log $\evlog$, we write $\tau^n \in \log$ to indicate that trace $\tau$ appears $n$ times in $\evlog$. Trace $\tau$ belongs to $\evlog$ if $\tau^n \in \log$ with $n > 0$. With these notions at hand, we say that a probabilistic constraint $\tup{\varphi,p}$ holds in a log $\evlog$ or, equivalently, that $\evlog$ satisfies $\tup{\varphi,p}$, if $\sum_{\tau^n \in \evlog, \tau \models \varphi} n = p$.

Note that the probabilistic constraint $\tup{\varphi,p}$ is equivalent to the probabilistic constraint $\tup{\neg \varphi,1-p}$. In fact, given a log $\evlog$, if the ratio of traces in $\evlog$ that satisfies $\varphi$ is $p$, then the remaining $1-p$ traces in $\evlog$ do not satisfy $\varphi$, i.e., they satisfy the constraint complement $\neg \varphi$.

\begin{example}
  Consider the probabilistic constraint $\tup{\constraint{existence}(\accept),0.8}$. It indicates that $80\%$ of the traces in a log contain at least one occurrence of $\accept$ or, equivalently, that $20\%$ of the traces do not contain any execution of $\accept$. This constraint holds in the log:
  $
    \left[
      \begin{array}{l}
        \tup{\accept}^2,
        \tup{\accept,\reject},
        \tup{\reject}^4,
        \tup{\accept,\cancel}^3,
        \tup{\accept,\pay,\ship}^{10}
      \end{array}
    \right]$,~since $16$ out of the $20$ traces contained therein include (at least) one occurrence of $\accept$, i.e., they satisfy $\constraint{existence} = \Diamond \accept$.
\end{example}

\subsection{ProbDeclare and the Issue of Multiple Interacting Constraints}
We now use the notion of probabilistic constraint as the basic building block to lift \declare to its probabilistic version, which we call \pdeclare.

\begin{definition}
A \pdeclare model is a pair $\tup{\Sigma,\mathcal{C}}$, where $\Sigma$ is a set of activities and $\mathcal{C}$ is a set of probabilistic constraints.
\end{definition}

A standard \declare model corresponds to a \pdeclare model where all probabilistic constraints have probability $1$. In the remainder of the paper, when drawing \pdeclare diagrams, we then adopt the following notation:
\begin{inparaenum}[\it (i)]
  \item whenever a constraint has probability $1$, we draw it as a standard \declare constraint;
  \item Whenever a constraint has probability $p<1$, we show it in light blue, and we annotate it with the probability value $p$.
\end{inparaenum}

The main issue that arises when multiple, genuinely probabilistic constraints are present in the same \pdeclare model is that they interact with each other depending on their constraint formulae and probabilities. In particular, to satisfy the probabilistic constraints contained in a \pdeclare model, a log must contain suitable fractions of traces so as to satisfy \emph{all} probabilistic constraints and their probabilities, with the effect that some of these traces may contribute to the computation of the ratios for different constraints.
%
The following examples intuitively illustrate this interplay. The first example shows that inconsistent \declare models may become consistent if the conflicting constraints are associated with suitable probabilities.

\begin{example}
\label{ex:consistent-prob}
Consider the following probabilistic variant of the (inconsistent) \declare diagram shown in Example~\ref{ex:inconsistency}.
\begin{center}
  \resizebox{3.8cm}{!}{
        \begin{tikzpicture}
        \node[task] (accept) {\accept};
        \node[task,right=of accept] (reject) {\reject};
        \node[pwords,left=0mm of accept] {1..*$_{\{0.8\}}$};
        \node[pwords,right=0mm of reject] {1..*$_{\{0.1\}}$};
        \draw[notcoexistence] (accept) -- (reject);
    \end{tikzpicture}
  }
\end{center}
This model contains two mutually exclusive activities, $\accept$ and $\reject$, and indicates that often (in $80\%$ of the cases) $\accept$ is selected, whereas rarely (in $10\%$ of the cases) $\reject$ is selected. This captures a form of probabilistic choice, which also implicitly contemplates that none of the two activities occurs. In fact, from this very simple model, we can infer the following conditions on satisfying logs:
\begin{compactenum}
\item The $\constraint{not-coexistence}$ constraint linking $\accept$ and $\reject$ is crisp, and consequently no trace in the log can contain both $\accept$ and $\reject$.
\item Point 1, combined with the probabilistic $\constraint{existence}$ constraint on $\accept$, means that a trace in the log has $0.8$ probability of containing $\accept$ (which means that $\reject$ will not occur), and $0.2$ probability of not containing $\accept$ (which means that $\reject$ may occur or not).
\item A similar line of reasoning can be applied to the existence of $\reject$, which must appear in $10\%$ of the traces in the log.
\end{compactenum}
All in all, combining all the constraints, we get that the $10\%$ of traces containing $\reject$ must be disjoint from the $80\%$ containing $\accept$. This implicitly means that in the remaining $10\%$ of the traces, none of the two activities occur.
\end{example}

The second example shows that a consistent \pdeclare model may become inconsistent by changing the values of probabilities.

\begin{example}
  \label{ex:inconsistent-prob}
  Consider again the \pdeclare diagram in Example~\ref{ex:consistent-prob}. Clearly, if we change to $1$ the constraint probabilities attached to the two $\constraint{existence}$ constraints, the model becomes identical to that in Example~\ref{ex:inconsistency}, consequently becoming inconsistent. More in general, the model becomes inconsistent whenever the sum of the two probabilities exceeds $1$. This witnesses that there must exist some traces in which both constraints are satisfied, which contradicts the fact that $\accept$ and $\reject$ should not coexist. More precisely, if we denote by $p_a$ and $p_r$ the probabilities attached to the two $\constraint{existence}$  constraints, then there is a probability $p_a+p_r-1$ of having a trace that contains both $\accept$ and $\reject$.  For example, if we set $p_a = 0.8$ and $p_r = 0.3$, we have that $10\%$ of the traces in the log should contain both $\accept$ and $\reject$, which is impossible given the fact that every trace in the log should satisfy $\constraint{not-coexistence}(\accept,\reject)$.
\end{example}

The last example shows that, as customary in models with uncertainty, it is misleading to just consider the probabilities attached to single constraints when one wants to assess the probability of satisfying all of them at once.

\begin{example}
  \label{ex:prob-interplay}
  Consider the following \pdeclare model:
  \begin{center}
  \resizebox{3.5cm}{!}{
    \begin{tikzpicture}[node distance=2cm]
      \node[task] (accept) {\accept};
      \node[pwords,left=0mm of accept,yshift=+3mm] {1..*$_{\{0.8\}}$};
      \node[left=0mm of accept,taskfg,yshift=-3mm] {0..1};

      \node[task,right=of accept] (pay) {\pay};
      \draw[response,pconstraint]
        ($(accept.east)+(0,2mm)$)
        -- node[above,pwords] {$_{\set{0.7}}$}
        ($(pay.west)+(0,2mm)$);
      \draw[precedence]
        ($(pay.west)-(0,2mm)$)
        --
        ($(accept.east)-(0,2mm)$);

    \end{tikzpicture}
  }
  \end{center}
  The model indicates that an order can be accepted at most once, and that often (in $80\%$ of the cases) it is actually accepted. In addition, it captures that with probability $0.7$ it is true that, whenever the order is accepted, then it is also consequently paid (multiple payment instalments are possible, by simply repeating the execution of $\pay$). Finally, payments are enabled only if the order has been previously accepted.

  By looking at the diagram, one could wrongly interpret that in $70\%$ of the cases it is true that the order is accepted and then paid. This is wrong because the $\constraint{response}(\accept,\pay)$ constraint can also be (vacuously) satisfied by a trace that does not contain at all occurrences of $\accept$. A natural question is then: what is the actual probability of observing traces that at some point contain $\accept$ and, later on, $\pay$ (possibly with other activity occurrences in between and afterward)? The answer is that this happens in half of the cases. To justify this non-trivial answer, one has to apply combined reasoning by considering the interplay of $\constraint{response}(\accept,\pay)$ and $\constraint{existence}(\accept)$, with their corresponding probabilities. More specifically, $\constraint{response}(\accept,\pay)$ can be satisfied in this model in two different ways:
  \begin{compactenum}
  \item by not executing at all $\accept$;
  \item by executing $\accept$ (which can be done only once, due to the presence of the crisp $\constraint{absence2}(\accept)$ constraint) and, later on, at least once $\pay$.
  \end{compactenum}
  These two situations, which we will call later on \emph{constraint scenarios}, should altogether cover exactly $70\%$ of the traces, as dictated by the constraint probability attached to $\constraint{response}(\accept,\pay)$. The first scenario must have probability $0.2$, because in $80\%$ of the traces $\accept$ must appear, as dictated by the $\constraint{existence}(\accept)$ constraint and its associated probability.
    But then, the second scenario, which is the one we are interested in, has probability $0.7-0.2 = 0.5$ (half of the traces in the log).
\end{example}

In the next section, we make the reasoning carried out in the discussed examples more systematic, showing how logical and probabilistic reasoning have to be combined towards a single, combined declarative framework.

%
\section{Reasoning on Time and Probabilities}
As we have seen in the previous section, to reason on conjunctions of probabilistic constraints, i.e., on \pdeclare models, we need to simultaneously take into account the temporal semantics of constraints and their associated probabilities.

Formally, this is done by relying on the probabilistic temporal logic over finite traces \PLTL, recently introduced in \cite{MaMP20}. More specifically, probabilistic constraints as defined here have a direct encoding into the fragment \PLTLz of \PLTL, also investigated in \cite{MaMP20}. We do not delve into the encoding, nor highlight the formal details on how to carry out this combined reasoning. We instead show algorithmically how to accomplish this, noticing that all the algorithmic techniques discussed next are correct thanks to \cite{MaMP20}. Again thanks to \cite{MaMP20}, we also get that, overall, the cost of reasoning on probabilistic constraints has the same complexity of reasoning with standard \LTLf constraints, i.e., \PS in the length of the constraints (this complexity bound is tight).


In the remainder of this section, we fix a \pdeclare model $\model = \tup{\tasks,\cset}$, where $\cset$ is partitioned into \emph{crisp constraints} $\crispc = \set{\tup{\varphi,p} \in \cset \mid p = 1}$ and (genuinely) \emph{probabilistic constraints} $\probc = \set{\tup{\varphi,p} \in \cset \mid p < 1}$. With a slight abuse of terminology, when we use the term ``crisp constraint'', we mean a constraint in $\crispc$, and, when we use the term  ``probabilistic constraint'', we mean a constraint in $\probc$. We also assume that $\crispc$ is a consistent \declare model, i.e., crisp constraints are satisfiable altogether. If not, then $\model$ has to be discarded, as it does not admit any conforming trace.


\subsection{Constraint Scenarios and Consistency of \pdeclare Models}
While crisp constraints must hold in every possible trace, probabilistic constraints may or may not hold (with a ratio specified by their probability). In addition, recall that when a constraint formula does not hold, then its negation must hold. Consequently, in the most general case, $\model$ is a compact description for the $2^{|\probc|}$ standard \declare models, each one obtained by considering all constraint formulae in $\crispc$, and by selecting, for each constraint $\tup{\varphi,p} \in \probc$, whether the constraint formula $\varphi$ or its complement $\neg \varphi$ is assumed to hold.

We call the so-obtained \declare models \emph{(constraint) scenarios}. To pinpoint a specific scenario, we fix an ordering over $\probc$, and we denote the scenario with a binary string of length $|\probc|$, where position number $i \in \set{1,\ldots,|\probc|}$ has value $1$ if the $i$-th probabilistic constraint in $\probc$ must hold, $0$ otherwise.

\begin{example}
  \label{ex:scenarios}
  Consider the \pdeclare model in Example~\ref{ex:prob-interplay}. By fixing the ordering over its probabilistic constraints where $\tup{\constraint{existence}(\accept),0.8}$ is first and $\tup{\constraint{response}(\accept,\pay),0.7}$ is second, we have the following $4$ scenarios:
  \begin{compactenum}
  \item Scenario $00$, where none of the two constraint formulae holds, and is consequently characterized by formula
  $\Box \neg \accept \land \Diamond (\accept \land \Box \neg \pay)$.
    \item Scenario $01$, where the $\constraint{response}$ constraint formula holds while the $\constraint{existence}$ one does not, and so has formula
  $\Box \neg \accept \land \Box (\accept \rightarrow \Diamond \pay)$.
    \item Scenario $10$, where the $\constraint{existence}$ constraint formula holds while the $\constraint{response}$ one does not, and so has formula
  $\Diamond \accept \land \Diamond (\accept \land \Box \neg \pay)$.
     \item Scenario $11$, where both formulae holds ($\Diamond \accept \land \Box (\accept \rightarrow \Diamond \pay)$).
  \end{compactenum}
\end{example}

Among the possible scenarios, only those that are logically consistent, i.e., are associated with a satisfiable formula, have to be retained. In fact, inconsistent scenarios do not admit any conforming trace. Obviously, when checking whether the scenario is consistent, its constraint formulae have to be conjoined with those in  $\crispc$.

\begin{example}
\label{ex:consistent-scenarios}
Consider the $4$ scenarios of Example~\ref{ex:scenarios}. Scenario $00$ has to be discarded because it is logically inconsistent: its formula $\Box \neg \accept \land \Diamond (\accept \land \Diamond \pay)$ is unsatisfiable (it is asking for the presence and absence of $\accept$). The other three scenarios are instead logically consistent.
\end{example}

\begin{example}
\label{ex:consistent-scenarios-2}
Consider the \pdeclare model in Example~\ref{ex:consistent-prob}. Also for this model there are $4$ scenarios, obtained by considering the two $\constraint{existence}$ constraints and their complements. The scenario where both constraints are not satisfied captures those traces where no decision is taken for the order, i.e., the order is not accepted nor rejected. The scenarios where one constraint is satisfied and the other is not account for those traces where a univocal decision is taken for the order. The scenario where both constraints are satisfied, thus requiring acceptance and rejection for the order, is inconsistent, due to the interplay of such constraints and the crisp $\constraint{not-coexistence}$ one. This corresponds to the standard \declare model of Example~\ref{ex:inconsistency}.
\end{example}

We have explicitly used the term \emph{logically} (in)consistent scenarios since there is no guarantee that these scenarios are actually plausible. This depends on their corresponding probabilities, which, in turn, are obtained by suitably combining the probabilities of their constitutive constraints in their positive or complemented form. This is done by enforcing the semantics of constraint probability, which requires to ensure the following: for every probabilistic constraint $\tup{\varphi,p}$, the sum of the probabilities assigned to those scenarios where $\varphi$ must hold must be equal to $p$.

To do so, we construct a system of linear inequalities whose variables represent the probabilities of possible scenarios \cite{MaMP20}. We denote such variables as $x_s$, where $s$ is the boolean string representing the scenario the variable is associated with. By considering a \pdeclare model $\model$, fixing $n = |\probc|$ and writing $i \in \set{0,\ldots,n-1}$ in binary, the system of inequalities $\Lmc_\model$ is:
\begin{align*}
x_i  &\geq  0  && 0\leq i < 2^n
  \tag{\text{$x_i$ are probabilities}} \\
 \sum_{i=0}^{2^n-1}x_i &= 1
  \tag{\text{$x_i$ are probabilities}}\\[2pt]
 \sum_{\text{$j$th position is 1}}x_i &= p_j &&  0\le j <n
  \tag{\text{constraint semantics}}\\
x_i &= 0 &&  \text{if scenario $i$ is logically inconsistent}
\end{align*}
Notably, $\Lmc_\model$ combines, at once, the logical and the probabilistic content of $\model$, on the one hand, imposing that the scenario probabilities agree with the constraint probabilities, and, on the other, forcing logically inconsistent scenario to have probability $0$.

 $\Lmc_\model$ may admit:
\begin{inparaenum}[\it (i)]
\item \emph{no solution}, witnessing that $\M$ is inconsistent;
\item \emph{one solution}, returning the exact probabilities for all the scenarios of $\M$,
\item \emph{multiple (possibly infinitely many) solutions}, witnessing that different probability distributions can be assigned to the scenarios.
\end{inparaenum} To obtain the ranges of probability for each scenario, one can turn the system of inequality into several optimizations problems where each probability variable is minimized and maximized.

It is worth noting that, when $\Lmc_\model$ is solvable, its solutions may force some scenario probabilities to be always equal to $0$. This witnesses the fact that even a logically consistent scenario may not have any conforming trace due to the interplay of constraint probabilities. We call \emph{plausible} those scenarios that have a probability $>0$.

\begin{example}
  Consider again the \pdeclare model in Example~\ref{ex:scenarios} with its 4 scenarios (one of which is logically inconsistent, as discussed in Example~\ref{ex:consistent-scenarios}). The four possible scenarios have corresponding probability variables $x_{00}$,  $x_{01}$, $x_{10}$ and $x_{11}$, constrained by the system of inequalities (we omit the fact that all variables are non-negative):
  $$
  \begin{array}{lclclclcl@{\qquad}l}
  x_{00} &+& x_{01} &+& x_{10} &+& x_{11} &=& 1 &
  \\
         & &        & & x_{10} &+& x_{11} &=& 0.8 &
  \text{semantics of $\tup{\constraint{existence}(\accept),0.8}$}
  \\
         & & x_{01} & &  &+&  x_{11}      &=& 0.7 &
  \text{semantics of $\tup{\constraint{response}(\accept,\pay),0.7}$}
  \\
  x_{00} & &       & &        & &         &=& 0 &
  \text{logical inconsistency of scenario $00$}
  \end{array}
  $$
The system admits a single solution, with $x_{00} = 0$, $x_{01} = 0.2$, $x_{10} = 0.3$ and $x_{11} = 0.5$, the last matching the informal discussion given in Example~\ref{ex:prob-interplay}.
\end{example}

We conclude the section with an informative \pdeclare model example that combines parts of the examples seen so far to capture a non-trivial fragment of an order-to-shipment process. We use parameters for constraint probabilities, then discussing the impact of grounding such probabilities to different actual values.

\begin{table}[t]
\caption{\label{tab:scenarios} Constraint scenarios of the \pdeclare model in Example~\ref{ex:running}, indicating whether they are logically consistent and, if so, providing the (shortest) conforming trace, and the scenario probability.}
{\scriptsize
\begin{tabularx}{\textwidth}{|C@{~~}C@{~~}C@{~~}C|c|l||c|}
\hline
\multicolumn{4}{|c|}{\textsc{scenario}} &
\textsc{logically} &
\multicolumn{1}{c||}{\textsc{shortest conforming}} &
\textsc{scenario}
\\
$\varphi_a$ &
$\varphi_r$ &
$\varphi_{ap}$ &
$\varphi_{ax}$ &
\textsc{consistent} &
\multicolumn{1}{c||}{\textsc{trace}} &
\textsc{probability}
\\
\hline
0 & 0 & 0 & 0 & N & & \\
\hline
0 & 0 & 0 & 1 & N &  & \\
\hline
0 & 0 & 1 & 0 & N &  & \\
\hline
0 & 0 & 1 & 1 & Y & empty trace & $1-p_a-p_r$\\
\hline
0 & 1 & 0 & 0 & N &  & \\
\hline
0 & 1 & 0 & 1 & N &  & \\
\hline
0 & 1 & 1 & 0 & N &  & \\
\hline
0 & 1 & 1 & 1 & Y & $\tup{\reject}$ & $p_r$\\
\hline
1 & 0 & 0 & 0 & Y & $\tup{\accept}$ & $2-p_a-p_{ap}-p_{ax}$\\
\hline
1 & 0 & 0 & 1 & Y & $\tup{\accept,\cancel}$ & $p_a+p_{ax}-1$\\
\hline
1 & 0 & 1 & 0 & Y & $\tup{\accept,\pay,\ship}$ & $p_a+p_{ap}-1$\\
\hline
1 & 0 & 1 & 1 & N &  & \\
\hline
1 & 1 & 0 & 0 & N &  & \\
\hline
1 & 1 & 0 & 1 & N &  & \\
\hline
1 & 1 & 1 & 0 & N &  & \\
\hline
1 & 1 & 1 & 1 & N &  & \\
\hline
\end{tabularx}
}


\end{table}

\begin{example}
  \label{ex:running}
Consider the following order-to-shipment \pdeclare  model:
    \begin{center}
  \resizebox{7cm}{!}{
    \begin{tikzpicture}[node distance=2cm]
      \node[task] (accept) {\accept};
      \node[pwords,above=0mm of accept] {1..*$_{\{p_a\}}$};
      \node[below=0mm of accept,taskfg] {0..1};

      \node[task,left = of accept] (reject) {\reject};
      \node[pwords,above=0mm of reject] {1..*$_{\{p_r\}}$};

      \draw[notcoexistence] (reject) -- (accept);


      \node[task,right=of accept] (pay) {\pay};

      \draw[response,pconstraint]
        ($(accept.east)+(0,2mm)$)
        -- node[above,pwords] {$_{\set{p_{ap}}}$}
        ($(pay.west)+(0,2mm)$);

      \draw[precedence]
        ($(pay.west)-(0,2mm)$)
        --
        ($(accept.east)-(0,2mm)$);



      \node[task,below=1cm of pay] (cancel) {\cancel};

      \draw[precedence,fill=white]
        ($(cancel.west)-(0,1.5mm)$)
        -|
        ($(accept.south east)-(3mm,0)$);

      \draw[response,pconstraint,fill=white]
        ($(accept.south east)-(1mm,0)$)
        |- node[above right,pwords] {$_{\set{p_{ax}}}$}
        ($(cancel.west)+(0,1.5mm)$)
        ;

      \draw[notcoexistence] (pay) -- (cancel);

       \node[task,right=of pay] (ship) {\ship};
       \draw[precedence]
        ($(ship.west)-(0,2mm)$)
        --
        ($(pay.east)-(0,2mm)$);
        \draw[response]
        ($(pay.east)+(0,2mm)$)
        --
        ($(ship.west)+(0,2mm)$);


    \end{tikzpicture}
  }
  \end{center}

To construct the $16$ possible scenarios for this model, the following constraints and \LTLf formulae have to be considered:
\begin{compactitem}[$\bullet$]
\item $\constraint{existence}(\accept)$ with formula $\varphi_a = \Diamond \accept$, and its complement $\Box \neg \accept$;
\item $\constraint{existence}(\reject)$ with formula $\varphi_r = \Diamond \reject$, and its complement $\Box \neg \reject$;
\item  $\constraint{response}(\accept,\pay)$ with formula $\varphi_{ap} = \Box( \accept \rightarrow \Diamond \pay)$,  and its complement $\Diamond(\accept \land \Box \neg \pay)$;
\item $\constraint{response}(\accept,\cancel)$ with formula $\varphi_{ax} = \Box( \accept \rightarrow \Diamond \cancel)$, and its complement $\Diamond(\accept \land \Box \neg \cancel)$.
\end{compactitem}

Table~\ref{tab:scenarios} summarizes the different constraint scenarios, their logical consistency and, in the last column, their probabilities computed by constructing and solving the system of inequalities described above.
Table~\ref{tab:cases} shows instead three different groundings for the constraint probability parameters and their impact on the probabilities of the scenarios. In particular, \emph{Case 1} is so that all the logically consistent scenarios may actually occur, even though with different probabilities. The most likely scenario, accounting for half of the traces, captures the happy path where the order is paid and shipped.
\emph{Case 2} assigns a different probability to $\constraint{response}(\accept,\cancel)$, causing scenario $1000$ to be not plausible anymore, being associated with probability $0$; intuitively, the interplay of constraints and their probabilities makes it impossible to just execute $\accept$ without taking further activities.
Finally, \emph{Case 3} increases the probability of $\constraint{response}(\accept,\cancel)$ even more, resulting in an inconsistent model.





\end{example}

\begin{table}[t]
\caption{\label{tab:cases} Three different groundings for the constraint probabilities used in the \pdeclare model in Example~\ref{ex:running}, and their impact on the scenario probabilities.}
{
\scriptsize
\begin{tabularx}{\textwidth}{|C@{~~}C@{~~}C@{~~}C|C|C|C|}
\hline
\multicolumn{4}{|c|}{\textsc{consistent scenario}} &
\textsc{case1} &
\textsc{case2} &
\textsc{case3}
\\
$\varphi_a$ &
$\varphi_r$ &
$\varphi_{ap}$ &
$\varphi_{ax}$ &
\scriptsize{
  $\begin{array}{@{}r@{~}l@{}}
      p_a =& 0.8\\
      p_r =& 0.1\\
      p_{ap} =& 0.7\\
      p_{ax} =& 0.3
  \end{array}$
  }
&
\scriptsize{
  $\begin{array}{@{}r@{~}l@{}}
      p_a =& 0.8\\
      p_r =& 0.1\\
      p_{ap} =& 0.7\\
      p_{ax} =& 0.5
    \end{array}$
  }
 &
\scriptsize{
  $\begin{array}{@{}r@{~}l@{}}
      p_a =& 0.8\\
      p_r =& 0.1\\
      p_{ap} =& 0.7\\
      p_{ax} =& 0.7
    \end{array}$
  }\\
\hline
0 & 0 & 1 & 1 & 0.1 & 0.1 &  \\
\cline{1-6}
0 & 1 & 1 & 1 & 0.1 & 0.1 & \\
\cline{1-6}
1 & 0 & 0 & 0 & 0.2 & 0 & inconsistent\\
\cline{1-6}
1 & 0 & 0 & 1 & 0.1 & 0.3 & \\
\cline{1-6}
1 & 0 & 1 & 0 & 0.5 & 0.5 &\\
\hline
\end{tabularx}
}
\end{table}






\section{Probabilistic Trace Alignment}
\texttt{\color{red}[TODO: introduction, motivation, and why it is useful to run it as a $k$-best search]}
\resizeableyellownote{3}{3}{
	Assuming that Rafael writes the definition of his ranking function, that can be expressed as $r(\tau^*,\tau)=d(\tau^*,\tau)w_\tau$ for a query trace $\tau^*$ and a trace $\tau\in\mathcal{W}_p^n(P)$
}

\subsection{Exact $k$-probabilistic Traces Alignment Problem}\label{subsec:exbkptap}
\texttt{\color{red}[TODO: the introduction of this section depends on how we want to formulate the problem. I suddenly start with the definition of the k-Nearest Neighbour, but I am aware that we need to provide first a bit of context]}

\begin{definition}[$k$-Nearest Neighbour]
Given a set of vectors\yellownote{Is this definition required, or an informal definition will do?}  $\mathcal{X}\subseteq \mathbb{R}^d$ within a $d$-dimensional Euclidean space and a query vector $q\in\mathbb{R}^d$, the $k$-nearest neighbour algorithm returns a subset $K\subseteq\mathcal{X}$ of $k$ elements minimizing the distance from $v$:
$$knn(k,q,\mathcal{X})=\begin{cases}
	\emptyset& k \leq 0\\
	\{c\}\cup knn(k-1,q,\mathcal{X}\backslash\{c\}) & k> 0 \Rightarrow c:={\arg\min}_{x\in\mathcal{X}}\|x-q\|_2\\
\end{cases}$$
\end{definition}

We can express our probabilistic trace match as finding the trace that maximizes both the trace's probability and its similarity with the query trace $\tau^*$. Still, the trace alignments problems are usually expressed via trace alignments cost functions, and not via trace similarities \cite{LeoniM17}. Given a generic trace cost function $d(\tau,\tau')$, it is always possible to convert it into a normalized similarity score $s_d(\tau,\tau'):=1/(d(\tau,\tau')+1)$, so that the maximum similarity of $1$ is reached when the distance is $0$ and the similarity decreases while the distance increases \cite{BergamiBM20}.

At this point, we can map each trace $\tau$ from $\braket{\tau,w_\tau}\in\mathcal{W}_p^n(P)$ that we need to align with $\tau^*$ as a point $\tau\overset{\mu_{\tau^*}}{\mapsto}(w_\tau,\; s_d(\tau,\tau^*))$ in the 2-dimensional similarity/probability space, so that the trace finding problem reduces to find a data point maximising the product $ps$ (Figure \ref{fig:spp}). At this stage, we can reduce the problem into a $k$-Nearest Neighbour search by providing a transformation $t$ such that the distance of $t(p,s)$ towards an arbitrary query point, e.g. the origin of the axes $\vec{0}$, corresponds to $\sfrac{1}{ps}$, so the transformed data points maximising $ps=k$ are all at the same distance $\sfrac{1}{k}$ from $\vec{0}$ (Figure \ref{fig:knnspace}). A possible transformation is the following:
\[t(p,s):=\left(\frac{1}{s\sqrt{p^2+s^2}},\; \frac{1}{p\sqrt{p^2+s^2}}\right)\]

\begin{example}
Figure \ref{fig:spp} shows a family of hyperbolae $ps=k$ describing all the points $(p,s)$ having $k$ as a weighted similarity score. For example,  point $\color{red}(1,1)$ represents the best possible trace match, as it means that there exists a trace $\braket{\tau,p}\in\mathcal{W}_p^n(P)$ with probability $p=1$ and trace similarity $s_d(\tau,\tau^*)=1$.

Figure \ref{fig:knnspace} shows that the transformation $t$ moves the points of the hyperbola $ps=k$ over a circumference $\overline{p}^2+\overline{s}^2=\sfrac{1}{k^2}$ describing a locus of the points equidistant from the origin of the axes $(0,0)$ with distance $\sfrac{1}{k}$. %This implies that now all the points $(p,s)$ having the same product $ps=k$ are equidistant from the origin of the axes, thus implying that we might now rank the points using a $k$-Nearest Neighbour algorithm using $q=\vec{0}$ as a query vector. 
This consideration will be formally proved in the subsequent lemma.
\end{example}

\begin{figure}[!t]
	\centering
	\subfloat[The similarity/probability space.]{\label{fig:spp}\includegraphics{images/original_space.pdf}}\qquad
	\subfloat[The transformed space for the $k$ nearest neighbours problem.]{\label{fig:knnspace}\includegraphics[scale=0.55]{images/transformed_space.pdf}}\\
	\caption{Two different characterizations of the probabilistic trace alignment problem. The best possible match is represented in red in both the similarity/probability space and in the transformed one.}
\end{figure}


 We can show with the next lemma that the following transformation is the one reducing the problem to the $k$-Nearest Neighbour problem:



\begin{lemma}
Given a value $k\in[0,1]\subseteq \mathbb{R}^+_0$, the set of points having the product $ps$ at least $k$ corresponts to the set of $t$-transformed points having a distance of at least $1/k$ from the origin of the axes.
\end{lemma}
\begin{proof}
\[\begin{aligned}
ps\geq k&\Leftrightarrow \frac{1}{ps}\leq\frac{1}{k} \\
	   &\Leftrightarrow \frac{\sqrt{p^2+s^2}}{ps\sqrt{p^2+s^2}}\leq\frac{1}{k} \\
	   &\Leftrightarrow \sqrt{\frac{p^2+s^2}{p^2s^2(p^2+s^2)}}\leq\frac{1}{k} \\
	   &\Leftrightarrow \sqrt{\frac{p^2}{p^2s^2(p^2+s^2)}+\frac{s^2}{p^2s^2(p^2+s^2)}}\leq\frac{1}{k} \\
	   &\Leftrightarrow \sqrt{\frac{1}{s^2(p^2+s^2)}+\frac{1}{p^2(p^2+s^2)}}\leq\frac{1}{k} \\
	   &\Leftrightarrow \left\|{\biggr({\frac{1}{s\sqrt{p^2+s^2}},\frac{1}{p\sqrt{p^2+s^2}}\biggr)}-\vec{0}}\right\|_2\leq\frac{1}{k} \\
	   &\Leftrightarrow \left\|t(p,s)-\vec{0}\right\|_2\leq\frac{1}{k} \\
\end{aligned}\]
\end{proof}
\begin{lemma}
Given a Petri Net $P$ and a trace $t^*$, the probabilistic trace alignment problem of the best $k$ traces reduces to the $k$-Nearest Neighbour problem $\mu_{\tau^*}^{-1}(knn(k,\vec{0},\mu_{\tau^*}(\mathcal{W}_0^{\aleph_0}(P))))$.
\end{lemma}
\begin{proof}
Trivial by definition of $\mu_{\tau^*}$ and for the previous lemma.
\end{proof}

At this stage, we can possibly solve the $k$-probabilistic trace alignment problem by generating a new instance of the $k$-Nearest Neighbour problem for each possible trace $\tau^*$ that we want to align towards the traces coming from a PPN. On the other hand, this solution might result quite costly, as solving the problem would require either to use a brute force search algorithm or to load and index our set of points each time.\yellownote{TODO: add references and explaination to the problem (we need a Related Work section\dots? I am accustumed to write such sections.)} In the next section we will discuss an approximated version of the problem providing a trade off between accuracy and efficiency.


\subsection{Approximate $k$-probabilistic Traces Alignment Problem}\label{subsec:akptap}
\texttt{\color{red}[TODO: provide some context and introduction. I'm facing the same problem as in the previous subsection, as the content in here depends on some content that I do not know if I have to write or not\dots]}

Given the characterization of a PPN as in \S\ref{subsec:ppn} and the embedding strategy proposed in Definition \ref{def:ppne}, we can generate an embedding for each possible weighted trace $\braket{\tau,w_\tau}\in\mathcal{W}_p^n(P)$ for a given PPN $P$ as described in the following definition:
\begin{definition}[Trace Embedding for PPNs]
Given a minimum probability threshold $p$, a maximum path length $n$, and a PPN $P=(s,t,L,R,w)$, we generate the set of the trace embeddings for $P$ as follows:
\begin{enumerate}
	\item for each weighted trace $\braket{\tau,w_\tau}\in\mathcal{W}_p^n(P)$ generated from a path $\pi_\tau=s\to n_2\rightsquigarrow n_m\to t$ over $R$, we generate a PPN $P_\tau=(s',t',L_\tau,R_\tau,w_\tau)$, where \begin{alphalist}
		\item $s'=s$ if $\textit{label}(s)\neq \varepsilon$ and $t'=n_2$ otherwise,
		\item $t'=t$ if $\textit{label}(t)\neq \varepsilon$ and $t'=n_m$ otherwise,
		\item $L_\tau$ (and $R_\tau$) is the submatrix of $L$ (and $R$) over the non-$\varepsilon$ labelled notes in $\pi_\tau$ and the labels from $\tau$,
		\item $w'$ is initialised by $w$ and then multiplied by $[R]_{s,n_2}$ (and also $[R]_{n_m,t}$) if $\textit{label}(s)=\varepsilon$ (and  $\textit{label}(t)=\varepsilon$);
	\end{alphalist}
	\item each $P_\tau$ is then represented as $\phi_{\mathcal{P}}(P_\tau)$ and added to the set $\mathbf{T}_p^n(P)$.
\end{enumerate}
\end{definition}

At this stage, the computation of $\underset{\braket{\tau,w_\tau}\in \mathcal{W}_p^n(P), P_\tau\in\mathbf{P}_p^n(P)}{\max\arg} k_{\phi_\mathcal{P}}(P_\tau, T)$ returns the best approximated trace alignment $\tau$ for a query trace represented as $T$. Similarly, we can provide the PPN $P\in\mathbf{P}$ providing the best approximated alignment for $T$ as $\underset{P}{\max\arg}\underset{ P_\tau\in\mathbf{P}_p^n(P)}{\max} k_{\phi_\mathcal{P}}(P_\tau, T)$. 

\begin{table}[!t]
\caption{Distinct USWNs and associated sets of unravelled traces from a single Sepsis Cases Event Log \cite{mannhardt_2016}.}\label{tab:dataset}
 \begin{adjustbox}{max width=\textwidth}
	\begin{tabular}{crl||cl|c}
		\toprule
		\textbf{Experiment Conf.} $(\mathcal{U})$ & \textit{Model} & $+$\textit{W. Estimator} & $n$ & $p_\theta$& $\;\;|\mathcal{W}_{p_\theta}^n(P_{\mathcal{U}})|$ \\
		\midrule
		
		\textbf{SM\_CONS\_20} &SplitMiner 2.0  \cite{AugustoCDRP19}       & +\texttt{Constant} & $20$ & $\;\;0$ & $157$  \\
		
		\textbf{SM\_FORK\_20} & SplitMiner 2.0  \cite{AugustoCDRP19}      & +Fork \cite{spdwe} & $20$ & $\;\;0$ & $32$  \\
		
		
		\textbf{SM\_PAIR\_20} & SplitMiner 2.0  \cite{AugustoCDRP19}      & +PairScale \cite{spdwe} & $20$ & $\;\;0$ & $157$ \\
		
		\textbf{SM\_PETRI\_20} & \multicolumn{2}{c||}{Rogge-Solti \cite{RoggeSoltiAW13}} & $20$ & $10^{-5}$ & $1612$ \\
		\bottomrule
	\end{tabular}
\end{adjustbox}
\end{table}
\section{Experimental Results}\label{sec:exp}
\subsection{Dataset}
For our experiments, we took the Sepsis Cases Event Log \cite{mannhardt_2016}, splitted the dataset into the ``\textit{happy traces}'' occurring approximately near to the trace average length ($\leq 2.3\cdot 10^{7}$ ms), and used this dataset to generate either a USWN via ProM \cite{RoggeSoltiAW13} and a BPMN with only exclusive gates using Split Miner 2.0 \cite{AugustoCDRP19} that was then converted into a Petri Net \cite{PPNFromLog}. Such Petri Net was later on converted into a USWN by using a firing weight estimator: we choose the Fork and the PairScale estimators from \cite{spdwe} and we denote as \texttt{Constant} a naïve estimator assuming that each all the transition probabilities from a given marking are equiprobable. No estimator was used for the USWN generated via ProM, as such engine already estimates the firing weights. From such USWNs, we generate distinct sets of unravelled traces. All these steps are summarised in Table \ref{tab:dataset}. The following experiments have the aim of evaluating the benefits of performing the Approximate-Ranking strategy over the Optimal-Ranking one.

\begin{figure*}[!t]
\begin{minipage}{.49\textwidth}

	\includegraphics[width=1.1\textwidth]{images/Prec.pdf}
	\caption{Approximation comparison.}\label{fig:app}
\end{minipage}\hfill \begin{minipage}{.49\textwidth}

	\includegraphics[width=1.1\textwidth]{images/kronos.pdf}
	\caption{$k$NN alignment benchmark.}\label{fig:kronos}
\end{minipage}
\end{figure*}
\subsection{Approximation}\label{subsec:apprp}
In order to assess how well the proposed Approximate-Ranking strategy approximates the Optimal-Ranking one, we use the Spearman Correlation Index \cite{} to express the correlation between each sub-embedding strategies for $\phi_{\mathcal{P}}$ ($\color{ggplotGreen}\epsilon^1\&\nu^1$, $\color{ggplotRed}\epsilon^2\&\nu^1$, $\color{ggplotPurple}\epsilon^1\&\nu^2$, and $\color{ggplotBlue}\epsilon^2\&\nu^2$) and the Optimal-Ranking one.
In general, we can see form the plots that when aligning longer traces, the correlation of the approximate rankings with the optimal one is lower. This is due to the fact that, in this case, the embedding representing the trace to align is longer and this implies also a larger approximation error. We can also observe that the sub-embeddings considering only information about the edges (i.e., the one where the features corresponding to the $\nu$ dimension are set to zero) have in general a higher correlation with the Optimal-Ranking strategy, but their correlation values are less stable with respect to the length of the trace to be aligned. In the case of \textbf{SM\_FORK\_20}, the correlation is maximum for all sub-embedding strategies.

%set of unravelled traces in Table \ref{tab:dataset} and the subset of the Sepsis Cases Event Log that was not used to generate the USWNs. For each of this log trace $\sigma^*$ we added controlled noise (transition addition, deletion, or swap) at either $20\%$ ($\tilde{\sigma}^*$) or $30\%$ ($\tilde{\tilde{{\sigma}}}^*$) of the log trace as for \cite{LeoniM17}. Then, we found the correlation between the ranking $R_\star$ induced by $k_{\phi_{\mathcal{P}}}(\sigma,\sigma^*)$ to the ranking induced by replacing $\sigma^*$ with one of the two noised traces (either a ranking $R_{20}$ induced by $k_{\phi_{\mathcal{P}}}(\sigma,\tilde{\sigma}^*)$ or $R_{30}$ induced by $k_{\phi_{\mathcal{P}}}(\sigma,\tilde{\tilde{\sigma}}^*)$). The correlation $\rho$ between these two rankings ($\rho(R_\star,R_{20})$ and $\rho(R_\star,R_{30})$) is performed via Spearman Correlation Index $\rho$: such index will return near-$1$ on increasing monotonic trend, near-$(-1)$ values on decreasing monotonic trend, and near-$0$ values where the two rankings are almost uncorrelated. Figure \ref{fig:app} shows the outcome of such experiments for all the possible combinations of $\epsilon$ and $\nu$ sub-embeddings for $\phi_{\mathcal{P}}$ while varying the log trace length. We can observe that strategies including traces' frequencies ($\nu^1$) are more stable if compared to strategies where such information is completely ignored ($\nu^2$). Furthermore, such approximation never reaches zero values, while that occurrence might happen for $\nu^2$-based strategies.

\subsection{Efficiency}\label{subsec:efficio}
With reference to the plots in Figure \ref{fig:kronos}, 
we evaluated the efficiency of computing the trace alignment over both Optimal- ({\color{ggplotPurple}Purple} and {\color{ggplotGreen}Green}) and Approximated-ranking ({\color{ggplotBlue}Blue} and {\color{ggplotRed}Red}) strategies over two different data structures enabling $k$NN queries, i.e., VP-Trees ({\color{ggplotBlue}Blue} and {\color{ggplotPurple}Purple}) and KD-Trees ({\color{ggplotRed}Red} and {\color{ggplotGreen}Green}). We conduct our experiments for $k=20$\ADD{, and we use the Levenshtein distance as a baseline distance for the alignment cost function}. While the query time for the Optimal-Ranking  includes the additional \textit{indexing time} for generating all the vectors to the neighbourhood search, the Approximated-Ranking  adds the neighbourhood search time with the embedding transformation of $x$; in particular, we consider the average embedding time for all the possible embedding strategies described in the previous experiment setting. Figure \ref{fig:kronos} plots the result of such experiments: the time required to generate and load all the $\phi_\star$ truly dominates the cost of generating the embedding $\phi_{\mathcal{P}}(x)$ for bigger datasets such as \textbf{STPETRI\_20}, while the cost for $\phi_{\mathcal{P}}(x)$ becomes non-negligible when we have just a few traces to align (\textbf{SM\_FORK\_20}). Last, while the $k$NN alignments over $\phi_{\mathcal{P}}$ always provide the best timing results via KD-Trees, we cannot elect a best data structure $\phi_\star$ that is valid for all datasets and all trace lengths. 
%\section{Reasoning with Constraint Scenarios}
Constraint scenarios can be used to perform a variety of tasks. We focus here on two fundamental ones: conformance checking and probabilistic constraint entailment.

\subsection{Conformance Checking}
In \declare, the simplest form of conformance checking amounts to check whether a given execution trace satisfies all constraints contained in the model, thus returning a yes/no answer.

In \pdeclare, this notion can be refined by considering the different constraint scenarios and their probabilities. Let $\M = \tup{\tasks,\cset}$ be a \pdeclare model, and $\tau$ be a trace over $\tasks$. The plausible scenarios of $\M$ are pairwise disjoint subsets of the overall set $\tasks^*$ of traces over $\tasks$. Disjointness comes from the fact that every pair of plausible scenarios is so that they disagree about at least one constraint, and no trace can conform with both of them. The complement of the traces accepted by the plausible scenarios then characterizes those traces that are not conforming with $\M$. To assess conformance, we can then proceed as follows:
\begin{inparaenum}[(1)]
\item Check $\tau$ against every plausible scenario of $\M$.
\item If one plausible scenario is so that $\tau$ holds there, output \emph{yes} together with the probability (or range of probabilities) attached to that scenario; the scenario probability gives an indication on whether the trace represents a ``mainstream'' execution of the process, or is instead an outlier behavior.
\item If no such scenario is found, then output \emph{no}.
\end{inparaenum}

\begin{example}
  Consider the \pdeclare model captured by \emph{Case 1} in Table~\ref{tab:cases}. Trace $\tup{\accept,\cancel,\pay}$ does not conform with the model, since paying and canceling are mutually exclusive. Trace $\tup{\accept,\cancel}$ is instead conforming, as it satisfies scenario $1001$. Since this scenario is associated with probability $0.1$, the analyzed trace represents an outlier behavior. Finally, trace $\tup{\accept, \pay, \ship, \ship, \pay, \ship}$ represents a mainstream behavior since it conforms with the most likely scenario $1010$, with probability $0.5$.
\end{example}

\subsection{Constraint Entailment}
It is well-known that \declare and other declarative process modeling languages have the issue of \emph{hidden dependencies} \cite{MPVC11}, namely the fact that constraints may interact with each other in subtle ways. This becomes even more complex in the case of probabilistic constraints. In this light, it becomes crucial to be able to ascertain whether a constraint is implied by a given model. Checking constraint implication in \declare is very simple: this simply amounts to check whether the \LTLf formula of the model implies the given constraint. In the case of \pdeclare, we extend this approach by computing, for a given \LTLf formula, what is the probability with which it is implied by the \pdeclare model. This is done as follows:
\begin{inparaenum}[(1)]
\item Initialize the constraint probability range to $0,0$.
\item For every plausible scenario, check whether the scenario implies the formula of interest in the classical \LTLf sense; if so, update the constraint probability by summing its minimum and maximum to the minimum and maximum probability associated with the scenario.
\item Return the constraint probability range.
\end{inparaenum}

\begin{example}
  Consider again the \pdeclare model captured by \emph{Case 1} in Table~\ref{tab:cases}. We want to check to what extent the model implies that the order is eventually shipped ($\Diamond \ship$). Shipment only occur if a payment occurs before, and therefore this formula is implied only by scenario $1010$, consequently getting a probability of $0.5$.

  We are also interested in checking to what extent the model implies that the order is not rejected ($\neg \Diamond \reject$). This formula holds in all those scenarios where $\constraint{existence}(\reject)$ is false. Therefore, this formula is implied with probability $0.9$.

  Finally, consider the \LTLf constraint $\neg(\Diamond \cancel \land \Diamond \ship)$, expressing the mutual exclusion between \cancel\ and \ship. This constraint is implied with probability $1$, due to the presence of the two crisp constraints $\constraint{not-coexistence}(\cancel,\pay)$ and $\constraint{precedence}(\pay,\ship)$, which must hold in every possible scenario (including the plausible ones).
\end{example}







%\section{Discovering \pdeclare Models from Event Logs}
\label{sec:discovery}
We now show that \pdeclare models can be discovered from event data using, off-the-shelf, already existing techniques, with a quite interesting guarantee: that the discovered model is always consistent.
%
We use the standard notation $[\cdot]$ for multisets, and use superscript numbers to identify the multiplicity of an element in the multiset.

A plethora of different algorithms have been devised to discover \declare models from event data \cite{LMMR07,MaCV12,CiccioM15,DBLP:conf/caise/SchonigRCJM16}. In general, the vast majority of these algorithms adopt the following approach to discovery:
\begin{inparaenum}[(1)] 
\item  Candidate constraints are generated by analyzing the activities contained in the log. 
\item For each constraint, its \emph{support} is computed as the fraction of traces in the log where the constraint holds.
\item Candidate constraints are filtered, retaining only those whose support exceeds a given threshold. 
\item Further filters (e.g., considering the ``relevance'' of a constraint \cite{DMMM18}) are applied. 
\item The overall model is checked for satisfiability, operating with different strategies if it is not; this is necessary since constraints with high support, but less than $1$, may actually conflict with each other \cite{DMMM17}.
\end{inparaenum}
%
In this procedure, the notion of support is formalized as follows.
\begin{definition}
The \emph{support} of an \LTLf constraint $\varphi$ in an event log $\mathcal{L} = [\tau_1, \dots,  \tau_n]$ is
\begin{equation*}
\mathit{supp}_{\mathcal{L}}(\varphi) = \frac{\vert\mathcal{L}_{\varphi}\vert}{\vert \mathcal{L}\vert}, \text{ where }\mathcal{L}_{\varphi} = [\tau \in \mathcal{L} \mid \tau \models \varphi]
\end{equation*}
\end{definition}
%
By the semantics of probabilistic constraints, we can adopt this approach off-the-shelf to discover \pdeclare constraints: \emph{we just use the constraint support as its associated probability, with operator $=$}. In other words, if $\varphi$ is discovered with support $p$, we turn it into the probabilistic constraint $\tup{\varphi,p}$. When doing so, we can also relax step (3), e.g., to retain constraints with a very low support, implying that their negated versions have a very high support.

\begin{example}
  Consider $\L = [\tup{\activity{close},\activity{acc}}^7,\tup{\activity{close},\activity{ref}}^2,\tup{\activity{close},\activity{acc},\activity{ref}}^1]$, capturing the evolution of 10 orders, 7 of which have been closed and then accepted, 2 of which have been closed and then refused, and 1 of which has been closed, then accepted, then refused. The support of constraint \constraint{response}(\activity{close},\activity{acc}) is $8/10 = 0.8$, witnessing that 8 traces satisfy such a constraint, whereas 2 violate it. This corresponds exactly to the interpretation of probability $0.8$ for the probabilistic \constraint{response}(\activity{close},\activity{acc}) constraint in Figure~\ref{fig:scenarios}. More in general, the entire \pdeclare model of Figure~\ref{fig:scenarios} can be discovered from $\L$ by considering the 6 constraints contained in that model and their corresponding support over $\L$.
\end{example}

A second key observation is that once this procedure is used to discover \pdeclare constraints, step (5) is unnecessary: the overall discovered model is in fact guaranteed to be satisfiable (in our probabilistic sense).
%
\begin{theorem}
Let $\Sigma$ be a set of activities, $\mathcal{L}$ be an event log over $\Sigma$,
and
$\mathcal{C}=\set{\tup{\varphi_1,p_1},\ldots,\tup{\varphi_n,p_n}}$ be a set of probabilistic constraints discovered from $\mathcal{L}$, such that for each $i \in \set{1,\ldots,n}$, $p_i = \mathit{supp}_{\mathcal{L}}(\varphi_i)$.
 The \pdeclare model $\tup{\Sigma,\mathcal{C}}$ is satisfiable.
\end{theorem}
%
\begin{proof}
Recall that $\tup{\Sigma,\mathcal{C}}$ is satisfiable if its corresponding \PLTLz formula
$\Phi:=\{\prob{p_1}\varphi_1,\ldots,\prob{p_n}\varphi_n\}$ is satisfiable.
To show this, we simply use $\mathcal{L}$ to build a model of $\Phi$.
%
For every set $I\subseteq \{1,\ldots,n\}$, let
$\varphi_I$ be the $\text{LTL}_f$ formula
\[
\varphi_I := \bigwedge_{i\in I} \varphi_i \land \bigwedge_{i\notin I}\neg\varphi_i,
\]
and let $\mathcal{L}_I$ be the sublog of $\mathcal{L}$ containing all the traces that satisfy $\varphi_I$. Note
that the sublogs $\mathcal{L}_I$ form a partition of $\mathcal{L}$; that is, every trace appears in exactly one
such $\mathcal{L}_I$.
%%
For each $I$ such that $\mathcal{L}_I$ is not empty, choose a representative $t_I\in\mathcal{L}_I$ and let
$p_I:=\frac{\vert \mathcal{L}_I \vert}{\vert \mathcal{L} \vert}$ be the fraction of traces that belong to $\mathcal{L}_I$. 
We build a stochastic language $\rho$ by setting $\rho(t_I)=p_I$ for each $I$ such that $\mathcal{L}_I\not=\emptyset$
and $\rho(\tau)=0$ for all other traces. We need to show that $\rho$ satisfies $\mathcal{C}$.
%%
Consider a constraint $\left<\varphi,p\right>\in\mathcal{C}$; we need to show that $\sum_{\tau\models\varphi}\rho(\tau)=p$.
Note that by construction, $\sum_{\tau\models\varphi}\rho(\tau)=\sum_{t_I\models\varphi}p_I$ and
since $\mathcal{L}_I$ form a partition the latter is in fact the fraction of traces that satisfy
$\varphi$. On the other hand, $p$ is also the
support of $\varphi$; that is, the proportion of traces satisfying $\varphi$. Hence, both values are equal, and $\rho$
satisfies the \pdeclare model.
\qed
\end{proof}
By this theorem, probabilistic constraints can be discovered in a \emph{purely local} way, having the guarantee that they will never conflict with each other. Obviously, non-local filters can still prove useful to prune implied constraints and select the most relevant ones.
Also, note that the probabilities of the discovered constraints can be easily adjusted when new traces are added to the log, by incrementally recomputing the support values after checking how many new traces satisfy the various constraints.

%The main open question related to the discovery of probabilistic constraints, which we intend to tackle in future work, is to determine when to stop the discovery procedure, and what is the impact of retaining constraints with various degrees of support in terms of over/under-fitting. 
%\newcommand{\formula}[1]{\mathit{formula}(#1)}
\newcommand{\probability}[1]{\mathit{prob}(#1)}

\section{Monitoring Probabilistic Constraints}
\label{sec:monitoring}
In Section~\ref{sec:scenarios}, we have shown how we can take a \pdeclare model and generate its constraint scenarios, together with their corresponding probability intervals. We now describe how this technique can be directly turned into an operational probabilistic monitoring and conformance checking framework.

Let $M = \tup{\Sigma,\mathcal{C}}$ be a \pdeclare model with $n$ probabilistic constraints. For simplicity, we do not distinguish between crisp and genuinely probabilistic constraints, nor prune away implausible scenarios: the produced monitoring results do not change, but obviously our implementation, presented at the end of this section, takes into account these aspects for optimization reasons.
%
$M$ generates $2^n$ constraint scenarios. As discussed in Section~\ref{sec:scenarios}, each scenario $S$ comes with a corresponding characteristic \LTLf formula, which amounts to the conjunction of positive and negated constraints in $\mathcal{C}$, where the decision of which ones are taken positive and which negative is defined by the scenario itself. We denote such a formula by $\formula{S}$. For example, if $\mathcal{C} = \set{\tup{\varphi_1,p_1},\tup{\varphi_2,p_2},\tup{\varphi_3,p_3}}$,  then $\formula{S_{101}}=\varphi_1 \land \neg \varphi_2 \land \varphi_3$.
%
In addition, each scenario $S$ comes with its probability interval, calculated by minimizing and maximizing its probability variable in the system of inequalities $\Lmc_M$. We denote such a probability interval by $\probability{S}$. From Example \ref{ex:intervals}, we have, e.g., that $\probability{S_{10}}= [0.7,0.8]$.

Since the characteristic formula of a scenario is in standard $\LTLf$, we can construct a \emph{scenario monitor} by recasting well-known techniques \cite{MMWV11,DDGM14}. Specifically, given an $\LTLf$ formula $\varphi$ over a set $\Sigma$ of activities, and a partial trace $\tau$ representing an ongoing process execution, a monitor outputs one of the four following truth values:
\begin{compactitem}[$\bullet$]
\item $\tau$ \emph{(permanently) satisfies} $\varphi$, if $\varphi$ is currently satisfied ($\tau \models \varphi$), and $\varphi$ stays satisfied no matter how the execution continues, that is, for every possible continuation trace $\tau'$ over $\Sigma$, we have $\tau \cdot \tau' \models \varphi$ (the $\cdot$ operator denotes the concatenation of two traces);
\item $\tau$ \emph{possibly satisfies} $\varphi$, if $\varphi$ is currently satisfied ($\tau \models \varphi$), but $\varphi$ may become violated in the future, that is, there exists a continuation trace $\tau'$ over $\Sigma$ such that $\tau \cdot \tau' \not\models \varphi$;
\item $\tau$ \emph{possibly violates} $\varphi$, if $\varphi$ is currently violated ($\tau \not\models \varphi$), but $\varphi$ may become satisfied in the future, that is, there exists a continuation trace $\tau'$ over $\Sigma$ such that $\tau \cdot \tau' \models \varphi$;
\item $\tau$ \emph{(permanently) violates} $\varphi$, if $\varphi$ is currently violated ($\tau \not\models \varphi$), and $\varphi$ stays violated no matter how the execution continues, that is, for every possible continuation trace $\tau'$ over $\Sigma$, we have $\tau \cdot \tau' \not\models \varphi$.
\end{compactitem}
In \cite{MMWV11,DDGM14}, it is shown that a monitor producing such outputs can be seamlessly obtained by constructing and determinizing the finite-state automaton $\aut_\varphi$ for $\varphi$, and assigning each automaton state to one of the four truth values.

%\input{monitor}
\begin{figure*}[t!]
	\centering	
		\begin{tabular}{ c  c  }
			\makecell{			
				\includegraphics[width=0.5\textwidth]{images/LTL2Automaton_01_log_c_r.png}			
			}
			&
			\makecell{		
				\includegraphics[width=0.5\textwidth]{images/LTL2Automaton_01_log_r_c.png}			
			} \\
			 trace $\tup{\activity{close},\activity{sign}}$ & trace $\tup{\activity{sign},\activity{close}}$
		\end{tabular}
	\caption{Output of the implemented tool on the example in \figurename\ref{fig:scenarios-2}.}
	\label{fig:ex0t1}
\end{figure*}

We proceed as follows. For each plausible constraint scenario $S$ over $M$, we construct the finite-state automaton $\aut_{\formula{S}}$, and turn it into a monitor as described above.\footnote{Implausible scenarios are irrelevant, since they produce an output that is associated to probability $0$, and would be therefore discarded when computing the aggregated probability intervals.} We then track the evolution of a running trace by delivering its events to \emph{all} such monitors in parallel, returning the truth values they produce. Note that, at runtime, we do not know in which scenario the trace will fall when completed. Hence, we actually do not know the exact truth value of the trace. For this reason, we compute the truth value of the trace probabilistically by aggregating the probabilities of the scenarios that produce the same truth value. In particular, we compute the \emph{aggregated probability interval} for each truth value, by taking the system of inequalities $\Lmc_M$ and calculating the extreme values of the aggregated interval by minimizing/maximizing the sum of the probability variables associated to the scenarios that produce that truth value. The aggregated probabilities give an indication of how probable it is that the trace can be associated to each specific truth value.

When a trace ends (which is signalled by a special, \emph{complete} event) all monitors currently outputting \emph{possible satisfaction} turn to \emph{permanent satisfaction}, and those outputting \emph{possible violation} turn to \emph{permanent violation}. Hence, when a trace ends, it either permanently violates all scenarios (thus being classified as a non-conforming one), or permanently violates all scenarios but one, which is permanently satisfied and consequently witnesses that the trace belongs to that scenario. Notably, this can be instrumental to classify traces into process variants.

%Interestingly, all monitors may agree on violation outputs, witnessing the non-conformance of the monitored trace. If, instead, monitors produce different outputs, their corresponding aggregated probabilities give a probabilistic indication of conformance that depends on how likely it is for the monitored trace to belong to the different plausible scenarios. The more the trace evolves, the more it becomes clear to which scenario it belongs, and in turn whether it configures itself as a common trace (belonging to a highly plausible scenario) or as an outlier one (belonging to a less plausible scenario).


\begin{example}
  Consider the \pdeclare model in Figure~\ref{fig:scenarios} with its three plausible scenarios (recall that four scenarios are logically plausible there, but one of those has probability $0$, so only three remains to be monitored). Figure~\ref{fig:monitoring} shows the result produced when monitoring a trace that at some point appears to belong to the most plausible scenario, but in the end turns out to conform to the less plausible one. From the image, we can also clearly see that the trace consisting only of a close order activity would be judged as non-conforming, as it would violate all scenarios.
\end{example}

This probabilistic monitoring technique has been fully implemented.\footnote{\url{https://bitbucket.org/fmmaggi/probabilisticmonitor/src/master/}} For solving systems of inequalities, we use the Java based LP solver\footnote{\url{http://lpsolve.sourceforge.net/5.5/}}. The implementation comes with various optimizations. First, scenarios are computed by directly imposing that crisp constraints with probability $1$ must hold in their positive form in all scenarios. Second, only plausible scenarios are retained for monitoring. Third, the results obtained by minimizing and maximizing for aggregate probability variables are cached, to avoid solving multiple times the same problem.
\figurename \ref{fig:ex0t1} shows the output of the implemented monitoring tool on the example in \figurename \ref{fig:scenarios-2} and for two different traces.\footnote{In the screenshots, 1 and 2 represent the probabilistic constraints labeled with 1 and 2 in \figurename \ref{fig:scenarios-2}, whereas 3 represents the crisp constraint in the same example.} Here, the aggregated probability intervals are shown with a dark gray or light gray background depending on whether their midpoint is closer to 1 or to 0, respectively, thus giving an indication of the most probable truth values for the monitored trace. The first trace (on the left) is classified as belonging to scenario $S_{01}$ and is an outlier because this scenario has low probability (corresponding to a probability interval of $\probability{S_{01}}= [0.0,0.1]$). The second trace (on the right) is classified as belonging to the highly plausible scenario $S_{10}$ (corresponding to a probability interval of $\probability{S_{10}}= [0.7,0.8]$).






% !TeX root=../main.tex

\section{Conclusions}
%\texttt{\color{red}[TODO]}
\label{sec:conclusion}

In this paper, we have presented an approach to tackle the probabilistic trace alignment as a $k$NN problem.
The approach balances between the likelihood of the aligned trace and the cost of the alignment by providing the top-k alignments instead of a single alignment as output. The experimentation shows that the approximated top-k ranking provides a good trade-off between accuracy and efficiency especially when the reference stochastic net generates several model traces.
Future works will investigate the probabilistic alignment over fuzzy-labeled nodes and declarative process models, thus allowing to generalize the proposed approach to noisy walks in plan recognition for non-rational agents \cite{RamirezG10}. Also, we will try to improve the performance (in terms of efficiency and accuracy) of the proposed approach by intervening both on the embedding and the algorithmic strategies, such as extending the Viterbi Algorithm  to perform $k$ candidates instead of the one maximizing probability and similarity value. Last, we will also investigate the possibility of representing POMDPs as Transition Graphs when the reward function is completely determined by the distance between traces' actions.

%\section*{Acknowledgements}
%This research has been partially supported by the project IDEE (FESR1133) funded by the Eur.\ Reg.\ Development Fund (ERDF) 
%Investment for Growth and Jobs Programme 2014-2020. 
%\input{sections/declarereasoning}


\bibliographystyle{splncs04}
\bibliography{biblio3}
%,bibliography2,references,libraryFiltered,main,Bibliography-CDC,DiCiccio,biblio}


\end{document}
